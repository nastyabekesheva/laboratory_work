\documentclass[a4paper,12pt]{article}
\usepackage[russian,english]{babel}
\usepackage{ucs}
\usepackage[utf8]{inputenc}
\usepackage[T2A]{fontenc}
\usepackage{graphicx}
\usepackage{wrapfig}
\usepackage{amsfonts}
\usepackage{amsthm}
\usepackage{amssymb}
\usepackage{amsmath}
\newtheorem{theorem}{Theorem}
\theoremstyle{remark}
\newtheorem*{remark}{Remark}
\begin{document}
\title{Математический анализ 2}
\maketitle
\date{}
\catcode`@=11
\def\caseswithdelim#1#2{\left#1\,\vcenter{\normalbaselines\m@th
  \ialign{\strut$##\hfil$&\quad##\hfil\crcr#2\crcr}}\right.}% you might like it without the \strut
\catcode`@=12
%
\def\bcases#1{\caseswithdelim[{#1}}
\def\vcases#1{\caseswithdelim|{#1}}
%
\newpage
\tableofcontents
\newpage
\section{Неопределенный интеграл}
\subsection{Понятие первообразной и неопределеного интеграла}
Функция $F(x)$ называется \textbf{первообразной} функции $f(x)$ на множестве $X\subset \mathbb{R}$, или $\forall x\in X\>\>F'(x)=f(x)$.
\begin{theorem}
Если $F(x)$ - некоторая первообразная для $f(x)$ на множестве $X$, то любая другая первообразная имеет вид: $F(x)+c$, где $c=const$ - произвольная.
	
\end{theorem}
\begin{proof}
Пусть $F(x)$ - первообразная функции  $f(x)$, т.е.  $F'(x)=f(x)$; Тогда: $(F(x)+c)'=F'(x)+0=f(x)\Rightarrow F(x)+c$ - первообразная для $f(x)(c=const)$.\\ 
Пусть $F_1(x)$ - тоже первообразная для $f(x)$, т.е. $F_1'(x)=f(x)$.\\
Рассмотрим разность: $F_1(x)-F(x);$\\
$(F_1(x)-F(x))'=F_1'(x)-F'(x)=f(x)-f(x)=0\Rightarrow F_1(x)-F(x)=c=const$, т.е. :$F_1(x)=F(x)+c$
\end{proof}
Таким образом множество всех первообразных функции $f(x)$ имеет вид $F(x)+c$.\\
Множество всех первообразных функции $f(x)$ называется \textbf{неопределенным интегралом} этой функции и обозначается $\int f(x)dx$.\\
$f(x)$ - подинтегральная функция, $f(x)dx$ - подинтегральное выражение, $x$ - переменная инегрирования, $\int$ - неопределенный интеграл.\\ \underline{Пример:}\\
$${
f(x) = sign(x) = \bcases{1,\>\> x>0\cr0,\>\> x=0\cr -1,\>\> x<0},\>\>x\in(-1;\>1).
}$$
Предположим, что существует такая первообразная $\exists F(x):\> \forall x\in(-1;\>1):\>\>F'(x)=sign(x)$, т.е.
$${
F'(x)=  \bcases{1,\>\> x\in(0;\>1)\cr0,\>\> x=0\cr -1,\>\> x<\in(-1;\>0)}\Rightarrow \begin{cases}
F'(x)=0\cr F_-'(x)=-1\cr F_+'(x)=1
\end{cases}\Rightarrow
}$$ 
$F'(0)$ - не существует $\Rightarrow$ противоречие $\Rightarrow F(x)$ не существует.
\begin{remark}
Достаточным условием существования первообразной у функции  на данном множестве является ее непрерывность на этом множестве.
\end{remark}
\newpage
\subsection{Свойства неопределенного интеграла}
Пусть $\int f(x)dx=F(x)+c\>(F'(x)=f(x))$.
\begin{enumerate}
	\item Производная от неопределенного интеграла равна подинтегральной функции, дифференциал неопределенного интеграла равен подинтегральному выражению. \\
	$$(\int f(x)dx)'_x=f(x);\>\> d(\int f(x)dx)=f(x)dx.$$
	\begin{proof}
		$(\int f(x)dx)'_x=(F(x)+c)'_x=F'(x)+c'=f(x);$	\\$d(\int f(x)dx)=(\int f(x)dx)'_xdx=f(x)dx$.
	\end{proof}
	\item Неопределенный интеграл от дифференциала некоторой функции равен сумме этой функции и произвольной постоянной.
	$$\int d(F(x))=F(x)+c.$$
	\begin{proof}
		$\int d(F(x))=\int F'(x)dx=\int f(x)dx=F(x)+c$
	\end{proof}
	\item Постоянный множитель можно выносить за знак интеграла.
	$$\int af(x)dx=a\int f(x)dx,\>\> a=const.$$
	\begin{proof}
		$\int af(x)dx=\int a F'(x)dx=\int(aFx)'dx=\int d(aF(x))=(aF(x)++c_1)=a(F(x)+\frac{c_1}{a}=\Big|c=\frac{c_1}{a}\Big|=a(f(x)+c)=a\int f(x)dx$
	\end{proof}
	\item Интеграл суммы двух функций равен сумме интегралов этих функций.
	$$\int(f(x)+g(x))dx=\int f(X)dx+\int g(x)dx.$$
	\begin{proof}
		Пусть $\int g(x)dx=G(x)+c$; тогда $\int(f(x)+g(x))dx=\int(F'(x)++G'(x))dx=\int(F(x)+G(x))'dx=\int d(F(x)+G(x))=F(x)+G(x)++c=\Big|c=c_1+c_2\Big|=(F(x)+c_1)+(G(x)c_2)=\int f(x)dx+\int g(x)dx$
	\end{proof}
\end{enumerate}
\begin{remark}\
	\begin{enumerate}
		\item[-]свойство 4 справедливо для любого конечного числа слогаемых 
		\item[-]свойство 3-4 называются свойством линейности неопределенного интеграла 
		\item[-]свойство 1-2 отражают связь операций дифференцирования и интегрирования  
	\end{enumerate}		
\end{remark}
\subsection{Таблица основных неопределенных интегралов}
\begin{enumerate}
	\item $$\int x^\alpha dx=\frac{x^{\alpha+1}}{\alpha+1}+c;\>\>\alpha\in\mathbb{R}\backslash\{-1\}$$
	\item $$\int\frac{dx}{x}=\ln|x|+c$$
	\item $$\int a^xdx=\frac{a^x}{\ln a}+c,\>\>a>0$$
	\item $$\int e^xdx=e^x+c$$
	\item $$\int\sin x\>dx	=-\cos x+c$$
	\item $$\int\cos x\>dx=\sin x+c$$
	\item $$\int\frac{dx}{\cos^2dx}=\tan x+c$$
	\item $$\int\frac{dx}{\sin^x}=-\cot x+c$$
	\item $$\int\textrm{sh}x\>dx=\textrm{ch}\>x+c$$
	\item $$\int\textrm{ch}x\>dx=\textrm{sh}\>x+c$$
	\item $$\int\frac{dx}{\textrm{ch}^2x}=\textrm{th}\>x+c$$
	\item $$\int\frac{dx}{\textrm{sh}^2x}=-\textrm{cth}\>x+c$$
	\item $$\int\frac{dx}{\sqrt{1-x^2}}=\arcsin x+c$$
		$$\int\frac{dx}{\sqrt{a-x^2}}=\arcsin \frac{x}{a}+c$$
	\item $$\int\frac{dx}{1+x^2}=\arctan x+c$$
		$$\int\frac{dx}{a^2+x^2}=\frac{1}{a}\arctan\frac{x}{a}+c$$
		Дополнительные формулы: 
	\item $$\int\frac{dx}{x^2-a^2}=\frac{1}{2a}\ln|\frac{x-a}{x+a}|+c\textrm{ - высокий логарифм}$$ 
	\item $$\int\frac{dx}{\sqrt{x^2+A}}=\ln|x+\sqrt{x^2+A}|+c\textrm{ - длинный логарифм }$$
	\item $$\int\sqrt{x^2+A}\>dx=\frac{x}{2}\sqrt{x^2+A}+\frac{A}{2}\ln|x+\sqrt{x^2+A}|+c$$
	\item $$\int\sqrt{a^2-x^2}\>dx=\frac{x}{2}\sqrt{a^2-x^2}+\frac{a^2}{2}\arcsin\frac{x}{a}+c$$
\end{enumerate}
В этих формулах вместо $x$ может быть записана произвольная дифференцируемая функция от $x$.
\subsection{Основные примеры интегрирования}
\subsubsection{Непосредственное интегрирование}
Непосредственное интегрирование заключается в использовании тождественных преобразований подинтегральнной функции, свойства линейности интеграла и таблицы интегралов.\\
\underline{Пример:} 
\begin{enumerate}
	\item $\int(\frac{\sqrt{x}+1}{\sqrt[3]{x}})^2dx=\int\frac{x+2x^{\frac{1}{2}}+1}{\sqrt[3]{x^2}}dx=\int(x^{\frac{1}{3}}+2x^{\frac{-1}{6}}+x^{\frac{-2}{3}})dx=\frac{3}{4}x^{\frac{4}{3}}+2\cdot\frac{x^{\frac{5}{6}}}{5}\cdot 6+x^{\frac{1}{3}}\cdot 3+c$
	\item $\int\frac{dx}{\sin^2x\cos^2x}=|\cos^2x+\sin^2x=1|=\int\frac{\cos^2x+\sin^2x}{\sin^2x\cos^2x}dx=\int\frac{1}{\sin^2x}dx+\int\frac{1}{\cos^2x}dx=\tan x-\cot x+c$
\end{enumerate}
\subsubsection{Замена переменной}
\begin{theorem}
Пусть на $\forall x\in(a;\>b)\>\>\int f(x)dx=F(x)+c$, (на всем интервале $(a;\>b)$ известна первообразная функции): $F'(x)=f(x)\>\>x=\varphi(t)$ - функция дифференцируемая; причем $\varphi(t):\>t\in(\alpha;\>\beta)$ и $\varphi:(\alpha;\>\beta)\rightarrow(a;\>b)$.
Тогда справедлива формула:
$$\int f(\varphi(t))\cdot \varphi_t'(t)dt=F(\varphi(t))+c$$	
\end{theorem}
\begin{proof}
	$(f(\varphi(t)))_t'=F_\varphi'(\varphi(t))\cdot\varphi_t'(t)=|\varphi(t)=x|=F_x'(x)\varphi_t'(t)=|F_x'(x)=f(x)|=f(x)\cdot\varphi(t)=|x=\varphi(t)|=f(\varphi(t))\cdot\varphi_t'(t)\Rightarrow F(\varphi(t))\textrm{ первообразная для } \\f(\varphi(t))\cdot\varphi_t'(t)\Rightarrow\int f(\varphi(t))\cdot\varphi_t'(t)dt=F(\varphi(t))+c$
\end{proof}








\end{document}