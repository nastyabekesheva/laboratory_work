\documentclass[a4paper,12pt]{article}

\usepackage[ukrainian,english]{babel}
\usepackage{ucs}
\usepackage[utf8]{inputenc}
\usepackage[T2A]{fontenc}

\usepackage{amsmath}
\usepackage{amsfonts}
\usepackage[document]{ragged2e}
\usepackage{graphicx}
\usepackage{wrapfig}
\usepackage{enumitem}
\usepackage{changepage}
\usepackage{cases}
\usepackage{multirow}
\usepackage{tikz}
 
\newcommand{\ord}[0]{\textrm{\hspace*{0.15cm}ord\hspace*{0.1cm}}}
\newcommand{\lcm}[0]{\textrm{\hspace*{0.15cm}lcm\hspace*{0.1cm}}}
\newcommand{\Tr}[1]{\textrm{\hspace*{0.15cm}$\textrm{Tr}_{#1}$\hspace*{0.1cm}}}

\usepackage[left=20mm, top=20mm, right=20mm, bottom=20mm, nohead, nofoot]{geometry}

\newcommand\tab[1][0.5cm]{\hspace*{#1}}
\begin{document}
	\begin{justify}
		\thispagestyle{empty}\setlength{\parindent}{0pt}
  		\topskip0pt
 		\vspace*{\fill}
  		\begin{center}
  			\noindent\makebox[\linewidth]{\rule{\paperwidth}{0.4pt}}
   			\LARGE{\bigbreak ДОМАШНЯ РОБОТА №13\\З ПРЕДМЕТУ\\''МАТЕМАТИЧНІ ОСНОВИ КРИПТОЛОГІЇ''\\\bigbreak} 
   			ФІ-12 Бекешева Анастасія 
   			\noindent\makebox[\linewidth]{\rule{\paperwidth}{0.4pt}}
  		\end{center}
 		\vspace*{\fill}\newpage
 		\begin{enumerate}
 			\item \begin{table}[htp]\centering
\begin{tabular}{|c|c|c|c|c|c|}
\hline
       &  & ind & ord & $\Tr{F_{16}}$ & $\Tr{F_{16}/F_4}$   \\ \hline
(0000) &  &     &     & 0            & 0                   \\ \hline
(0001) &  &     & 1   & 0            & 0                   \\ \hline
(0010) &  &     & 15  & 1            & $\alpha^3+\alpha+1$ \\ \hline
(0011) &  &     & 5   & 1            & $\alpha^3+\alpha+1$ \\ \hline
(0100) &  &     & 15  & 1            & $\alpha^3+\alpha$   \\ \hline
(0101) &  &     & 5   & 1            & $\alpha^3+\alpha$   \\ \hline
(0110) &  &     & 15  & 0            & 1                   \\ \hline
(0111) &  &     & 15  & 0            & 1                   \\ \hline
(1000) &  &     & 5   & 1            & $\alpha^3+\alpha+1$ \\ \hline
(1001) &  &     & 15  & 1            & $\alpha^3+\alpha+1$ \\ \hline
(1010) &  &     & 3   & 0            & 0                   \\ \hline
(1011) &  &     & 3   & 0            & 0                   \\ \hline
(1100) &  &     & 15  & 0            & 1                   \\ \hline
(1101) &  &     & 15  & 0            & 1                   \\ \hline
(1110) &  &     & 15  & 1            & $\alpha^3+\alpha$   \\ \hline
(1111) &  &     & 5   & 1            & $\alpha^3+\alpha$   \\ \hline
\end{tabular}
\end{table}
			\item Незвідні поліноми матимуть вигляд $ax^2+bx+c$. Переберемо усі можливі значення у $F_4(\{0,1,\beta,\beta+1\})$. Зазначимо, що $a\neq0$ (інакше поліном буде першого степеню) і $c\neq0$ (інакше поліном точно звідний). \\\\\tab(Вибачте, дуже багато переписувати, тому чернетку як я перебирала усе це я \tab прикріплю іншим файлом)\\\\ Випишемо незвідні поліноми $x^2+x+1,x^2+\beta x+1,\beta x^2+1,\beta x^2+x+1,\beta x^2+\beta x+1,\\(\beta+1)x^2+1,(\beta+1)x^2+x+1,(\beta+1)x^2+(\beta+1)x+1,x^2+\beta,x^2+x+\beta,x^2+\beta x+\beta,\\\beta x^2+x+\beta,\beta x^2+\beta x+\beta,\beta x^2+(\beta+1) x+\beta,(\beta+1)x^2+\beta,(\beta+1)x^2+\beta x+\beta,\\(\beta+1)x^2+(\beta+1)x+\beta,x^2+(\beta+1),x^2+x+(\beta+1),x^2+(\beta+1)x+(\beta+1),\beta x^2+(\beta+1),\\\beta x^2+\beta x+(\beta+1),\beta x^2+(\beta+1)x+(\beta+1),(\beta+1)x^2+x+(\beta+1),(\beta+1)x^2+\beta x+(\beta+1),\\(\beta+1)x^2+(\beta+1)x+(\beta+1)$.
			\item Контрприклад $x^2+2x$
			\item $f(x)=x^5+x^4+1=(x^2+x+1)(x^3+x+1)$. Нехай $\alpha$ - корінь $x^2+x+1$. $\\\alpha^2=\alpha+1,\>\alpha^3=\alpha^2+1=1$. Тоді $\ord(\alpha)=3\Longrightarrow\ord(x^2+x+1)=3$. Нехай $\beta$ - корінь $x^3+x+1$. $\beta^4=\beta\cdot\beta^3=\beta(\beta+1)=\beta^2+\beta\neq1\Longrightarrow\ord(\beta)=7$.\\ Знайдемо $t=\min\{s|2^s\geq\max\{1,1\}\}=0.\ord(f(x))=2^0\lcm(3,7)=21$
			\item $\dfrac{\varphi(5^5-1)}{5}=\dfrac15\cdot\varphi(3124)=\dfrac15\cdot\varphi(2^2\cdot11\cdot71)=\dfrac{1400}5=280$
 		\end{enumerate}
 	\end{justify}
\end{document}