\documentclass[a4paper,12pt]{article}

\usepackage[ukrainian,english]{babel}
\usepackage{ucs}
\usepackage[utf8]{inputenc}
\usepackage[T2A]{fontenc}

\usepackage{amsmath}
\usepackage[document]{ragged2e}
\usepackage{graphicx}
\usepackage{wrapfig}
\usepackage{enumitem}
\usepackage{cases}

\newcommand{\ord}[0]{\textrm{\hspace*{0.15cm}ord\hspace*{0.1cm}}}

\usepackage[left=20mm, top=20mm, right=20mm, bottom=20mm, nohead, nofoot]{geometry}

\newcommand\tab[1][0.5cm]{\hspace*{#1}}
\begin{document}
	\begin{justify}
		\thispagestyle{empty}\setlength{\parindent}{0pt}
  		\topskip0pt
 		\vspace*{\fill}
  		\begin{center}
  			\noindent\makebox[\linewidth]{\rule{\paperwidth}{0.4pt}}
   			\LARGE{\bigbreak ДОМАШНЯ РОБОТА №4\\З ПРЕДМЕТУ\\''МАТЕМАТИЧНІ ОСНОВИ КРИПТОЛОГІЇ''\\\bigbreak} 
   			ФІ-12 Бекешева Анастасія 
   			\noindent\makebox[\linewidth]{\rule{\paperwidth}{0.4pt}}
  		\end{center}
 		\vspace*{\fill}\newpage
 		\begin{enumerate}
 			\item Таблиця Келі для $\sigma_3:$
 				\begin{table}[htp]\centering
\begin{tabular}{|c|c|c|c|c|c|c|}
\hline
$\cdot$ & $e$     & $\pi_1$ & $\pi_2$ & $\pi_3$ & $\pi_4$ & $\pi_5$ \\ \hline
$e$     & $e$     & $\pi_1$ & $\pi_2$ & $\pi_3$ & $\pi_4$ & $\pi_5$ \\ \hline
$\pi_1$ & $\pi_1$ & $e$     & $\pi_3$ & $\pi_2$ & $\pi_5$ & $\pi_4$ \\ \hline
$\pi_2$ & $\pi_2$ & $\pi_4$ & $e$     & $\pi_5$ & $\pi_1$ & $\pi_3$ \\ \hline
$\pi_3$ & $\pi_3$ & $\pi_5$ & $\pi_1$ & $\pi_4$ & $e$     & $\pi_2$ \\ \hline
$\pi_4$ & $\pi_4$ & $\pi_2$ & $\pi_5$ & $e$     & $\pi_3$ & $\pi_1$ \\ \hline
$\pi_5$ & $\pi_5$ & $\pi_3$ & $\pi_4$ & $\pi_1$ & $\pi_2$ & $e$     \\ \hline
\end{tabular}
\end{table}
 			\item Порядки елементів $\sigma_3:\\\pi_0=e\Longrightarrow\ord(\pi_0)=1,\tab \pi_1\cdot\pi_1=e\Longrightarrow\ord(\pi_1)=2\\\pi_2\cdot\pi_2=e\Longrightarrow\ord(\pi_2)=2,\tab\pi_3\cdot\pi_3=\pi_4,\pi_4\cdot\pi_3=e\Longrightarrow\pi_3^3=e\Longrightarrow\\\ord(\pi_3)=3,\tab\pi_4\cdot\pi_4=\pi_3,\pi_3\cdot\pi_4=e\Longrightarrow\ord(\pi_4)=3,\\ \pi5\cdot\pi_5=e\Longrightarrow\ord(\pi_5)=2$
 			\item Знайти усі підгрупи $\sigma_3:\\$Дві тривіальні підрупи це $\{e\}$ і $\sigma_3$. Підгрупи порядку 2: $\{e,\pi_1\},\{e,\pi_2\},\{e,\pi_5\}$. Підгрупa порядку 3: $\{e,\pi_3,\pi_4\}$.
 			\item $\langle a\rangle_{18}:\\\ord(a^1)=\ord(a^5)=\ord(a^7)=\ord(a^{11})=\ord(a^{13})=\ord(a^{17})=\dfrac{18}1=18,\\\ord(a^2)=\ord(a^4)=\ord(a^8)=\ord(a^{10})=\ord(a^{14})=\ord(a^{16})=\dfrac{18}2=9\\\ord(a^3)=\ord(a^{15})=\dfrac{18}3=6,\tab\ord(a^6)=\ord(a^{12})=\dfrac{18}6=3,\\\ord(a^9)=\dfrac{18}9=2,\tab\ord(a^{18})=\dfrac{18}{18}=1\\$
 				Підгрупи: тривіальні $\{e=a_{18}\},\langle a\rangle_{18},$ порядок 18: $\{a^1,a^5,a^7,a^{11},a^{13},a^{17}\},$ порядок 9: $\{e,a_2,a_4,a_8,a_{10},a_{14},a_{16}\},$  порядок 6: $\{e,a_3,a_{15}\},$  порядок 3: $\{e,a_6,a_{12}\},$  порядок 2: $\{e,a_9\}.\\\phi(18)=6$, отже кількість твірних елементів 6 і їх порядок 18,$\\\varphi(9)=6,$  у підгрупі порядку 9 - 6 твірних елементів, $\varphi(6)=2,$  у підгрупі порядку 6 - 2 твірних елементів, $\varphi(3)=2,$  у підгрупі порядку 3 - 2 твірних елементів, $\varphi(2)=1,$ у підгрупі порядку 2 - 1 твірний елементів. 
 			\item $\\$\begin{itemize}
 				\item [[реф.]] Нехай $a\in H\Longrightarrow aa^{-1}=e\in H$, бо $H$ підгрупа $H$. Тоді $a\equiv a\mod H\tab\forall a\in G$. Відношення рефлексивне.
 				\item [[сим.]] $a,b\in G:\tab a\equiv b\mod H\Longrightarrow ab^{-1}\in H\Longrightarrow (ab^{-1})^{-1}\in H\Longrightarrow ba^{-1}\in H\Longrightarrow b\equiv a\mod H$. Відношення симетричне. 
 				\item [[транз.]] $\forall a,b,c\in G:\tab a\equiv b \mod H,b\equiv c\mod H\Longrightarrow ab^{-1}\in H, bc^{-1}\in H\Longrightarrow(ab^{-1})^{-1}(bc^{-1})^{-1}\in H\Longrightarrow ac^{-1}\in H\Longrightarrow a\equiv c\mod H$. Відношення транзитивне.
 			\end{itemize}
 		\end{enumerate}
	\end{justify}
\end{document}