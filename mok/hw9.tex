\documentclass[a4paper,12pt]{article}

\usepackage[ukrainian,english]{babel}
\usepackage{ucs}
\usepackage[utf8]{inputenc}
\usepackage[T2A]{fontenc}

\usepackage{amsmath}
\usepackage{amsfonts}
\usepackage[document]{ragged2e}
\usepackage{graphicx}
\usepackage{wrapfig}
\usepackage{enumitem}
\usepackage{changepage}
\usepackage{cases}
 \usepackage{multirow}
 
\newcommand{\charr}[0]{\textrm{\hspace*{0.15cm}char\hspace*{0.1cm}}}

\usepackage[left=20mm, top=20mm, right=20mm, bottom=20mm, nohead, nofoot]{geometry}

\newcommand\tab[1][0.5cm]{\hspace*{#1}}
\begin{document}
	\begin{justify}
		\thispagestyle{empty}\setlength{\parindent}{0pt}
  		\topskip0pt
 		\vspace*{\fill}
  		\begin{center}
  			\noindent\makebox[\linewidth]{\rule{\paperwidth}{0.4pt}}
   			\LARGE{\bigbreak ДОМАШНЯ РОБОТА №9\\З ПРЕДМЕТУ\\''МАТЕМАТИЧНІ ОСНОВИ КРИПТОЛОГІЇ''\\\bigbreak} 
   			ФІ-12 Бекешева Анастасія 
   			\noindent\makebox[\linewidth]{\rule{\paperwidth}{0.4pt}}
  		\end{center}
 		\vspace*{\fill}\newpage
 		\begin{enumerate}
 			\item $g(x)=x^2+2x+4,\tab f(x)=4x^5+2x^4+x^3+1\\f(x)=(4x^3+4x^2+2x)g(x)+(2x+1)\\g(x)=(3x+2)(2x+1)+(2)\\2=g(x)-(2x+1)(3x+2)=g(x)-(3x+2)(f(x)-(4x^3+4x^2+2x)g(x))=\\=(-3x-2)f(x)+(12x^4+20x^3+14x^2+4x+1)g(x)
 				\\g^{-1}\mod f(x)=2x^4+4x^2+4x+1$
 			\item Фактор кільце є полем, тоді й тільки тоді, коли поліном є незвідним. У випадку з $x^2+k$, поліном є незвідним коли в нього нема коренів. Тобто перевіримо $x^2+k$, $ k\in\mathbb{Z}_7.$
 				\begin{itemize}
 					\item [(0)] $(0^2+k)\mod7=0\Rightarrow k =0$
 					\item [(1)] $(1^2+k)\mod 7=0\Rightarrow k =6$
 					\item [(2)] $(2^2+k)\mod 7=0\Rightarrow k =3$
 					\item [(3)] $(3^2+k)\mod 7=0\Rightarrow k =4$
 					\item [(4)] $(4^2+k)\mod 7=0\Rightarrow k =5$
 					\item [(5)] $(5^2+k)\mod 7=0\Rightarrow k =3$
 					\item [(6)] $(6^2+k)\mod 7=0\Rightarrow k =6$
 				\end{itemize}
 				Отже нам підходять усі $k$, окрім $k\in\{0,3,4,5,6\}$. Тобто $k=1,2$
 			\item $\\$
 				\begin{table}[htp]\centering
\begin{tabular}{|c|c|c|c|c|c|c|c|c|}\hline
векторна ф-ма & (0,0,0) & (0,0,1) & (0,1,0) & (0,1,1) & (1,0,0) & (1,0,1) & (1,1,0) & (1,1,1)   \\\hline
поліноми      & 0       & 1       & $x$     & $x+1$   & $x^2$   & $x^2+1$ & $x^2+x$ & $x^2+x+1$\\\hline
\end{tabular}
\end{table}
\begin{table}[htp]\centering
\begin{adjustwidth}{-0.9cm}{}
\begin{tabular}{|c|c|c|c|c|c|c|c|c|}
\hline
$\cdot$   & 0 & 1         & $x$       & $x+1$     & $x^2$     & $x^2+1$   & $x^2+x$   & $x^2+x+1$ \\ \hline
0         & 0 & 0         & 0         & 0         & 0         & 0         & 0         & 0         \\ \hline
1         & 0 & 1         & $x$       & $x+1$     & $x^2$     & $x^2+1$   & $x^2+x$   & $x^2+x+1$ \\ \hline
$x$       & 0 & $x$       & $x^2$     & $x^2+x$   & $x+1$     & 1         & $x^2+x+1$ & $x^2+1$   \\ \hline
$x+1$     & 0 & $x+1$     & $x^2+x$   & $x^2+1$   & $x^2+x+1$ & $x^2$     & 1         & $x$       \\ \hline
$x^2$     & 0 & $x^2$     & $x+1$     & $x^2+x+1$ & $x^2+x$   & $x^2$     & $x^2+1$   & 1         \\ \hline
$x^2+1$   & 0 & $x^2+1$   & 1         & $x^2$     & $x$       & $x^2+x+1$ & $x+1$     & $x^2+1$   \\ \hline
$x^2+x$   & 0 & $x^2+x$   & $x^2+x+1$ & 1         & $x^2+1$   & $x+1$     & $x$       & $x^2$     \\ \hline
$x^2+x+1$ & 0 & $x^2+x+1$ & $x^2+1$   & $x$       & 1         & $x^2+x$   & $x^2$     & $x+1$     \\ \hline
\end{tabular}
\end{adjustwidth}
\end{table}
 		\end{enumerate}
 	\end{justify}
\end{document}