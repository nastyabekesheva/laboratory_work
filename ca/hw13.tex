\documentclass[a4paper,12pt]{article}

\usepackage[ukrainian,english]{babel}
\usepackage{ucs}
\usepackage[utf8]{inputenc}
\usepackage[T2A]{fontenc}

\usepackage{amsmath}
\usepackage{amsfonts}
\usepackage{amssymb}
\usepackage[document]{ragged2e}
\usepackage{graphicx}
\usepackage{wrapfig}
\usepackage{enumitem}
\usepackage{subfigure}

\usepackage{color,soul}
\setul{0.5ex}{0.3ex}

\usepackage[left=20mm, top=20mm, right=20mm, bottom=20mm, nohead, nofoot]{geometry}

\usepackage{tikz,pgfplots}
\usetikzlibrary{arrows.meta} 
\usetikzlibrary{shapes}

\makeatletter
\newcommand{\skipitems}[1]{%
  \addtocounter{\@enumctr}{#1}%
}
\makeatother

\newcommand\tab[1][0.5cm]{\hspace*{#1}}
\newcommand\dx[1]{\hspace*{0.2cm}\textrm{d}{#1}\hspace*{0.1cm}}
\newcommand\Res[0]{\hspace*{0.2cm}\textrm{Res}\hspace*{0.1cm}}
\newcommand\dint[0]{\displaystyle\int}
\newcommand\doint[0]{\displaystyle\oint}
\newcommand{\Int}[0]{\mathfrak{I}}

\newcommand*\circled[1]{\tikz[baseline=(char.base)]{
            \node[shape=circle,draw,inner sep=2pt] (char) {#1};}}
 

\renewcommand{\Im}[0]{\mathfrak{Im}}
\renewcommand{\Re}[0]{\mathfrak{Re}}

\begin{document}
	\begin{justify}
		\thispagestyle{empty}\setlength{\parindent}{0pt}
  		\topskip0pt
 		\vspace*{\fill}
  		\begin{center}
  			\noindent\makebox[\linewidth]{\rule{\paperwidth}{0.4pt}}
   			\LARGE{\bigbreak ДОМАШНЯ РОБОТА №13\\З ПРЕДМЕТУ\\''ТЕОРІЯ ФУНКЦІЇ КОМПЛЕКСНОЇ ЗМІННОЇ''\\\bigbreak} 
   			ФІ-12 Бекешева Анастасія 
   			\noindent\makebox[\linewidth]{\rule{\paperwidth}{0.4pt}}
  		\end{center}
 		\vspace*{\fill}\newpage
 		\begin{enumerate}
 			\item \begin{enumerate}
 			\item $a_0=\dfrac{1}{\pi}\dint\limits_{-\pi}^{\pi}f(x)=\dfrac{1}{\pi}\left(\dint\limits_{-\pi}^01\dx{x}+\dint\limits_0^{\pi}3\dx{x}\right)=\dfrac1\pi\left((0-(-\pi)+(3\pi-0)\right)=\dfrac{4\pi}\pi=4\\a_n=\dfrac1\pi\dint\limits_{-\pi}^{\pi}f(x)\cos nx=\dfrac{1}{\pi}\left(\dint\limits_{-\pi}^0\cos nx\dx{x}+\dint\limits_0^{\pi}3\cos nx\dx{x}\right)=\dfrac1{\pi n}(\sin0n-\sin-\pi n+\\+3\sin\pi n-3\sin0 n)=0\\b_n=\dfrac1\pi\dint\limits_{-\pi}^{\pi}f(x)\sin nx=\dfrac{1}{\pi }\left(\dint\limits_{-\pi}^0\cos nx\dx{x}+\dint\limits_0^{\pi}3\cos nx\dx{x}\right)=\dfrac1{\pi n}(-\cos0+\cos\pi n -\\-\cos\pi n+\cos 0)=\dfrac1{\pi n}\left(-1+(-1)^n-3(-1)^n+3\right)=\dfrac2{\pi n}\left(1+(-1)^{n+1}\right)\\f(x)=2+\sum\limits_{n=1}^\infty\dfrac2{\pi n}(1+(-1)^{n+1})\sin nx$
 			\begin{figure*}[htp]\centering
 				\begin{tikzpicture}
 						\begin{axis}[axis lines=middle, xmin=-5, xmax=5,ymin=-5,ymax=5, xtick = {-3.14,3.14},
        				xticklabels = {$-\pi$,$\pi$},extra x ticks = {-4,-2,2,4},extra y ticks={1,3}]	
						\end{axis}
						\draw (3.43,4.55) node[circle,fill,inner sep=1pt] {};
						\draw (3.43,3.42) node[circle,fill,inner sep=1pt,color=red] {};
						\draw (5.58,3.42) node[circle,fill,inner sep=1pt,color=red] {};
						\draw (5.58,4.55) node[circle,fill,inner sep=1pt,color=red] {};
						\draw (1.28,3.42) node[circle,fill,inner sep=1pt,color=red] {};
						\draw (1.28,4.55) node[circle,fill,inner sep=1pt,color=red] {};
						\draw (1.28,4.55) -- (0.,4.55);
						\draw (1.28,3.42) -- (3.43,3.42);
						\draw (3.43,4.55) -- (5.58,4.55);
						\draw (6.7,3.42) -- (5.58,3.42);
						\draw (3.43,0) node[below] {$f(x)$};
 				\end{tikzpicture}
 				\begin{tikzpicture}
 						\begin{axis}[axis lines=middle, xmin=-5, xmax=5,ymin=-5,ymax=5, xtick = {-3.14,3.14},
        				xticklabels = {$-\pi$,$\pi$},extra x ticks = {-4,-2,2,4},extra y ticks={1,3}]	
						\end{axis}
						\draw (3.43,4.55) node[circle,fill,inner sep=1pt, ,color=red] {};
						\draw (3.43,3.42) node[circle,fill,inner sep=1pt,color=red] {};
						\draw (5.58,3.42) node[circle,fill,inner sep=1pt,color=red] {};
						\draw (5.58,4.55) node[circle,fill,inner sep=1pt,color=red] {};
						\draw (1.28,3.42) node[circle,fill,inner sep=1pt,color=red] {};
						\draw (1.28,4.55) node[circle,fill,inner sep=1pt,color=red] {};
						\draw (1.28,3.985) node[circle,fill,inner sep=1pt] {};
						\draw (3.43,3.985) node[circle,fill,inner sep=1pt] {};
						\draw (5.58,3.985) node[circle,fill,inner sep=1pt] {};
						\draw (1.28,4.55) -- (0.,4.55);
						\draw (1.28,3.42) -- (3.43,3.42);
						\draw (3.43,4.55) -- (5.58,4.55);
						\draw (6.7,3.42) -- (5.58,3.42);
						\draw (3.43,0) node[below] {$S(x)$};
 				\end{tikzpicture}
 			\end{figure*}
 			\item $a_0=\dfrac{1}{\pi}\dint\limits_{-\pi}^{\pi}f(x)=\dfrac1\pi\left(\dint\limits_0^\pi\sin x\dx{x}\right)=\dfrac1\pi(-\cos0-\cos\pi)=\dfrac2\pi\\a_1=\dfrac{1}{\pi}\dint\limits_{-\pi}^{\pi}f(x)\cos x=\dfrac1\pi\left(\dint\limits_0^\pi\sin x\cos x\dx{x}\right)=\dfrac1\pi\left(\dint\limits_0^\pi\sin x\dx{(\sin x)}\right)=\\=\dfrac1{2\pi}\left(\sin^2\pi-\sin^20\right)=0\\b_1=\dfrac{1}{\pi}\dint\limits_{-\pi}^{\pi}f(x)\sin x=\dfrac1\pi\left(\dint\limits_0^\pi\sin^2 x\dx{x}\right)=\dfrac1{2\pi}\left(\dint\limits_0^\pi(1-\cos2x)\dx{x}\right)=\\=\dfrac1{2\pi}\left(\pi-\dfrac{\sin2\cdot\pi}2+0-\dfrac{\sin2\cdot0}2\right)=\dfrac12
 				 \\a_n=\dfrac{1}{\pi}\dint\limits_{-\pi}^{\pi}f(x)\cos nx\dx{x}=\dfrac1\pi\left(\dint\limits_0^\pi\sin x\cos nx\dx{x}\right)=\dfrac1{2\pi}\left(\dint\limits_0^\pi\sin(x+nx)\dx{x}\right.+\\+\left.\dint\limits_0^\pi\sin(x-nx)\dx{x}\right)=\dfrac1{2\pi}\left(-\dfrac{\cos(x+nx)}{1+n}\bigg|_{0}^{\pi}-\dfrac{\cos(x-nx)}{1-n}\bigg|_{0}^{\pi}\right)=\\=\dfrac1{2\pi}\left(\dfrac{(1-n)\cos(x+nx)+(1+n)\cos(x-nx)}{1-n^2}\right)=\\=\dfrac1{2\pi}\left(\dfrac{(1-n)(-1)^n+(1+n)(-1)^n+2}{1+n^2}\right)=\dfrac{(-1)^n(1-n+n+1)+2}{1-n^2}=\dfrac{(-1)^n+1}{\pi(1-n^2)}
 				 \\b_n=\dfrac1{\pi}\dint\limits_{-\pi}^{\pi}f(x)\sin nx\dx{x}=\dfrac1\pi\left(\dint\limits_0^\pi\sin x\sin nx\dx{x}\right)=\dfrac1{2\pi}\left(\dint\limits_0^\pi\cos(x-nx)\dx{x}\right.-\\-\left.\dint\limits_0^\pi\cos(x+nx)\dx{x}\right)=\dfrac1{2\pi}\left(-\dfrac{\sin(x-nx)}{1+n}\bigg|_{0}^{\pi}-\dfrac{\sin(x+nx)}{1-n}\bigg|_{0}^{\pi}\right)=\\=\dfrac1{2\pi}\left(\dfrac{(1+n)\sin(x-nx)+(n-1)\sin(x+nx)}{1-n^2}\right)=\\=\dfrac1{2\pi}\left(\dfrac{(1+n)\sin\pi n-(n-1)\sin\pi n}{1+n^2}\right)=0\\f(x)=\dfrac1\pi+\sum\limits_{n=2}^\infty\dfrac{(-1)^n+1}{\pi(1-n^2)}\cos nx+\dfrac12\sin x$
 				 \begin{figure*}[htp]\centering
 				\begin{tikzpicture}
 						\begin{axis}[axis lines=middle, xmin=-5, xmax=5,ymin=-5,ymax=5, xtick = {-3.14,3.14},
        				xticklabels = {$-\pi$,$\pi$},extra x ticks = {-4,-2,2,4},extra y ticks={1,3}]	
        					\addplot[thick,domain=0:pi] {sin(deg(x))};
        					\addplot[thick,domain=-3*pi:-pi] {sin(deg(x))};
						\end{axis}
						\draw[thick] (3.43,2.85) -- (1.28,2.85);
						\draw[thick] (6.8,2.85) -- (5.58,2.85);
						%\draw[thick] (5.58,2.85) arc (30:150:1.24);
						%\draw[thick] (1.29,2.85) arc (30:130:1);
 				\end{tikzpicture}
 			\end{figure*}
			\item $a_0=\dfrac1\pi\dint\limits_{0}^{2\pi} f(x)\dx{x}=\dint\limits_{0}^{2\pi}(2x-3)\dx{x}=\dfrac1\pi(4\pi^2-6\pi)=4\pi-6\\a_n=\dfrac1\pi\dint\limits_{0}^{2\pi}f(x)\cos nx\dx{x}=\dfrac1\pi\left(2\dint\limits_{0}^{2\pi}x\cos nx\dx{x}-3\dint\limits_{0}^{2\pi}\cos nx\dx{x}\right)=\\=\dfrac1\pi\left(\dfrac{2x\sin nx -3\sin nx}{n}+\dfrac{2\cos nx}{n^2}\right)\Bigg|_0^{2\pi}=\dfrac{4\pi n\cdot 0-3n\cdot 0+2\cdot(-1)-2}{\pi n^2}=\dfrac{0}{2\pi}\\b_n=\dfrac1\pi\dint\limits_{0}^{2\pi}f(x)\sin nx\dx{x}=\dfrac1\pi\left(2\dint\limits_{0}^{2\pi}x\sin nx\dx{x}-3\dint\limits_{0}^{2\pi}\sin nx\dx{x}\right)=\\=\dfrac1\pi\left.\left(\dfrac{2x\sin nx}{n}+\dfrac{2\cos nx}{n^2}-\dfrac{3\sin nx}{n}\right)\right|_0^{2\pi}=\dfrac{4\pi n\sin2\pi n-3n\sin2\pi n-4\pi n-\cos2\pi n}{\pi n^2}=\\=-\dfrac{4n}{\pi n}=-\dfrac4n\\f(x)=2\pi-3+\sum\limits_{n=1}^\infty-\dfrac4n\sin nx$
			\begin{figure*}[htp]\centering
 				\begin{tikzpicture}
 						\begin{axis}[axis lines=middle, xmin=-10, xmax=10,ymin=-10,ymax=10, xtick = {-6.28,6.28},
        				xticklabels = {$-2\pi$,$2\pi$},extra x ticks = {-4,-2,2,4},extra y ticks={-3,3}]	
						\end{axis}
						\draw (5.58,5.6) node[circle,fill,inner sep=1pt,color=red] {};
						\draw (3.43,5.6) node[circle,fill,inner sep=1pt,color=red] {};
						\draw (3.43,2) -- (5.58,5.6);
						\draw (1.28,2) -- (3.43,5.6);
						\draw (5.58,2) -- (6.5,3.37);
						\draw (3.43,0) node[below] {$f(x)$};
 				\end{tikzpicture}
 				\begin{tikzpicture}
 						\begin{axis}[axis lines=middle, xmin=-10, xmax=10,ymin=-10,ymax=10, xtick = {-6.28,6.28},ytick = {3.28},
        				xticklabels = {$-2\pi$,$2\pi$}, yticklabels={$2\pi-3$},extra x ticks = {-4,-2,2,4},extra y ticks={-10,-5,-3,5,10}]	
						\end{axis}
						\draw (5.58,5.6) node[circle,fill,inner sep=1pt,color=red] {};
						\draw (5.58,2) node[circle,fill,inner sep=1pt,color=red] {};
						\draw (3.43,5.6) node[circle,fill,inner sep=1pt,color=red] {};
						\draw (3.43,3.79) node[circle,fill,inner sep=1pt] {};
						\draw (5.58,3.79) node[circle,fill,inner sep=1pt] {};
						\draw (1.28,3.79) node[circle,fill,inner sep=1pt] {};
						\draw (3.43,2) node[circle,fill,inner sep=1pt,color=red] {};
						\draw (3.43,5.6) node[circle,fill,inner sep=1pt,color=red] {};
						\draw (1.28,2) node[circle,fill,inner sep=1pt,color=red] {};
						\draw (3.43,2) -- (5.58,5.6);
						\draw (1.28,2) -- (3.43,5.6);
						\draw (5.58,2) -- (6.5,3.37);
						\draw (3.43,0) node[below] {$S(x)$};
 				\end{tikzpicture}
 			\end{figure*}
			\item $a_0=\dfrac1\pi\dint\limits_{-\pi}^\pi f(x)\dx{x}=\dfrac1\pi\cdot\left.\dfrac{x^3}3\right|_{-\pi}^\pi=\dfrac23\pi^2\\a_n=\dfrac1\pi\dint\limits_{-\pi}^\pi f(x)\cos nx\dx{x}=\left\langle t=nx\right\rangle=\dfrac1\pi\dint\limits_{-\pi}^\pi\dfrac{t^2\cos t}{n^3}\dx{t}=\dfrac1{\pi n^3}\left(t^2\sin t-\dint\limits_{-\pi}^\pi2t\sin t\right)=\\=\dfrac1{\pi n^3}\left(t^2\sin t+2t\cos t+\dint\limits_{-\pi}^\pi\cos t dt\right)=\dfrac1{\pi n^3}((nx^2)\sin nx-2(nx(-\cos nx)+\sin nx))\big|_{-\pi}^\pi=\\=\dfrac1{\pi n^3}(4\pi^2n^2\sin\pi n+4\pi n\cos\pi n-4\sin\pi n)=\dfrac{4(-1)^n}n\\f(x)=\dfrac{\pi^2}3+\sum\limits_{n=1}^\infty\dfrac{4(-1)^n}{n^2}\cos n(x)$
			\begin{figure*}[htp]\centering
				\begin{tikzpicture}
					\begin{axis}[axis lines=middle, xmin=-10, xmax=10,ymin=-10,ymax=10, xtick = {-6.28,-3.14,3.14,6.28},
        				xticklabels = {$-2\pi$,$-\pi$,$\pi$,$2\pi$}, extra y ticks={1,3}]	
        				\addplot [domain=-pi:pi] {(x^2)};
        				\addplot [domain=-3*pi:-pi] {((x+2*pi)^2)};
        				\addplot [domain=pi:3*pi] {(x-2*pi)^2)};
						\end{axis}
				\end{tikzpicture}
			\end{figure*}
			%TODO finish graphs 
 			\end{enumerate}
 		\end{enumerate}
 	\end{justify}
\end{document}