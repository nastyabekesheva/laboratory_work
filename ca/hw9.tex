\documentclass[a4paper,12pt]{article}

\usepackage[ukrainian,english]{babel}
\usepackage{ucs}
\usepackage[utf8]{inputenc}
\usepackage[T2A]{fontenc}

\usepackage{amsmath}
\usepackage{amsfonts}
\usepackage{amssymb}
\usepackage[document]{ragged2e}
\usepackage{graphicx}
\usepackage{wrapfig}
\usepackage{enumitem}
\usepackage{subfigure}

\usepackage{color,soul}
\setul{0.5ex}{0.3ex}

\usepackage[left=20mm, top=20mm, right=20mm, bottom=20mm, nohead, nofoot]{geometry}

\usepackage{tikz,pgfplots}
\usetikzlibrary {arrows.meta} 

\makeatletter
\newcommand{\skipitems}[1]{%
  \addtocounter{\@enumctr}{#1}%
}
\makeatother

\newcommand\tab[1][0.5cm]{\hspace*{#1}}
\newcommand\dx[1]{\hspace*{0.2cm}\textrm{d}{#1}\hspace*{0.1cm}}
\newcommand\dint[0]{\displaystyle\int}

\renewcommand{\Im}[0]{\mathfrak{Im}}
\renewcommand{\Re}[0]{\mathfrak{Re}}
\begin{document}
	\begin{justify}
		\thispagestyle{empty}\setlength{\parindent}{0pt}
  		\topskip0pt
 		\vspace*{\fill}
  		\begin{center}
  			\noindent\makebox[\linewidth]{\rule{\paperwidth}{0.4pt}}
   			\LARGE{\bigbreak ДОМАШНЯ РОБОТА №9\\З ПРЕДМЕТУ\\''ТЕОРІЯ ФУНКЦІЇ КОМПЛЕКСНОЇ ЗМІННОЇ''\\\bigbreak} 
   			ФІ-12 Бекешева Анастасія 
   			\noindent\makebox[\linewidth]{\rule{\paperwidth}{0.4pt}}
  		\end{center}
 		\vspace*{\fill}\newpage
 		\begin{enumerate}
 			\skipitems{1}
 			\item $f(z)=\dfrac1{z(z-3)},\>z=0,\>z=3$
 			\begin{figure*}[htp]\centering
 					\begin{tikzpicture}
 						\begin{axis}[ticks=none,axis lines=middle, xmin=-0.5, xmax=0.5,ymin=-0.5,ymax=0.5 ]	
						\end{axis}
						\draw (5.43,2.85) node[circle,fill,inner sep=1pt,color=red] {};
						\draw (5.43,2.85) node[right] {$3$} ;
						\draw (3.43,2.85) node[circle,fill,inner sep=1pt,color=red] {};
						\draw (3.43,2.85) node[right] {$0$} ;
						\draw (3.43,2.85) circle (2);
 					\end{tikzpicture}
 					\begin{tikzpicture}
 						\begin{axis}[ticks=none,axis lines=middle, xmin=-0.265, xmax=1,ymin=-0.5,ymax=0.5 ]	
						\end{axis}
						\draw (3.43,2.85) node[circle,fill,inner sep=1pt,color=red] {};
						\draw (3.43,2.85) node[right] {$3$} ;
						\draw (1.43,2.85) node[circle,fill,inner sep=1pt,color=red] {};
						\draw (1.43,2.85) node[right] {$0$} ;
						\draw (3.43,2.85) circle (2);
 					\end{tikzpicture}
 				\end{figure*}
 				$\\z_0=0,\tab |z|<3,\tab \dfrac1{z-3}=-\dfrac13\cdot\dfrac1{1-\frac z3}=\left\langle\left|\dfrac z3\right|<1 \right\rangle=-\dfrac13\sum\limits_{n=0}^\infty\left(\dfrac z3\right)^n=-\sum\limits_{n=0}^\infty\dfrac{z^{n}}{3^{n+1}}\\f(z)=-\dfrac1z\sum\limits_{n=0}^\infty\dfrac{z^n}{3^{n+1}}=-\sum\limits_{n=0}^\infty\dfrac{z^{n-1}}{3^{n+1}}\\z_0=3,\tab |z-3|<3,\tab \dfrac1z=-\dfrac{1}{-3-(z-3)}=\dfrac13\cdot\dfrac1{1-\left(-\frac{z-3}{3}\right)}=\dfrac13\sum\limits_{n=0}^\infty\left(-\dfrac {z-3}3\right)^n=\\=\sum\limits_{n=0}^\infty\dfrac{(-1)^n(z-3)^n}{3^{n+1}}\\f(z)=\dfrac1{z-3}\sum\limits_{n=0}^\infty\dfrac{(-1)^n(z-3)^n}{3^{n+1}}=\sum\limits_{n=0}^\infty\dfrac{(-1)^n(z-3)^{n-1}}{3^{n+1}}$
 			\item $f(z)=\dfrac{5z-50}{2z^3+5z^2-25z},\tab z_0=0$
 				\begin{figure*}[htp]\centering
 					\begin{tikzpicture}
 						\begin{axis}[ticks=none,axis lines=middle, xmin=-0.5, xmax=0.5,ymin=-0.5,ymax=0.5 ]	
						\end{axis}
						\draw (5.43,2.85) node[circle,fill,inner sep=1pt,color=red] {};
						\draw (5.43,2.85) node[right] {$5$} ;
						\draw (4.43,2.85) node[circle,fill,inner sep=1pt,color=red] {};
						\draw (4.43,2.85) node[right] {$2.5$} ;
						\draw (3.43,2.85) circle (2);
						\draw (3.43,2.85) circle (1);
						\draw(4,3.3) node {$D_1$};
						\draw(4.4,4) node {$D_2$};
						\draw(4.9,4.8) node {$D_3$};
 					\end{tikzpicture}
 				\end{figure*}
 				$\\\left[\begin{array}{lll}
 					z_1=0,\tab&D_1:\>\>&0<|z|<2.5\\
 					z_2=2.5&D_2:&2.5<|z|<5\\
 					z_3=-5&D_3:&|z|>5
 				\end{array}\right.,\\ f(z)=\dfrac{5}{z}\cdot\dfrac{z-10}{(z+5)(2z-5)}=\dfrac5z\cdot\left(\dfrac A{z+5}+\dfrac B{2z-5}\right)=\dfrac5z\cdot\left(\dfrac 1{z+5}+\dfrac 1{2z-5}\right)\\\dfrac1{z+5}=\dfrac15\cdot\dfrac1{1-\left(-\frac z5\right)}=\dfrac15\sum\limits_{n=0}^\infty\left(-\dfrac z5\right)=\sum\limits_{n=0}^\infty\dfrac{(-1)^nz^n}{5^{n+1}}\in D_1, D_2\\
 				\dfrac1{z+5}=\dfrac1z\cdot\dfrac1{1-\left(-\frac 5z\right)}=\dfrac1z\sum\limits_{n=0}^\infty\left(-\dfrac 5z\right)=\sum\limits_{n=0}^\infty\dfrac{(-1)^n5^n}{z^{n+1}}\in D_3\\
 				\dfrac1{2z-5}=-\dfrac15\cdot\dfrac1{1-\frac {2z}5}=-\dfrac15\sum\limits_{n=0}^\infty\left(\dfrac{2z}5\right)=\sum\limits_{n=0}^\infty\dfrac{(-1)^n2^nz^n}{5^{n+1}}\in D_1\\
 				\dfrac1{2z-5}=\dfrac1{2z}\cdot\dfrac1{1-\frac 5{2z}}=\dfrac1{2z}\sum\limits_{n=0}^\infty\left(\dfrac 5{2z}\right)=\sum\limits_{n=0}^\infty\dfrac{5^n}{2^{n+1}z^{n+1}}\in D_1,D_3\\
 				\begin{array}{lll}
 					D_1:\tab&0<|z|<2.5:\tab&f(z)=\dfrac5z\left(\sum\limits_{n=0}^\infty\dfrac{(-1)^nz^n}{5^{n+1}}+\sum\limits_{n=0}^\infty\dfrac{2^nz^n}{5^{n+1}}\right)=\sum\limits_{n=0}^\infty\dfrac{((-1)^n+2^n)z^{n-1}}{5^n}\\
 					D_2:&2.5<|z|<5:&f(z)=\sum\limits_{n=0}^\infty\dfrac{(-1)^nz^{n-1}}{5^n}-\sum\limits_{n=0}^\infty\dfrac{5^{n+1}}{2^{n+1}z^{n+2}}\\
 					D_3:&|z|>5:&f(z)=\sum\limits_{n=0}^\infty\dfrac{(-1)^n-2^{-n-1}5^{n+2}}{z^{n+2}}
 				\end{array}$
 		\end{enumerate}
 	\end{justify}
\end{document}