\documentclass[a4paper,12pt]{bookest}
\usepackage[russian,english]{babel}
\usepackage{ucs}
\usepackage[utf8]{inputenc}
\usepackage[T2A]{fontenc}
\usepackage{graphicx}
\usepackage{wrapfig}
\usepackage{amsfonts}
\usepackage{amsthm}
\usepackage{amssymb}
\usepackage{amsmath}
\usepackage{mdframed} 
\usepackage{xcolor}
\usepackage{xparse} 
\usepackage{enumerate}	
\usepackage{multicol}
\newtheorem{theorem}{Theorem}[section]
\theoremstyle{remark}
\newtheorem*{remark}{Remark}
\newtheorem{cons}{Consequence}
\newtheorem*{cons*}{\textbf{Consequence}}
\newcommand\tab[1][1cm]{\hspace*{#1}}
\newmdenv[   linecolor=black,
  topline=false,
  bottomline=false,
  rightline=false,
  skipabove=\topsep,
  skipbelow=\topsep
]{leftrule}

\NewDocumentEnvironment{example}{O{\textbf{Example:}}} {\begin{leftrule}\noindent\textcolor{black}{#1}\par}
{\end{leftrule}}

\begin{document}
\title{Математический анализ 2}
\maketitle
\date{}
\catcode`@=11
\def\caseswithdelim#1#2{\left#1\,\vcenter{\normalbaselines\m@th
  \ialign{\strut$##\hfil$&\quad##\hfil\crcr#2\crcr}}\right.}\catcode`@=12

\def\bcases#1{\caseswithdelim[{#1}}
\def\vcases#1{\caseswithdelim|{#1}}

\newpage
\tableofcontents
\newpage
\chapter{Неопределенный интеграл}
\section{Понятие первообразной и неопределеного интеграла}
Функция $F(x)$ называется \textbf{первообразной} функции $f(x)$ на множестве $X\subset \mathbb{R}$, или $\forall x\in X\>\>F'(x)=f(x)$.
\begin{theorem}
Если $F(x)$ - некоторая первообразная для $f(x)$ на множестве $X$, то любая другая первообразная имеет вид: $F(x)+c$, где $c=const$ - произвольная.
	
\end{theorem}
\begin{proof}
Пусть $F(x)$ - первообразная функции  $f(x)$, т.е.  $F'(x)=f(x)$; Тогда: $(F(x)+c)'=F'(x)+0=f(x)\Rightarrow F(x)+c$ - первообразная для $f(x)(c=const)$.\\ 
Пусть $F_1(x)$ - тоже первообразная для $f(x)$, т.е. $F_1'(x)=f(x)$.\\
Рассмотрим разность: $F_1(x)-F(x);$\\
$(F_1(x)-F(x))'=F_1'(x)-F'(x)=f(x)-f(x)=0\Rightarrow F_1(x)-F(x)=c=const$, т.е. :$F_1(x)=F(x)+c$
\end{proof}
Таким образом множество всех первообразных функции $f(x)$ имеет вид $F(x)+c$.\\
Множество всех первообразных функции $f(x)$ называется \textbf{неопределенным интегралом} этой функции и обозначается $\int f(x)dx$.\\
$f(x)$ - подинтегральная функция, $f(x)dx$ - подинтегральное выражение, $x$ - переменная инегрирования, $\int$ - неопределенный интеграл.\\ \begin{example}
$${
f(x) = sign(x) = \bcases{1,\>\> x>0\cr0,\>\> x=0\cr -1,\>\> x<0},\>\>x\in(-1;\>1).
}$$
Предположим, что существует такая первообразная $\exists F(x):\> \forall x\in(-1;\>1):\>\>F'(x)=sign(x)$, т.е.
$${
F'(x)=  \bcases{1,\>\> x\in(0;\>1)\cr0,\>\> x=0\cr -1,\>\> x<\in(-1;\>0)}\Rightarrow \begin{cases}
F'(x)=0\cr F_-'(x)=-1\cr F_+'(x)=1
\end{cases}\Rightarrow
}$$ 
$F'(0)$ - не существует $\Rightarrow$ противоречие $\Rightarrow F(x)$ не существует.\end{example}
\begin{remark}
Достаточным условием существования первообразной у функции  на данном множестве является ее непрерывность на этом множестве.
\end{remark}
\newpage
\section{Свойства неопределенного интеграла}
Пусть $\int f(x)dx=F(x)+c\>(F'(x)=f(x))$.
\begin{enumerate}
	\item Производная от неопределенного интеграла равна подинтегральной функции, дифференциал неопределенного интеграла равен подинтегральному выражению. \\
	$$(\int f(x)dx)'_x=f(x);\>\> d(\int f(x)dx)=f(x)dx.$$
	\begin{proof}
		$(\int f(x)dx)'_x=(F(x)+c)'_x=F'(x)+c'=f(x);$	\\$d(\int f(x)dx)=(\int f(x)dx)'_xdx=f(x)dx$.
	\end{proof}
	\item Неопределенный интеграл от дифференциала некоторой функции равен сумме этой функции и произвольной постоянной.
	$$\int d(F(x))=F(x)+c.$$
	\begin{proof}
		$\int d(F(x))=\int F'(x)dx=\int f(x)dx=F(x)+c$
	\end{proof}
	\item Постоянный множитель можно выносить за знак интеграла.
	$$\int af(x)dx=a\int f(x)dx,\>\> a=const.$$
	\begin{proof}
		$\int af(x)dx=\int a F'(x)dx=\int(aFx)'dx=\int d(aF(x))=(aF(x)++c_1)=a(F(x)+\frac{c_1}{a}=\Big|c=\frac{c_1}{a}\Big|=a(f(x)+c)=a\int f(x)dx$
	\end{proof}
	\item Интеграл суммы двух функций равен сумме интегралов этих функций.
	$$\int(f(x)+g(x))dx=\int f(X)dx+\int g(x)dx.$$
	\begin{proof}
		Пусть $\int g(x)dx=G(x)+c$; тогда $\int(f(x)+g(x))dx=\int(F'(x)++G'(x))dx=\int(F(x)+G(x))'dx=\int d(F(x)+G(x))=F(x)+G(x)++c=\Big|c=c_1+c_2\Big|=(F(x)+c_1)+(G(x)c_2)=\int f(x)dx+\int g(x)dx$
	\end{proof}
\end{enumerate}
\begin{remark}\
	\begin{enumerate}
		\item[-]свойство 4 справедливо для любого конечного числа слогаемых 
		\item[-]свойство 3-4 называются свойством линейности неопределенного интеграла 
		\item[-]свойство 1-2 отражают связь операций дифференцирования и интегрирования  
	\end{enumerate}		
\end{remark}
\section{Таблица основных неопределенных интегралов}
\begin{enumerate}
	\item $$\int x^\alpha dx=\frac{x^{\alpha+1}}{\alpha+1}+c;\>\>\alpha\in\mathbb{R}\backslash\{-1\}$$
	\item $$\int\frac{dx}{x}=\ln|x|+c$$
	\item $$\int a^xdx=\frac{a^x}{\ln a}+c,\>\>a>0$$
	\item $$\int e^xdx=e^x+c$$
	\item $$\int\sin x\>dx	=-\cos x+c$$
	\item $$\int\cos x\>dx=\sin x+c$$
	\item $$\int\frac{dx}{\cos^2dx}=\tan x+c$$
	\item $$\int\frac{dx}{\sin^x}=-\cot x+c$$
	\item $$\int\textrm{sh}x\>dx=\textrm{ch}\>x+c$$
	\item $$\int\textrm{ch}x\>dx=\textrm{sh}\>x+c$$
	\item $$\int\frac{dx}{\textrm{ch}^2x}=\textrm{th}\>x+c$$
	\item $$\int\frac{dx}{\textrm{sh}^2x}=-\textrm{cth}\>x+c$$
	\item $$\int\frac{dx}{\sqrt{1-x^2}}=\arcsin x+c$$
		$$\int\frac{dx}{\sqrt{a-x^2}}=\arcsin \frac{x}{a}+c$$
	\item $$\int\frac{dx}{1+x^2}=\arctan x+c$$
		$$\int\frac{dx}{a^2+x^2}=\frac{1}{a}\arctan\frac{x}{a}+c$$
		Дополнительные формулы: 
	\item $$\int\frac{dx}{x^2-a^2}=\frac{1}{2a}\ln|\frac{x-a}{x+a}|+c\textrm{ - высокий логарифм}$$ 
	\item $$\int\frac{dx}{\sqrt{x^2+A}}=\ln|x+\sqrt{x^2+A}|+c\textrm{ - длинный логарифм }$$
	\item $$\int\sqrt{x^2+A}\>dx=\frac{x}{2}\sqrt{x^2+A}+\frac{A}{2}\ln|x+\sqrt{x^2+A}|+c$$
	\item $$\int\sqrt{a^2-x^2}\>dx=\frac{x}{2}\sqrt{a^2-x^2}+\frac{a^2}{2}\arcsin\frac{x}{a}+c$$
\end{enumerate}
В этих формулах вместо $x$ может быть записана произвольная дифференцируемая функция от $x$.
\section{Основные примеры интегрирования}
\subsection{Непосредственное интегрирование}
Непосредственное интегрирование заключается в использовании тождественных преобразований подинтегральнной функции, свойства линейности интеграла и таблицы интегралов.\\
\begin{example}
\begin{enumerate}
	\item $\int(\frac{\sqrt{x}+1}{\sqrt[3]{x}})^2dx=\int\frac{x+2x^{\frac{1}{2}}+1}{\sqrt[3]{x^2}}dx=\int(x^{\frac{1}{3}}+2x^{\frac{-1}{6}}+x^{\frac{-2}{3}})dx=\frac{3}{4}x^{\frac{4}{3}}+2\cdot\frac{x^{\frac{5}{6}}}{5}\cdot 6+x^{\frac{1}{3}}\cdot 3+c$
	\item $\int\frac{dx}{\sin^2x\cos^2x}=|\cos^2x+\sin^2x=1|=\int\frac{\cos^2x+\sin^2x}{\sin^2x\cos^2x}dx=\int\frac{1}{\sin^2x}dx+\int\frac{1}{\cos^2x}dx=\tan x-\cot x+c$
\end{enumerate}
\end{example}
\subsection{Замена переменной}
\begin{theorem}
Пусть на $\forall x\in(a;\>b)\>\>\int f(x)dx=F(x)+c$, (на всем интервале $(a;\>b)$ известна первообразная функции): $F'(x)=f(x)\>\>x=\varphi(t)$ - функция дифференцируемая; причем $\varphi(t):\>t\in(\alpha;\>\beta)$ и $\varphi:(\alpha;\>\beta)\rightarrow(a;\>b)$.
Тогда справедлива формула:
$$\int f(\varphi(t))\cdot \varphi_t'(t)dt=F(\varphi(t))+c$$	
\end{theorem}
\begin{proof}
	$(f(\varphi(t)))_t'=F_\varphi'(\varphi(t))\cdot\varphi_t'(t)=|\varphi(t)=x|=F_x'(x)\varphi_t'(t)=|F_x'(x)=f(x)|=f(x)\cdot\varphi(t)=|x=\varphi(t)|=f(\varphi(t))\cdot\varphi_t'(t)\Rightarrow F(\varphi(t))\textrm{ первообразная для } \\f(\varphi(t))\cdot\varphi_t'(t)\Rightarrow\int f(\varphi(t))\cdot\varphi_t'(t)dt=F(\varphi(t))+c$
\end{proof}
\begin{remark}
$$\varphi_t'(t)dt=d(\varphi(t))\Rightarrow\int f(\varphi(t))\cdot d\varphi = F(\varphi)+c$$	
\end{remark}
\begin{enumerate}
	\item \textbf{Внесения выражения под знак дифференциала}$$\int f(x)dx=\int g(\varphi(x))\cdot\varphi_x'(x)dx=\int g(\varphi)d\varphi=|G(x)-\textrm{известно}\>G'(x)=g(x)|=$$$$=G(\varphi)+c=G(\varphi(x))+c.$$
	Часто используются преобразование дифференциала $dx=d(x++a)=\frac{1}{k}d(kx)=\frac{1}{k}d(kx+b)\\ x^{n-1}dx=\frac{1}{n}d(x^n)$\\
	\textbf{Преобразования дифференциалов}\\
	$\sin x\>dx=-d(\cos x)$\\
	$\cos x\>dx=d(\sin x)$\\
	$\frac{dx}{\cos^2x}=d(\tan x)$\\
	$\frac{dx}{x}=d(\ln x)$\\
	$\frac{dx}{1+x^2}=d(\arctan x)$\\
	$\frac{dx}{\sqrt{1-x^2}}=d(\arcsin x)$\\
	\begin{example}
	$\int \sin^3x\>dx=\int\sin x\sin^2x\>dx\int(1-\cos^2x)\cdot(-d(\cos x))=\int(\cos^2-1)d(\cos x)=\\=\frac{\cos^3x}{3}-\cos x+c.$\end{example}
	\item \textbf{Вынесения выражения из-под знак дифференциала}\\
	$\int f(x)\>dx = |x=\varphi(t)\Rightarrow dx=\varphi'(t)dt|=\int f(\varphi(t))\cdot\varphi'(t)dt=|g(t)=\\=G'(t)|=G(t)+c=|x=\varphi(t)\>\>t=\varphi^{-1}(x)|=G(\varphi^{-1}(x))+c$\\\\
	\begin{example}
	$\int\sqrt{a^2-x^2}dx=|x=a\sin t\>\>dx=a\cos t\>dt|=\int\sqrt{a^2-a^2\sin^2t}a\cos t\>dt=\\=a^2\int\sqrt{1-\sin^2t}\cdot\cos t\>dt=a^2\int\cos^@ t\>dt=\frac{a^2}{2}\int(1+\cos^2t)dt=\frac{a^2}{2}(t+\frac{\sin2t}{2})+c=\frac{a^2}{2}(t+\sin t\cos t)+c=|\cos t=\sqrt{1-\sin t}\>\>\frac{x}{a}\>\>\\t=\arcsin\frac{x}{a}|=\frac{a^2}{2}(\arcsin\frac{x}{a}+\sqrt{1-\frac{x^2}{a^2}}\cdot\frac{x}{a})=\frac{a^2}{2}\arcsin\frac{x}{a}+\frac{x}{2}\sqrt{a^2-x^2}+c$\end{example}
\end{enumerate}
\subsection{интегрирование по частям}
Пусть $u=u(x),\>v=v(x)$ - две диффренцируемые функции. \\
по свойству дифференциала:\\$d(uv)=udv+vdu\Rightarrow\int d(uv)=\int udv+\int vdu$ - формула интегрирования по частям.\\
В исходном интеграле $\int f(x)dx$ подинтегральное выражение представляется в виде двух сомножителей. Как правило, это можно сделать неоднозначно.\\
После того как $u$ и $dv$ выбраны, находим $du,\>v,\>...$\\
$\int f(x)dx = |f(x)=u,\>dx=dv|\Rightarrow du=u'dx=...\Rightarrow v=\int dv$ \\
в результате применения формулы полученный интеграл оказывается более простым, чем исходный.\\
При необходимости формула интегрирования по частям применяется несколько раз.
\begin{enumerate}[I.]
	\item $\int P_n(x)$
		$\left\{\begin{array}{lr}
        \sin(kx+b)\\ \cos(kx+b)\\a^{kx}\\e^{kx}\\\textrm{sh}kx,\>\textrm{ch}(kx)
        \end{array}\right\}dx\tab U=Pn(x);\>\>dv=\big\{\dots\big\}$
        \item $\int P_n(x)$
		$\left\{\begin{array}{lr}
        \arcsin x\\ \arccos x\\\arctan x\\\ln x\\
        \end{array}\right\}dx\tab U=\big\{\dots\big\};\>\>dv=Pn(x)dx$
        \item $\int e^{kx}$
		$\left\{\begin{array}{lr}
        \sin (ax+b)\\ \cos (ax+b)\end{array}\right\}dx\tab U=e^{kx};\>\>dv=\big\{\dots\big\}dx$
\end{enumerate}

\begin{example}
$\int_{I} e^x\sin2xdx\underset{III}{=}\bigg|u=e^x\Rightarrow du=e^xdx;\>\sin2xdx=dv;\>v=\int\sin2xdx=\\=-\frac{\cos2z}{2}\bigg|=-\frac{e^x\cos2x}{2}+\int\frac{\cos2x}{2}\cdot e^xdx=\bigg|u=e^x;\>du=e^x;\>dv=\cos2xdx;\>v=\\=\frac{\sin2x}{2}\bigg|=-\frac{e^x\cos2x}{2}+\frac{1}{2}(\frac{e^x\sin2x}{2}-\int\frac{\sin2x}{2}\cdot e^xdx)=-\frac{e^x\cos2x}{2}+\frac{1}{4}e^x\sin2x-\\-\frac{1}{4}\int \underset{I}{e^x\sin2x}dx\\I=-\frac{e^x\cos2x}{2}+\frac{1}{4}e^x\sin2x-\frac{1}{4}I;\>\>I=-\frac{e^x\cos2x}{2}+\frac{1}{4}e^x\sin2x.\\I=\frac{4}{5}(\frac{1}{4}e^x\sin2x-\frac{1}{2}e^x\cos2x)+c$
\end{example}
\section{Интегрирование рациональных функций}
\subsection{Основные сведения о рациональных функциях}
\begin{enumerate}
	\item \textbf{Многочлен(целая рациональная функция)\\ Многочленом} $P_n(x)$ называется функция вида $P_n(x)=a_nx^n+a_{n-1}x^{n-1}+\dots+a_1x^1+a_0;$ где $n\in\mathbb{n},\>\>a_i\in\mathbb{r},\>\>i=\overline{0,n}$\\
	\textbf{Корнем} многочлена называется значение $x_0$(вообще говоря, комплексное) аргумента $x$, при котором многочлен обращается в ноль.\\$x_0$ - корень $P_n(x)$ или $P_n(x_0)=0$ 
	\begin{theorem}
		$$\textrm{Если }x_0 - \textrm{корень многочлена } P_n(x), \textrm{ то многочлен делится нацело на} (x-x_0), $$$$\textrm{т.е. } P_n(x)\textrm{ представлется в виде: } P_n(x)=(x-x_0)\cdot Q_{n-1}(x),$$$$ \textrm{где } Q - \textrm{многочлен степени} n-1$$
	\end{theorem}
	\begin{theorem}
		$$\textrm{Всякий многочлен степени }n>0\textrm{ имеет по крайней мере один корень,}$$$$ \textrm{действительный или комплексный}$$
	\end{theorem}
	\begin{cons*}$\\$
		\begin{enumerate}[(1)]
			\item Многочлен $n$-ой степени можно представить в виде: $P_n(x)=a_n(x-x_1)(x-x_2)\dots(x-x_n),$ где $x_1,\>\dots,\>x_n$ - корни $P_n(x),\>a_n$ - старший коэффициент 
			\item Если среди корней многочлена имеются одинаковые, то объединим соответствующие или множители. Получим: \\$P_n(x)=a_n(x-x_1)^{k_1}(x-x_2)^{k_2}\dots (x-x_n)^{k_m},$ где $k_1+k_2+\dots+k_m=n.$ для $x_i:\>(x-x_i)^k_i;\>k_i$ - кратность корня $x_i$. \\ Такое представление называется разложением многочлена на линейные множители. 
		\end{enumerate}	
	\end{cons*}
	\begin{theorem}
		$\\$Известно, что если многочлен имеет комплексный корень $\\ x_0=a_i+ib(a,\>b\in\mathbb{R};\>x_0\in\mathbb{C}),$  то комплексное спряженое число  $\\ \bar{x}=a-ib$ - тоже корень $P_n(x).$ Таким образом, в разложении многочлена комплексно спряженные числа входят парами, пeремножим: \\$(x-(a+ib))(x-(a-ib))=x^2-x(a+ib)-x(a-ib)+(a+ib)(a-ib)=x^2-ax-ibx-ax+ibx+a^2+b^2=x^2-2ax+a^2+b^2$.\\Полученый трехчлен имеет действительный коэффициент, причем дискретный $D=B^2-4A\cdot C=4a^2-4(a^2+b^2)=-4b^2<0$\\Получаем, что пару множителей, соответсвующую двум комплексных сопряженным корням можно заменить квадратный трехчлен с действительным коэффициентом и $D<0$.\\Окончательно получим разложение на множители в виде:\\ $P_n(x)=(x-x_1)^{k_1}(x-x_2)^{k_2}(x-x_5)^{k_5}(x^2+p_1x+q_1)^{l_1}\dots(x^2p_mx+q_m)^{l_m},$ где $x_1,\>\dots,\>x_5\in\mathbb{R}$ - корни многочлена $Pn(x); p_i,\>q_i\in\mathbb{R},\>i=\overline{1,m};\\D_i=p_i^2-4q_i<0.\tab k_1+\dots+k_5+2(l_1+\dots+l_m)=n$
	\end{theorem}
	\textbf{Многочлен} называется \textbf{тождественно равным нулю} \\
	$$Pn(x)\equiv0,\textrm{ если }\forall x\in\mathbb{R}\>\>Pn(x)=0$$
	\begin{theorem}
		$$\textrm{Многочлен тожественно равен нулю тогда и только тогда, когда}$$$$ \textrm{все его коэффициенты равны нулю } a_i=0,\>i=\overline{0,n}$$
	\end{theorem}
	\begin{cons*}$\\$
		Два многочлена тождественно равны, если их степени одинаковы и имеют одинаковые коэффициенты при одинаковых степенях $x$
		\begin{proof}
			$P_n(x)\equiv Q_n(X)\\ P_n(x)-Q_n(x)\equiv0\\ \underset{=0}{(a_n+b_n)}x^n+\underset{=0}{(a_{n-1}+b_{n-1})}x^{n-1}+\dots+(a_0+b_0)=0$
		\end{proof} 
	\end{cons*}
	\begin{example}
		$P_3(x)=3x^2-2x+4\\Q_4(x)=a_4x^4+a_3x^3-a_2x^2+a_1x+a_0\\P_3(x)\equiv Q_4(x)\Rightarrow\begin{cases}
			a_4=0\\a_3=3\\a_2=0\\a_1=-2\\a_1=4
		\end{cases}$	
	\end{example}
	\item \textbf{Дробная рациональная функция}\\
	\textbf{Дробной рациональной функцией} называется отношение двух многочленов.
	$\frac{P_n(x)}{Q_m(x)}>$ многочлены $\big\{$ дробная рациональная функция, рациональная дробь. Если $n\geq m$, то рациональная дробь \textbf{неправильная}, если $n<m$ - \textbf{правильная}. 
	\begin{theorem}
		$$\textrm{Неправильная рациональная дробь может быть представлена в виде}$$ $$\textrm{суммы многочлена и правильной рациональной дроби.}$$ $$\frac{P_n(x)}{Q_m(x)}=\underset{\text{многочлен}}{\overset{\text{целая часть}}{L_{n-m}(x)}}+\frac{R_k(x)}{Q_m(x)},\>k<m,\>R_n(x)\textrm{ - многочлен.}$$
	\end{theorem}
	Элементарные(простейшие) рациональные дроби: 
	\begin{enumerate}[I.]
		\item $$\frac{A}{x-a}\tab A,\>a\in\mathbb{R}$$
		\item $$\frac{A}{(x-a)^k}\tab k\in\mathbb{N},\>k>1,\>A,\>a\in\mathbb{R}$$
		\item $$\frac{Mx+N}{x^2+px+q}\tab M,\>n,\>p,\>q\in\mathbb{R},\>D=p^2-4q<0$$
		\item $$\frac{Mx+N}{(x^2+px+q)^k}\tab M,\>n,\>p,\>q\in\mathbb{R},\>D=p^2-4q<0,\>k\in\mathbb{N},\>k>1$$
	\end{enumerate}
	\begin{theorem}
		$\\$Пусть $\frac{P_n(x)}{Q_m(x)}$ - правильная рациональная дробь($n<m$), и знаменатель дроби $Q_m(x)$ разложен на множители: $$Q_m(x)=\underbrace{(x-x_1)^{k_1}\dots(x-x_5)^{k_5}}_{\textrm{действительные корни}}\underbrace{(x^2+p_1x+q_1)^{l_1}\dots(x^2+p_mx+q_m)^{l_m}}_{D<0}$$ Тогда заданная дробь раскладывается в сумму простых дробей следующего вида:$$\frac{P_n(x)}{Q_m(x)}=\frac{A_1}{x-x_1}+\frac{A_2}{(x-x_1)^2}+\dots+\frac{A_{k_1}}{(x-x_1)^{k_1}}+\dots+$$ $$+\frac{F_1}{x-x_5}+\frac{F_2}{(x-x_5)^2}+\dots+\frac{F_{k_5}}{(x-x_5)^{k_5}}+\frac{M_1x+N_1}{(x^2+p_1x+q_1)}+\dots+\frac{M_{l_1}+N_l}{(x^2+p_lx+q_l)^l}+$$При этом: $$(x-x_i)^{k_i}\leftrightarrow\frac{A_1}{x-x_i}+\frac{A_2}{(x-x_i)^2}+\dots+\frac{A_{k_i}}{(x-x_	i)^{K}};$$$$(x^2+p_jx+q_j)^{l_j}\leftrightarrow\frac{M_1x+N_1}{(x^2+p_jx+q_j)}+\frac{M_2x+N_2}{(x^2+p_jx+q_j)}+\dots+\frac{M_{l_j}x+N_{l_j}}{(x^2+p_jx+q_j)}$$В разложении появляются так называемые неопределенные коэффициенты, которые подлежат дальнейшиму определению.
	\end{theorem}
	\begin{example}
		$\frac{3x-2}{(x-1)^3(x+2)(x^2+1)(x^2+2x+3)^2}=\frac{A_1}{x-1}+\frac{A_2}{(x-1)^2}+\frac{A_3}{(x-1)^3}+\frac{B}{x+2}+\frac{Cx+D}{x^2+1}+\frac{Ex+F}{x^2+2x+3}+\\+\frac{Mx+N}{(x^2+2x+3)^2}$
	\end{example}
	Для того, чтобы найти неопределенные коэффициенты в полученном выражении, умножают обе части тождества на знаменатель левой части. Таким образом, получают 2 тождественно равных многочлена. Раскрывая скобки справа, после сего приравнивают коэффициенты при одинаковых степенях. Получают систему линейных уравнений для определения неизвестных коэффициентов.
	\begin{example}
		$\frac{x^4+2x^3+5x^2-1}{x(x^2+1)^2}=\frac{A}{x}+\frac{Bx+C}{x^2+1}+\frac{Dx+E}{(x^2+1)^2}\>\>\bigg|x(x^2+1)^2\\x^4+2x^3+5x^2-1=a(x^2+1)^2+(Bx+C)x(x^2+1)+(Dx+E)x=\\=A(x^4+2x^2+1)+(Bx+C)(x^3+x)+Dx^2+Ex=\\=Ax^4+2Ax^2+A+Bx^4+Cx^3+Bx^2+Cx+Dx^2+Ex=\\=(A+B)x^4+Cx^3)+(2A+B+D)x^2+(E+C)x+A.\\\begin{cases}
			x^4:\tab A+B=1\tab\tab\>\>A=1\\ x^3:\tab C=2\tab\tab\tab  B=2\\ x^2:\tab 2A+B+D=5\tab\>C=2\\ x^1:\tab C+E=0\tab\tab\>D=5\\ x^0:\tab A=-1\tab\tab\>\>\>\>\>\ E=-2\\
		\end{cases}\\\\ \frac{x^4+2x^3+5x^2-1}{x(x^2+1)^2}=- \frac{1}{x}+\frac{2x+2}{x^2+1}+\frac{5x-2}{(x^2+1)^2}$
	\end{example}
	В некоторых случаях для нахождения неопределенных коэффициентов можно воспользоваться так называемым методом частных значений аргумента. Он состоит в том, что аргументу $x$ придаются конкретные числовые значения столько раз, сколько содержится неизвестных коэффициентов в разложении. При этом удобно выбирать $x$ равным значению действительного корня знаменателя. 
	\begin{example}
		$\frac{3x-4}{x(x-2)(x+1)}=\frac{A}{x}+\frac{B}{x-2}+\frac{C}{x+1}\\ A=\frac{3x-4}{(x-2)(x+1)}\Big|_{x=0}=\frac{-4}{-2\cdot 1}=2\\ B=\frac{3x-4}{x(x+1)}\Big|_{x=2}=\frac{6-4}{2\cdot 3}=\frac{1}{3}\\C=\frac{3x-4}{x(x-2)}\Big|_{x=-1}=\frac{-3-4}{-1\cdot (-3)}=-\frac{7}{3}$
	\end{example}
	\begin{example}
		$\frac{x^2+1}{x(x-1)^2}=\frac{A}{x}+\frac{B}{x-1}+\frac{C}{(x-1)^2}\\\\ A=\frac{x^2+1}{(x-1)^2}\Big|_{x=	0}=1\\ C=\frac{x^2+1}{x}\Big|_{x=1}=2\\\\$ при $x=2:\\B=\frac{5}{2}-\frac{1}{2}-2=0$ 
	\end{example} 
\end{enumerate}
\subsection{Интегрирование простейших дробей}
\begin{enumerate}[I.]
	\item $\int \frac{A}{x-a}dx=A\int\frac{d(x-a)}{x-a}=A\ln |x-a|+c$
	\item $\int\frac{A}{(x-a)^k}dx=A\int(x-a)^{-k}d(x-a)=A\frac{(x-a)^{-k+1}}{-k+1}+c$
	\item $\int\frac{Mx+N}{x^2+px+q}=\int\frac{Mx+N}{x^2+2x\cdot{p\over 2}+{p^2\over 4}-{p^2\over4}+q}=M\int \frac{x+\frac p2-\frac p2}{(x+\frac p2)^2+q-{p^2\over 4}}d(x+\frac p2)+\\+N\int\frac{d(x+\frac p2)}{(x+\frac p2)^2+q-{p^2\over 4}}=M\int \frac{(x+\frac p2-\frac p2)d(x+\frac p2)}{(x+\frac p2)^2+q-{p^2\over 4}}+(N-\frac{Mp}2)\int\frac{d(x+\frac p2)}{(x+\frac p2)^2+q-{p^2\over 4}}=\vcases{(x+\frac p2)=t \cr q-\frac{p^2}4=a^2}\bigg|=M\int \frac{tdt}{t^2+a^2}+(N-\frac{Mp}2)\int\frac{dt}{t^2+a^2}=\frac M2(\int\frac{d(t^2+a^2)}{(t^2+a^2)^k})+(N-\frac{Mp}2)\cdot I_k=\frac M2\frac{(t^2+a^2)^{-k+1}}{-k+1}+\\+(N-\frac{Mp}2)\cdot I_k\\\\$ Найдем $I_k=\int \frac{dt}{(t^2+a^2)^k}=\vcases{U=\frac{1}{(t^2+a^2)^k}\Rightarrow dU=-k(t^2+a^2)^{-k-1}\cr dV=dt;\>V=t;\>2tdt=-2k\frac{tdt}{(t^2+a^2)^{k+1}}}\Bigg|=\\=\frac{t}{(t^2+a^2)^k}+\int t\cdot 2k\frac{tdt}{(t^2+a^2)^k}=\frac{t}{(t^2+a^2)^k}+2k\int\frac{(t^2+a^2-a^2)dt}{(t^2+a^2)^{k+1}}=\frac{t}{(t^2+a^2)^k}+\\+2k\int\left( \frac{1}{(t^2+a^2)^k}-\frac{a^2}{(t^2+a^2)^{k+1}} \right)dt=\frac{t}{t^2+a^2}+2k\left(I_k-a^2I_{k+1}\right)=\\=\frac{t}{t^2+a^2}+2kI_k-2ka^2I_{k+1}\Rightarrow2ka^2I_{k+1}=\frac{t}{(t^2+a^2)^k}+I_k(2k-1);\\\\$ Пусть $k+1=n\Rightarrow k=n-1\\$ Получим: $I_n=\frac{1}{a^2(2n-2)}\cdot\frac{t}{(t^2+a^2)^{n-1}}+\frac{2n-3}{2n-2}\cdot\frac 1{a^2}\cdot I_{n-1};\>n>2$
\end{enumerate}
\subsection{Общая схема интегрирования рациональных дробей}






\end{document}