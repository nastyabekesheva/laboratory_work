\documentclass[a4paper,12pt]{article}
\usepackage[ukrainian,english]{babel}
\usepackage{ucs}
\usepackage[utf8]{inputenc}
\usepackage[T2A]{fontenc}
\usepackage{amsmath}
\usepackage{amsfonts}
\usepackage{graphicx}
\newcommand\tab[1][1cm]{\hspace*{#1}}
\usepackage[left=20mm, top=20mm, right=10mm, bottom=20mm, nohead, nofoot]{geometry}
\begin{document}
\begin{center}
{\LARGE Домашня робота 3}%TODO 1.59(б,в), 1.60(б,в), 1.65
\end{center}
\begin{itemize}
	\item [1.58] Prove that (3, 5, 7) is the only prime triplet.\\
		$(p,\>p+2,\>p+4),\>p\geq3. \\$ Let $p=3n,\>n\in\mathbb{N}
		\Rightarrow \forall x\in\mathbb{Z},\>x=\left\{\begin{array}{l}
			3n+1\\3n+2
		\end{array}\right.\\$\begin{enumerate}
			\item $n=1\Rightarrow p=3,\>p+2=5,\>p+4=7$ - ok
			\item $p=3n+1\Rightarrow (3n+1,\>2n+2,\>3(n+1))$ - contradiction
			\item $p=3n+2\Rightarrow (3n+2,\>3n+4n,\>3(n+2))$ - contradiction
		\end{enumerate}
	\item [1.59] \begin{itemize} 
	\item [(b)] Prove that there exists infinitely many numbers in form $4n+3$.\\Let $A=\{p\>|\>p=4n+3,\>p$ - prime$\}$ be finite$.\>p_i\in A:\>\>b=4\cdot p_1\cdot p_2\dots p_n+3\\b$ - either prime or composite.\begin{align*}
		p\textrm{ - prime} &\tab  p_i\in A:\tab b\neq p_i&\textrm{ - contradiction}\\
		 p\textrm{ - composite} &\tab\exists d\in A:\tab b\>\vdots\>d
	\end{align*}\\$x$ - prime: $\left\{\begin{array}{l}
		x=4n+1\\x=4n+3
	\end{array}\right.\Rightarrow b=4n+1\Rightarrow\exists d\in A\\\dfrac bd=\dfrac{4\cdot p_1\cdot p_2\dots p_n+1}d=\dfrac{4\cdot p_1\cdot p_2\dots p_n}d+\dfrac1d\Rightarrow \dfrac1d\not\in \mathbb{N}\Rightarrow d\not\in A\\\Rightarrow$ there exists infinitely many numbers in form $4n+3$
	\item [(c)] Prove that there infinitely many numbers in form $6n-1$.\\Let $A=\{p\>|\>p=6n-1,\>p$ - prime$\}$ be finite$.\>p_i\in A:\>\>b=6\cdot p_1\cdot p_2\dots p_n-1\\b$ - either prime or composite.\begin{align*}
		p\textrm{ - prime} &\tab  p_i\in A:\tab b\neq p_i&\textrm{ - contradiction}\\
		 p\textrm{ - composite} &\tab\exists d\in A:\tab b\>\vdots\>d
	\end{align*}
	\end{itemize}
	\item [1.60] \begin{itemize}
		\item [(a)] Prove that $p_{n+1}<p_1p_2\dots p_n,\>p_i$ - prime.\\Let $a=p_1p_2\dots p_n+1,\>t\>\vdots\>d,\>d\not\in\{p1,\dots,\>p_n\}\Rightarrow d=p_{n+k},\>p_{n+1}\leq p_{n+k}\leq a\\p_{n+1}\leq p_1p_2\dots p_n,\>p_i+1\Rightarrow p_{n+1}<p_1p_2\dots p_n,\>p_i$
		\item [(b)] Prove that $p_{n}\leq2^{2^{n-1}},\>p_i$ - prime. \begin{enumerate}
			\item $n=1:\>2\leq2^{2^0},\>n=2:\>3\leq2^{2^1}$
			\item Let $p_{n}\leq2^{2^{n-1}}$
			\item $p_{n+1}\leq2^{2^{n}}\\p_{n+1}\leq p_1p_2\dots p_n+1= 2^{2^1}2^{2^2}\dots2^{2^n}+1=2^{(2^1+2^2+\dots+2^n)}+1=2^{2^{n+1}-1}+1\\\Rightarrow p_{n+1}\leq 2^{2^n}$
		\end{enumerate}
		\item [(c)] Prove that $p_{n}>2n,\>p_i$ - prime.
	\end{itemize}
\end{itemize}





































\end{document}