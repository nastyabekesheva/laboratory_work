\documentclass[a4paper,12pt, centered]{bookest}
\usepackage[ukrainian,english]{babel}

% Input Type and AMS-Packages 
\usepackage{ucs}
\usepackage[utf8]{inputenc}
\usepackage[T2A]{fontenc}
\usepackage{amsmath, amsthm, amsfonts}

% Math
\usepackage{cases}
\usepackage{mathtools}
\usepackage[makeroom]{cancel}

% Typography
\usepackage{enumitem}

% Aligment
\usepackage[document]{ragged2e}
\usepackage{graphicx}
\usepackage{wrapfig}
\usepackage{booktabs}          
\usepackage[flushleft]{threeparttable}

% Coluors and other things
\usepackage{xparse}
\usepackage{mdframed}
\usepackage{xcolor}
\usepackage{xpatch}
\usepackage[many]{tcolorbox}

% Refs
\usepackage{nameref,hyperref}
\usepackage[capitalize]{cleveref}

% Plots
\usepackage[figurename=Рис.]{caption}
\usepackage{tikz,pgfplots}
\usetikzlibrary{arrows.meta} 
\usepackage{tikz-3dplot}

\tdplotsetmaincoords{60}{115}
\pgfplotsset{compat=newest}

% ------------Custom Enviroments------------

% ****Proof****
\xpatchcmd{\prooff}{\itshape}{\bfseries\itshape}{}{}

\tcolorboxenvironment{prooff}{
  blanker,
  before skip=\topsep,
  after skip=\topsep,
  borderline west={0.4pt}{0.4pt}{black},
  breakable,
  left=12pt,
  right=12pt, 
}

% ****Example****
\newmdenv[   linecolor=black,
  topline=false,
  bottomline=false,
  rightline=false,
  skipabove=\topsep,
  skipbelow=\topsep
]{leftrule}

\definecolor{black}{cmyk}{.30,0,0,.67} 
\NewDocumentEnvironment{example}{O{\textbf{Приклад:$ $}}} {\begin{leftrule}\noindent\textcolor{black}{#1}\par}
{\end{leftrule}}
\catcode`@=11

\def\caseswithdelim#1#2{\left#1\,\vcenter{\normalbaselines\m@th
  \ialign{\strut$##\hfil$&\quad##\hfil\crcr#2\crcr}}\right.}\catcode`@=12
  
% ------------Enviroments adjusment------------

% ****Chapter****
\makeatletter
\renewcommand{\@chapapp}{Частина}
\makeatother

% ****Table of contents****
\addto\captionsenglish{
  \renewcommand{\contentsname}
    {Зміст}
}
  
% ------------Custom commands------------

\newcommand\tab [1][0.5cm]{\hspace*{#1}}
\newcommand{\Nn}[0]{\mathbb{N}}
\newcommand{\Zn}[0]{\mathbb{Z}}
\newcommand{\Qn}[0]{\mathbb{Q}}
\newcommand{\Rn}[0]{\mathbb{R}}
\newcommand{\Cn}[0]{\mathbb{C}}
\renewcommand{\Im}[0]{\mathfrak{Im}}
\renewcommand{\Re}[0]{\mathfrak{Re}}
\renewcommand{\th}[0]{\textrm{\hspace*{0.15cm}th\hspace*{0.1cm}}}
\renewcommand{\cth}[0]{\textrm{\hspace*{0.15cm}cth\hspace*{0.1cm}}}
\newcommand{\Arg}[0]{\hspace*{0.15cm}\textrm{Arg}\hspace*{0.1cm}}
\newcommand{\Ln}[0]{\hspace*{0.15cm}\textrm{Ln}\hspace*{0.1cm}}
\newcommand{\deff}[0]{\tab\tab\textbf{def}\tab}
\newcommand{\limntoinf}[0]{\lim\limits_{n\to\infty}}
\newcommand{\eqdef}[1]{\underset{\mathrm{def\>#1}}{=\joinrel=}}
\newcommand{\equset}[1]{\underset{\mathrm{#1}}{=\joinrel=}}

% ------------Enviroments------------

\newtheorem{thm}{Теорема}[chapter]
\newtheorem*{prooff}{Доведення}
\newtheorem{cons}{Наслідок}[chapter]
\newtheorem*{cons*}{Наслідок}
\newtheorem{remark}{Зауваження}[chapter]
\newtheorem*{remark*}{Зауваження}
\newtheorem{statement}{Твердження}[chapter]
\newtheorem*{statement*}{Твердження}


\title{{Теорія функції комплексної змінної}\thispagestyle{empty}}

\begin{document}
\begin{justify}
	\maketitle
	\let\cleardoublepage\clearpage
	\tableofcontents
	\newpage
	\chapter{Комплексні числа та функції комплексної змінної}
\section{Основні поняття.}
$$\underset{\text{натуральні}}{\Nn}\subset\underset{\text{цілі}}{\Zn}\subset\underset{\text{раціональні}}{\Qn}\subset\underset{\text{дійсні}}{\Rn}\subset\underset{\text{комплексні}}{\Cn}$$
$(x,y):x,y\in\Rn$ - пара дійсних чисел.
\begin{figure*}[htp]
	\begin{center}
	\begin{tikzpicture}
		\begin{axis}[ticks=none,axis lines=middle, xmin=-1, xmax=2,ymin=-1,ymax=1.5]
		\end{axis}	
		\draw[-Stealth] (2.29,2.28) -- (5,4) node[midway,above,sloped] {$\rho$};
		\draw (5,4) node[circle,fill,inner sep=0.5pt] {};
		\draw[-Stealth] (2.29,2.28) -- (5,0.56) node[right] {};
		\draw (5,0.56) node[circle,fill,inner sep=0.5pt] {};
		\draw (5,0.56) node[right] {$\bar{z}$};
		\draw (5,4) node[right] {$z$};
		\draw[dashed] (5,4) -- (5,0.56) node[midway, anchor=south west ] {$x$};
		\draw[dashed] (5,4) -- (2.29,4) node[left] {$y$};
		\draw (2.29,2.9) node[circle,fill,inner sep=0.5pt] {};
		\draw (2.29,2.9) node[left] {$i$};
		\draw (3,2.28) arc (0:23:1);
		\draw (3,2.28) node[anchor=south west] {{\tiny$\varphi$}};
		\draw (3,2.28) arc (0:-23:1);
		\draw (3,2.28) node[anchor=north west] {{\tiny$-\varphi$}};
		\draw (2.29,5.4) node[left] {$\Im$};
		\draw (6.5,2.3) node[above] {$\Re$};
		\draw (7,4) node[above] {$z\in\Cn$};
	\end{tikzpicture}
\end{center}
\end{figure*}
$$z=x+iy\textrm{ - алгебраїчна форма }z$$
Якщо $y=0$, то $z=x\in\Rn$. $\Re z=x$ - дійсна частина $z$. Якщо $x=0$, то $z=iy$ - чисто уявне число. $\Im z=y$ - уявна частина $z$. Значення $x,y\in\Rn$ - дійсні. Для $z=i:x=0,y=1,i$ - уявна одиниця. Якщо $z=x+iy$, то $\bar{z}=x-iy$ - спряжене до $z$.
Нехай $\rho,\varphi$ - полярні координати. Тоді модуль $z:|z|=\rho=\sqrt{x^2+y^2}$. $|z|\geq0,\forall z\in\Cn$. Якщо $|z|=0\Leftrightarrow z=0$. $\varphi$ - аргумент $z$ (кут, утворений радіус-вектором, проведеним в точку $z$ у додатньому напрямку з осі $Ox$).\\
$\Arg z$ - множина значень аргумента $z$. $\Arg z=\arg z+2\pi k,k\in\Zn,\arg z$ - головне значення аргумента. $\arg z=\varphi\in(-\pi,\pi]$ 
$$\arg z=\begin{cases}
	\arctan\frac xy,x>0\tab(I,IV\textrm{ чв.})\\\arctan\frac yx+\pi,x<0,y>0\tab(II\textrm{ чв.})\\\arctan\frac yx-\pi,x<0,y<0\tab(III\textrm{ чв.})
\end{cases},\tab\arg z\textrm{ визначений для }z\neq0!$$
$z=\begin{vmatrix}
	x=|z|\cos\varphi\\y=|z|\sin\varphi
\end{vmatrix}=|z|(\cos\varphi+i\sin\varphi)$ - тригонометрична форма числа $z$.

\begin{thm}[Формула Ойлера]
	$$e^{iy}=\cos\varphi+i\sin\varphi\tab(\varphi\in\Rn)$$
\end{thm}
\begin{cons}
	$\\z=|z|e^{i\varphi}$ - показникова форма $z$. $\bar{z}=|z|(\cos(-\varphi+i\sin(-\varphi))=\underset{\textrm{не тригонометрична форма}}{|z|(\cos\varphi-i\sin\varphi)}$. $$\bar{z}=|z|e^{-i\varphi}$$
\end{cons}
\section{Операції над комплексними числами.}
\begin{enumerate}
	\item \underline{Порівняння}\\ В комплексній області відношення $">"$ чи $<$ не визначено. Числа порівнюють тільки за допомогою відношення $"="$ або $"\neq"$.
		\begin{align*}
			z_1=z_2\textrm{ у алгебраїчній формі }&\Longleftrightarrow\begin{cases}
				x_1=x_2\\y_1=y_2
			\end{cases}\\
			z_1=z_2\textrm{ у тригонометричній формі }&\Longleftrightarrow\begin{cases}
				|z_1|=|z_2|\\\varphi=\varphi+2\pi k,k\in\Zn
			\end{cases}
		\end{align*}
	\item \underline{Додавання/Віднімання}
		\begin{figure*}[htp]
		\begin{center}
			\begin{tikzpicture}[scale=0.7]
				\begin{axis}[ticks=none,axis lines=middle, xmin=-0.1, xmax=0.5,ymin=-0.1,ymax=1]
				\end{axis}	
				\draw (3.5,6) node[below] {\scriptsize 1: Додавання};
				\draw[-Stealth] (1.15,0.52) -- (4.6,3.65) node[right] {$z_1+z_2$};
				\draw[-Stealth] (1.15,0.52) -- (2.3,2.6) node[left] {$z_2$};
				\draw[dashed] (2.3,2.6) -- (4.6,3.65) node[left] {};
				\draw[-Stealth] (1.15,0.52) -- (3.44,1.55) node[below] {$z_1$};
				\draw[dashed] (3.44,1.55) -- (4.6,3.65) node[left] {};
			\end{tikzpicture}
			\begin{tikzpicture}[scale=0.7]
				\begin{axis}[ticks=none,axis lines=middle, xmin=-0.2, xmax=0.5,ymin=-0.1,ymax=1]
				\end{axis}	
				\draw (4,6) node[below] {\scriptsize 2: Віднімання};
				\draw[-Stealth] (1.95,0.52) -- (3,2.6) node[left] {$z_2$};
				\draw[-Stealth] (1.95,0.52) -- (4.9,1.55) node[right] {$z_1$};
				\draw[-Stealth] (4.9,1.55) -- (3,2.6) node[above,midway,sloped] {$|z_1-z_2|$};
				\draw[-Stealth] (1.95,0.52) -- (0,1.55) node[above,midway,sloped] {$z_2-z_1$};
			\end{tikzpicture}
			\begin{tikzpicture}[scale=0.7]
				\begin{axis}[ticks=none,axis lines=middle, xmin=-0.1, xmax=0.5,ymin=-0.1,ymax=1]
				\end{axis}	
				\draw (3.5,6) node[below] {\scriptsize 3:};
				\draw (5,3) arc (0:360:1.5);
				\draw (3.5,3) node[circle,fill,inner sep=0.5pt] {};
				\draw (3.5,3) node[below] {$z_0$};
				\draw (3.5,3) -- (4.6,4) node[sloped,midway,above] {$R$};
				\draw (4.6,4) node[right] {$z_0$};
			\end{tikzpicture}
		\end{center}
		\end{figure*}
		$$1:z_1+z_2=(x_1+x_2)+i(y_1+y_2)$$
		$$2:z_1-z_2=(x_1-x_2)+i(y_1-y_2)$$
		$$2:|z_1-z_2|=\sqrt{(x_1-x_2)^2+(y_1-y_2)^2}\textrm{ - відстань між }z_1,z_2$$
		$$3:|z-z_0|=R$$
	\item \underline{Множення/ділення і підносення до степеню}\\\\
		$\deff z_1\cdot z_2=\underbrace{x_1x_2-y_1y_2}_{\Re z}+i\underbrace{(x_1y_2+x_2y_1)}_{\Im z}$
		\begin{cons}
			для $z_1=z_2=i:i^2=-1\tab (x_1=x_2=0,y_1=y_2=1)$
		\end{cons}
		\begin{cons}
			$z_1\cdot z_2=x_1x_2+i^2y_1y_2+ix_1y_2+ix_2y_1=x_1(x_2+iy_2)+iy_1(x_2+iy_2)=\\=(x_1+iy_1)(x_2+iy_2)$
		\end{cons}
		Таким чином, комплексні числа перемножаються як звичайні і при цьому \\зберігаються усі формули скороченого множення.\\\\
		$\deff \dfrac{z_1}{z_2}=\dfrac{x_1+iy_1}{x_2+iy_2}\cdot\dfrac{x_2-iy_2}{x_2-iy_2}=\dfrac{A+iB}{x_2^2+y_2^2}=X+iY$\\
		$\deff z^n=(x+iy)^n=\sum\limits_{k=0}^nC_n^kx^k(iy)^{n-k},\tab C_n^k=\dfrac{n!}{k!(n-k)!}\\\left[\begin{array}{l}
			i=i\\i^2=-1\\i^3=-i\\i^4=i\\i^5=1
		\end{array}\right.\dots\left[\begin{array}{l}
			i^{4k}=1\\i^{4k+1}=i\\i^{4k+2}=-1\\i^{4k+3}=-i
		\end{array}\right.$
	\item \underline{Множення/ділення і підносення до степеню (в тригонометричнії і показниковій формі)}\\
		$z_1=|z_1|(\cos\varphi_1+i\sin\varphi_1),z_2=|z_2|(\cos\varphi_2+i\sin\varphi_2).$ Тоді у тригономтричній формі: $z_1\cdot z_2=|z_1|(\cos\varphi_1+i\sin\varphi_1)\cdot |z_2|(\cos\varphi_2+i\sin\varphi_2)=|z_1|\cdot|z_2|\cdot(\cos\varphi_1\cos\varphi_2+\\+i\cos\varphi_1\sin\varphi_2+i\sin\varphi_1\cos\varphi_2+i^2\sin\varphi_1\sin\varphi_2)=|z_1|\cdot|z_2|\cdot((\cos\varphi_1\cos\varphi_2-\sin\varphi_1\sin\varphi_2)++i(\sin\varphi_1\cos\varphi_2+\cos\varphi_1\sin\varphi_2)=|z_1|\cdot|z_2|\cdot(\cos(\varphi_1+\varphi_2)+i\sin(\varphi_1+\varphi_2)\\\\$
		$\deff z_1\cdot z_2=|z_1|\cdot|z_2|\cdot(\cos(\varphi_1+\varphi_2)+i\sin(\varphi_1+\varphi_2))\\\tab\tab\tab\tab\tab\tab|z_1\cdot z_2|=|z_1|\cdot|z_2|,\varphi=\varphi_1+\varphi_2$
		\begin{cons}
			$z_1=z,z_2=\bar{z}\Rightarrow z\cdot\bar{z}=|z|^2(\cos0+i\sin0)=|z|^2$
		\end{cons}
		У показниковій формі(множення):
		$z_1=|z_1|\cdot e^{i\varphi_1},z_2=|z_2|\cdot e^{i\varphi_2}\\\\$
		$\deff z_1\cdot z_2=|z_1|\cdot|z_2|\cdot e^{i(\varphi_!+\varphi_2)}\\$
		У тригонометричній формі(ділення): $\dfrac{z_1}{z_2}=\dfrac{|z_1|}{|z_2|}\cdot\dfrac{\cos\varphi_1+i\sin\varphi_1}{\cos\varphi_2-i\sin\varphi_2}=\dfrac{|z_1|}{|z_2|}\cdot\\\dfrac{(\cos\varphi_1+i\sin\varphi_1)(\cos\varphi_2+i\sin\varphi_2)}{\cos^2\varphi_2+\sin^2\varphi_2}=\dfrac{|z_1|}{|z_2|}\cdot(\cos\varphi_1+i\sin\varphi_1)\cdot(\cos(-\varphi_2)+i\sin(-\varphi_2))\\\\$
		$\deff \dfrac{z_1}{z_2}=\dfrac{|z_1|}{|z_2|}(\cos(\varphi_1-\varphi_2)+i\sin(\varphi_1-\varphi_2))),\tab\left|\dfrac{z_1}{z_2}\right|=\dfrac{|z_1|}{|z_2|},\varphi=\varphi_1-\varphi_2\\\\$
		У показниковій формі(ділення):\\\\
		$\deff \dfrac{z_1}{z_2}=\dfrac{|z_1|}{|z_2|}\cdot e^{i(\varphi_1-\varphi_2)}\\\\$
		$\deff z^n=\underbrace{z\cdot z\cdot z\cdot\dots\cdot z }_{n\textrm{ разів}}=|z|^n(\cos n\varphi+i\sin n\varphi)=|z|^n\cdot e^{in\varphi},\forall n\in\Zn$
	\item \underline{Винесення з під кореня $\sqrt[n]{z}$}\\\\
		$\deff W=\sqrt[n]{z},$ якщо $W^n=z$.\\\\
		Нехай обидва записані у тригонометричній формі: $z=|z|(\cos\varphi+i\sin\varphi),\\W=|W|(\cos\psi+i\sin\psi),W^n=|W|^n(\cos n\psi+i\sin n\psi)$. Умова "=" в тригоно-\\-метричній формі: $W^n=z\Rightarrow\left\{\begin{array}{l}
			|W|^n=|z|\\n\psi=\varphi+2\pi k,k\in\Zn
		\end{array}\right.\Rightarrow\left\{\begin{array}{l}
			|W|=\sqrt[n]{|z|}\\\psi=\dfrac{\varphi+2\pi k}{n},k\in\Zn
		\end{array}\right.$
	\begin{figure*}[htp!]
		\begin{wrapfigure}[11]{l}{0.4\textwidth} 
    		\centering
    		\begin{tikzpicture}[scale=0.8]
    			\begin{axis}[ticks=none,axis lines=middle, xmin=-0.5, xmax=0.5,ymin=-0.5,ymax=0.5]
				\end{axis}	
				\draw (4.5,3.6) node[below] {\tiny$\dfrac{2\pi}{n}$};
				\draw (4,4.1) node[below] {\tiny$\dfrac{2\pi}{n}$};
				\draw (4.9,2.9) arc (0:360:1.5);
				\draw (3.9,2.85) arc (0:108:0.5);
				\draw (3.43,2.85) node[circle,fill,inner sep=0.5pt] {};
				\draw (4.64, 3.74) node[circle,fill,inner sep=0.5pt] {};
				\draw (3.88, 4.32) node[circle,fill,inner sep=0.5pt] {};
				\draw (2.88, 4.31) node[circle,fill,inner sep=0.5pt] {};
				\draw (3.43,2.85) -- (4.65, 3.75) node[above, right] {\scriptsize$\dfrac{\varphi}{n}(k=0)$};
				\draw (3.43,2.85) -- (3.88, 4.33) node[above] {\scriptsize$k=1$};
				\draw (3.43,2.85) -- (2.88, 4.31) node[above] {\scriptsize$k=2$};
    		\end{tikzpicture}
		\end{wrapfigure}$\\\\\sqrt[n]{z}=W=\sqrt[n]{|z|}\left(\cos\dfrac{\varphi+2\pi k}{n}+i\sin\dfrac{\varphi+2\pi k}{n}\right)\\k=0,1,\dots,n-1\\\psi_k=\dfrac{\varphi}{n}+2\pi\dfrac{k}{n},\tab\Delta\psi=\dfrac{2\pi}{n}\\$
	\end{figure*}
	\begin{example}
		\begin{wrapfigure}{l}{0.5\textwidth} 
			\centering
    		\begin{tikzpicture}[scale=0.4]
    			\begin{axis}[ticks=none,axis lines=middle,thick, xmin=-1, xmax=1,ymin=-1,ymax=1]
				\end{axis}	
				\draw (3.43,2.85) node[circle,fill,inner sep=0.5pt] {};
				\draw (3.43,1.4) node[circle,fill,inner sep=0.5pt] {};
				\draw (3.43,4.4) node[circle,fill,inner sep=0.5pt] {};
				\draw (1.9,2.85) node[circle,fill,inner sep=0.5pt] {};
				\draw (4.9,2.85) node[below] {\scriptsize1};
				\draw (4.9,2.85) node[circle,fill,inner sep=0.5pt] {};
    		\end{tikzpicture}
		\end{wrapfigure}
		$\\\\\\\sqrt[4]{1}=\Delta\varphi=\dfrac{2\pi}{4}=\dfrac{\pi}{2}\\$
	\end{example}
\end{enumerate}
\section{Послідовності комплексних чисел.}
Послідовність комплексних чисел --- це комплекснозначна функція натурального аргумента. $\{z_n\}$ - послідовність.\\\\
$\deff n\in\Nn\rightarrow f(n)=z_n\in\Cn$
\begin{enumerate}
	\item \tab\begin{figure*}[htp!]
		\begin{wrapfigure}{l}{0.4\textwidth} 
			\begin{tikzpicture}[scale=0.8]
				\begin{axis}[ticks=none,axis lines=middle, xmin=-0.1, xmax=0.5,ymin=-0.1,ymax=0.5,xshift=-4cm]
				\end{axis}	
				\draw (1,3.5) arc (0:360:1);
				\draw (0,3.5) node[circle,fill,inner sep=0.5pt] {};
				\draw (0,3.5) -- (0.7,4.2) node[midway,sloped,above] {$\varepsilon$};
				\draw[->] (2.3, 2) -- (0.7,3.2);
				\draw (3, 1.6) node[below,left] {\scriptsize$\forall n\geq N(\varepsilon)$};
				\pgfmathsetseed{24122015}
    			\foreach \p in {1,...,200}
    			{ \pgfmathsetmacro{\x}{2*rand}
        			\pgfmathsetmacro{\y}{-rand*sqrt(4-pow(\x,2))+3.5}
        			\fill[black]    (\x,\y) circle (0.02);
    			}
			\end{tikzpicture}
		\end{wrapfigure}
		$\\\\\limntoinf z_n=z_0,z_0\in\Cn\\\\\tab\tab\Updownarrow\\\\\forall\varepsilon\tab\exists N=N(\varepsilon)\in\Nn:\forall n\geq N(\varepsilon)\tab|z_n-z_0|<\varepsilon$
	\end{figure*}\newpage
	\item \tab\begin{figure*}[htp]
		\begin{wrapfigure}{l}{0.4\textwidth} 
			\begin{tikzpicture}[scale=0.8]
			\begin{axis}[ticks=none,axis lines=middle, xmin=-0.5, xmax=0.5,ymin=-0.5,ymax=0.5,xshift=-3.4cm]
			\end{axis}
			\pgfmathsetseed{24122015}
    		\foreach \p in {1,...,200}
   			{ \pgfmathsetmacro{\x}{2*rand}
       			\pgfmathsetmacro{\y}{-rand*sqrt(4-pow(\x,2))+2.8}
       			\fill[black]    (\x,\y) circle (0.02);
   			}
   			\draw (0,2.8) circle (1);
   			\draw (0.9,3.1) node[right] {\scriptsize$M$};
   			\draw (1,2.85) node[circle,fill,inner sep=1pt] {};
		\end{tikzpicture}
		\end{wrapfigure}
		$\\\\\\{z_n}$ - обмежена, якщо $\\\\\exists M>0\tab\forall n\in\Nn\tab|z_n|<M$
	\end{figure*}\\
\end{enumerate}$\\$
\begin{thm}\label{thm:zn+z0}
	\begin{align*}
		\textrm{Нехай}\tab& z_n=x_n+iy_n,&\\
		&z_0=x_0+iy_0,&(x_n,y_n,x_0,y_0\in\Rn)\\
		\textrm{Тоді:}\tab& \limntoinf z_n=z_0\tab\Longleftrightarrow\tab 
		\begin{cases}
			\limntoinf x_n=x_0\\\limntoinf y_n=y_0
		\end{cases}
	\end{align*}
\end{thm}
\begin{figure*}[htp]
	\begin{center}
		\begin{tikzpicture}[scale=1]
				\begin{axis}[ticks=none,axis lines=middle, xmin=-0.1, xmax=0.5,ymin=-0.1,ymax=1]
				\end{axis}	
				\draw (4,5) node[below] {$|z_1+z_2|\leq|z_1|+|z_2|$};
				\draw[-Stealth] (1.15,0.52) -- (4.6,3.65) node[above,midway,sloped] {\tiny$|z_1+z_2|$} node[right] {$z_1+z_2$};
				\draw[-Stealth] (1.15,0.52) -- (2.3,2.6) node[above,midway,sloped] {\tiny$|z_2|$} node[left] {$z_2$};
				\draw[dashed] (2.3,2.6) -- (4.6,3.65) node[left] {};
				\draw[-Stealth] (1.15,0.52) -- (3.44,1.55) node[above,midway,sloped] {\tiny$|z_1|$} node[below] {$z_1$};
				\draw[dashed] (3.44,1.55) -- (4.6,3.65) node[left] {};
			\end{tikzpicture}
			\caption*{До доведення \cref{thm:zn+z0}}
	\end{center}
\end{figure*}
\begin{prooff}$ $
	\begin{enumerate}[label=\alph*)]
		\item необхідність:\\
			Нехай $\limntoinf z_n=z_0$, тобто $\forall\varepsilon>0\tab\exists N(\varepsilon)\in\Nn\tab\forall n\geq N(\varepsilon)\tab|z_n-z_0|<\varepsilon\\|z_n-z_0|=|(x_n+iy_n)-(x_0+iy_0)|=|(x_n-x_0)+i(y_n-y_0)|\Longrightarrow\\\Longrightarrow\sqrt{(x_n-x_0)^2+i(y_n-y_0)^2}<\varepsilon ,\tab (x_n-x_0)^2+i(y_n-y_0)^2<\varepsilon^2\Longrightarrow\\\Longrightarrow\begin{cases}
				(x_n-x_0)^2<\varepsilon^2\\(y_n-y_0)^2<\varepsilon^2
			\end{cases}\Longrightarrow\begin{cases}
				|x_n-x_0|<\varepsilon\\|y_n-y_0|<\varepsilon
			\end{cases},\forall n\geq N(\varepsilon)\tab\Longleftrightarrow$
			\begin{center}
				$\Longleftrightarrow\tab \begin{cases}
			\limntoinf x_n=x_0\\\limntoinf y_n=y_0
			\end{cases}$
			\end{center}
		\item достатність:\\Нехай $\begin{cases}
			\limntoinf x_n=x_0\\\limntoinf y_n=y_0
			\end{cases}\tab\begin{array}{ll}
				\forall\varepsilon>0:&\exists N_1(\varepsilon)\in\Nn:\forall n\geq N_1(\varepsilon)\tab |x_n-x_0|<\frac{\varepsilon}{2} \\
				&\exists N_2(\varepsilon)\in\Nn:\forall n\geq N_2(\varepsilon)\tab |y_n-y_0|<\frac{\varepsilon}{2}
			\end{array}\\|z_n-z_0|=|(x_n+iy_n)-(x_0+iy_0)|=|(x_n-x_0)+i(y_n-y_0)|\leq\\\leq|x_n-x_0|+|i|\cdot|y_n-y_0|=|x_n-x_0|+|y_n-y_0|<\dfrac{\varepsilon}{2}+\dfrac{\varepsilon}{2}=\varepsilon$. \\при $\forall n\geq\max(N_1,N_2)\Longrightarrow\limntoinf z_n=z_0$
	\end{enumerate}
\end{prooff}
\begin{statement}
	$$\limntoinf z_n=z_0\Longleftrightarrow\limntoinf|z_n-z_0|=0$$
\end{statement}
\begin{prooff}
	$\\\forall\varepsilon>0\tab\exists N(\varepsilon)\in\Nn:\forall n\geq N(\varepsilon)\tab|z_n-z_0|<\varepsilon$. Тоді: $||z_n-z_0|-0|=|z_n-z_0|<\varepsilon$
\end{prooff}
\begin{thm}\label{thm:zn-z0}
	$$\textrm{Якщо }\limntoinf z_n=z_0,\textrm{ то }\limntoinf|z_n|=|z_0|$$
\end{thm}
\begin{figure*}[htp]
	\begin{center}
		\begin{tikzpicture}
				\begin{axis}[ticks=none,axis lines=middle, xmin=-0.2, xmax=0.5,ymin=-0.1,ymax=1]
				\end{axis}	
				\draw[-Stealth] (1.95,0.52) -- (3,2.6) node[above,midway,sloped] {\tiny$|z_n|$} node[left] {$z_n$};
				\draw[-Stealth] (1.95,0.52) -- (4.9,1.55) node[below,midway,sloped] {\tiny$|z_0|$} node[right] {$z_0$};
				\draw[-Stealth]  (4.9,1.55) -- (3,2.6) node[above,midway,sloped] {$|z_n-z_0|$};
			\end{tikzpicture}
			\caption*{До доведення \cref{thm:zn-z0}}
	\end{center}
\end{figure*}
\begin{prooff}$\\$
	Покажемо, що $\limntoinf||z_n|-|z_0||=0$ (у зворотний бік невірно).\\
	$\left.\begin{array}{ll}
		\textrm{Нерівність }\triangle \textrm{ для }|z_n|:&|z_n|\leq|z_0|+|z_n-z_0|\\&|z_n|-|z_0|\leq|z_n-z_0|\\\textrm{Нерівність }\triangle \textrm{ для }|z_-0|:&|z_0|\leq|z_n|+|z_n-z_0|\\&|z_0|-|z_n|\leq|z_n-z_0|
	\end{array}\right\}\Longrightarrow||z_n|-|z_0||\leq|z_n-z_0|$\\
	Окрім того, $0\leq||z_n|-|z_0||\leq|z_n-z_0|\Longrightarrow(\limntoinf0=0,\limntoinf|z_n-z_0|=0$ - за умовою) $\Longrightarrow\limntoinf||z_n|-|z_0||=0\Longleftrightarrow\limntoinf|z_n|=|z_0|$
\end{prooff}
\begin{thm}
	$$\begin{cases}
		\limntoinf|z_n|=|z_0|\\\limntoinf\varphi_n=\varphi_0
	\end{cases}\xRightarrow[\begin{array}{l}
		{\scriptscriptstyle z_n=|z_n|\cdot e^{i\varphi_n}}\\{\scriptscriptstyle z_0=|z_0|\cdot e^{i\varphi_0}}
	\end{array}]{}\limntoinf z_n=z_0$$
\end{thm}
\begin{prooff}
	З арифметичних властивостей $\lim z_n$
\end{prooff}
\section{Розширена множина $\Cn$. Нескінченно віддалена точка.}
\begin{figure*}[htp!]
		\begin{wrapfigure}{l}{0.45\textwidth} 
			\begin{tikzpicture}
				\begin{axis}[ticks=none,axis lines=middle, xmin=-1, xmax=1,ymin=-1,ymax=1,xshift=-3.4cm]
				\end{axis}	
				\draw (0,2.9) circle (30pt);
				\draw[->] (3, 5) -- (2,4);
				\draw (3, 5) node[right,above] {\scriptsize$\forall n\geq N(E)$};
				\pgfmathsetseed{24122015}
    			\foreach \p in {1,...,200}
    			{ \pgfmathsetmacro{\x}{2*rand}
        			\pgfmathsetmacro{\y}{-rand*sqrt(4-pow(\x,2))+2.9}
        			\fill[black]    (\x,\y) circle (0.02);
    			}
			\end{tikzpicture}
		\end{wrapfigure}
		$\\\\\limntoinf z_n=\infty\Leftrightarrow\forall E>0\tab\exists N(E)\in\Nn\\\forall n\geq N(E)\tab |z_n|>E.$ Невласне комплексне число $\infty$: поняття дійсної та уявної частини, а також, аргумента - невизначені. $|\infty|=\infty.\tab \limntoinf z_n=\infty\Rightarrow\limntoinf|z_n|=+\infty.\tab\limntoinf\dfrac1{z_n}=0\\\\\\$
\end{figure*}
\begin{figure*}[htp]
	\begin{wrapfigure}{l}{0.65\textwidth} 
		\underline{Операції} над $"\infty"$ і $a\in\Cn:\\\\$
	$\deff "\infty\pm a"=\infty\\\deff"\infty\cdot a=\infty,\tab a\neq0"\\\deff"\dfrac\infty a"=\infty\\\deff"\dfrac a\infty"=0\\\deff"\infty\cdot\infty"=\infty\\\\$
	\end{wrapfigure}
	$\\$\underline{Невизначеності}: $0\cdot\infty,\tab\dfrac\infty\infty,\tab\dfrac00,\tab \infty+\infty,\tab\infty-\infty.\\\\\\\\$
\end{figure*}

\begin{figure*}[htp]\centering
	\begin{tikzpicture}
		\begin{axis}[ticks=none,axis lines=middle, xmin=-0.1, xmax=0.9,ymin=-0.5,ymax=0.5]
				\end{axis}
				\draw (3.5,0) -- (3.5,5.5);
				\draw[->] (0.69,2.85) -- (3.5,5);
				\draw[->] (0.69,2.85) -- (3.5,4.5);
				\draw[->] (0.69,2.85) -- (3.5,4);
				\draw[->] (0.69,2.85) -- (3.5,0.7);
				\draw[->] (0.69,2.85) -- (3.5,1.2);
				\draw[->] (0.69,2.85) -- (3.5,1.7);
				\draw (4.5,2.85) node[circle,fill,inner sep=1pt] {};
				\draw (5,2.85) node[circle,fill,inner sep=1pt] {};
				\draw (5.5,2.85) node[circle,fill,inner sep=1pt] {};
				\draw (5,4) node {\scriptsize результат};
				\draw (5,3.5) node {\scriptsize (постійна)};
	\end{tikzpicture}\tab
	\begin{tikzpicture}
		\coordinate (P) at ({1/sqrt(3)},{1/sqrt(3)},{1/sqrt(3)});
 
% Draw circle
		\draw (0,1,0) circle (1cm);
% draw arcs 		
		\tdplotsetrotatedcoords{0}{0}{0};
		\draw[dashed, tdplot_rotated_coords, gray] (0,0,1) circle (1);
% Axes in 3 d coordinate system
		\draw[-stealth] (0,0,0) -- (4,0,0) node[below left] {$x$}; 
		\draw[-stealth] (0,0,0) -- (0,3,0) node[below right] {$y$};
		\draw[-stealth] (0,0,0) -- (0,0,5) node[above] {$z$};
% nodes		
		\draw (0,1,0) node[circle,fill,inner sep=1pt] {};
		\draw (0,2,0) node[circle,fill,inner sep=1pt] {};
		\draw (1.2,0.97,1) node[circle,fill,inner sep=1pt] {};
		\draw (3,0,3) node[circle,fill,inner sep=1pt] {};
		\draw (3,3,0) node {$z\leftrightarrow z'$};
		\draw (3,2,0) node {$\infty\leftrightarrow N$};
		\draw (1.5,-2,0) node {\scriptsize$N$ образ нескінченно віддаленої точки};
% lines		
		\draw[dashed]  (0,2,0) node[anchor=south west] {$N$} -- (1.2,0.97,1);
		\draw (1.2,0.97,1) -- (3,0,3) node[below] {$z$};
		\draw[->] (1.2,0.97,1) -- (2,0.5,-0.5) node[right] {$z'$};

	\end{tikzpicture}
\end{figure*}\newpage
\section{Множина на комплексній площині.}
Областю на комплексній площині називається множина точок, що володіє властиво-стями відкритості та зв'язності. Відкритість означає, що будь-яка точка множини належить їй разом з деяким околом. Зв'язність означає, що будь-які дві точки множини можна з'єднати лінією, що складається цілком з точек цієї ж множини. Точка, що сама не належить області, але будь-якій її окіл має в собі точки цієї області, називається граничною точкою області. Сукупність граничних точок області називається границею області. \\Далі будемо вважати, що границя області може складатись із скінченого числа замкнутих ліній (контурів), незамкнутих ліній (розрізів) та окремих точок. Область називається обмеженою, якщо її можна укласти всередину деякого кола з центром у початку координат. Область разом з приєднаною до неї границею називається замкненою областю. \\
\textbf{Позначення. }\\$D,G$ - області. $\Gamma,\gamma,L,l$ - границі області. $\overline{D}=D\cup\Gamma$ - замкнена область.
\begin{figure*}[htp]\centering
	\begin{tikzpicture}
		\draw (0,-2.5) node {однозв'язна область};
		\draw (0,0) node {\LARGE\textbf{D}};
		\draw (-3,0) -- (-3.3,0) node[left] {$\Gamma$};
	    \draw (0,0) ellipse (3 and 2);
 	    \clip (0,0) ellipse (3 and 2);
 		\pgfmathsetseed{24122015}
    	\foreach \p in {1,...,1000}
    	{ \pgfmathsetmacro{\x}{3*rand}
        	\pgfmathsetmacro{\y}{rand*sqrt(9-pow(\x,2))}
        	\fill[black]    (\x,\y) circle (0.02);
    	}
	\end{tikzpicture}\tab 
	\begin{tikzpicture}
		\draw (0,-2.5) node {$n$-зв'язна область};
		\draw (-1.4,0.4) node[left,below] {$\Gamma_1$};
		\draw (1.4,1) node {$\Gamma_2$};
		\draw (0.5,-1) node[right] {$\Gamma_3$};
		\draw (-2.5,0) node {\LARGE\textbf{D}};
	    \draw (0,0) ellipse (3 and 2);
 	    \clip (0,0) ellipse (3 and 2);
 		\pgfmathsetseed{24122015}
    	\foreach \p in {1,...,1000}
    	{ \pgfmathsetmacro{\x}{3*rand}
        	\pgfmathsetmacro{\y}{rand*sqrt(9-pow(\x,2))}
        	\fill[black]    (\x,\y) circle (0.02);
    	}
    	\draw[fill=white, rotate=15] (-1,1) ellipse (0.6 and 0.3);
    	\draw[fill=white, rotate=-35] (1,1) ellipse (0.6 and 0.3);
    	\draw[fill=white, rotate=-5] (0,-1) ellipse (0.6 and 0.3);
	\end{tikzpicture}
\end{figure*}\\
Порядком зв'язності області називається число зв'язних елементів її границі. Додат-нім напрямком обходу границі рахуєьтся той, при котрому область залишається зліва.
\begin{figure*}[htp]\centering
	\begin{tikzpicture}
		\draw (0,-1) node {2-зв'язна область};
	    \draw (0,0) ellipse (1 and 0.5);
 	    \clip (0,0) ellipse (1 and 0.5);
 		\pgfmathsetseed{24122015}
    	\foreach \p in {1,...,1000}
    	{ \pgfmathsetmacro{\x}{3*rand}
        	\pgfmathsetmacro{\y}{rand*sqrt(9-pow(\x,2))}
        	\fill[black]    (\x,\y) circle (0.02);
    	}
    	\draw[fill=white] (-0.5,0) circle (0.15);
    	\draw[fill=white] (0.5,0) circle (0.15);
	\end{tikzpicture}\tab\tab  
	\begin{tikzpicture}
 		\pgfmathsetseed{24122015}
    	\foreach \p in {1,...,200}
    	{ \pgfmathsetmacro{\x}{rand}
        	\pgfmathsetmacro{\y}{rand*\x*sqrt(1-abs(\x))}
        	\fill[black]    (\x,\y) circle (0.02);
    	}
    	\draw (0,-1) node {не область};
	\end{tikzpicture}
\end{figure*}
\section{Поняття функції комплексної змінної.}
$z\overset{f}{\longmapsto}W=f(z)\in\Cn,\tab z\in\Cn$
\begin{example}
	$\left.\begin{array}{l}
		f(z)=\Re z\\f(z)=\Im z\\f(z)=|z|
	\end{array}\right\}\in\Rn$, однозначні
\end{example}
\begin{example}
	$\left.\begin{array}{l}
		f(z)=z^n\\\bar{z}
	\end{array}\right\}\in\Cn$, однозначні
\end{example}
\begin{example}
	$f(z)=\Arg z=\arg z+2\pi k,k\in\Zn$ - многозначна(нескінченна кількість значень)
\end{example}
\begin{example}
	$f(z)=\sqrt[n]{z}\in\Cn,$ - многозначна ($n$ - значна)
\end{example}
$W=f(z)=u(x,y)+iv(x,y),\tab u,v\in\Rn$
\begin{figure*}[htp]\centering
	\begin{tikzpicture}
		\begin{axis}[ticks=none,axis lines=middle, xmin=-0.1, xmax=0.9,ymin=-0.1,ymax=0.9,xshift=-3.4cm]	
		\end{axis}
		\draw (-2.5,5.5) node {$y$};
		\draw (3,1) node {$x$};
		\draw (1,3) node[circle,fill,inner sep=1pt] {};
		\draw (1,3) node[right] {$z$};
		\draw (0,0) node {\small$z=x+iy,z\in D$};
		\draw (0,3) node {\textbf{D}};
		\draw (-0.5,1.8) -- (-1,1.5) node[left] {$\Gamma$};
	    \draw (0,3) ellipse (2 and 1);
 	    \clip (0,3) ellipse (2 and 1);
 		\pgfmathsetseed{24122015}
    	\foreach \p in {1,...,1000}
    	{ \pgfmathsetmacro{\x}{3*rand}
        	\pgfmathsetmacro{\y}{rand*sqrt(9-pow(\x,2))+2}
        	\fill[black]    (\x,\y) circle (0.02);
    	}	
    \end{tikzpicture}
    \begin{tikzpicture}
		\begin{axis}[ticks=none,axis lines=middle, xmin=-0.1, xmax=0.9,ymin=-0.1,ymax=0.9,xshift=-3.4cm]	
		\end{axis}
		\draw (-2.5,5.5) node {$y$};
		\draw (3,1) node {$x$};
		\draw (-0.8,3.5) node[circle,fill,inner sep=1pt] {};
		\draw (-0.8,3.5) node[right] {$W$};
		\draw (0,0) node{\small$W\in G=f(D),W=f(z)=u+iv$};
		\draw (0.8,2.5) node {\textbf{G}};
		%\draw (-0.5,1.8) -- (-1,1.5) node[left] {$\Gamma$};
 		\pgfmathsetseed{24122015}
    	\foreach \p in {1,...,300}
    	{ \pgfmathsetmacro{\x}{2*rand}
        	\pgfmathsetmacro{\y}{rand*sqrt(4-pow(\x,2))+3}
        	\fill[black]    (\x,\y) circle (0.02);
    	}
    	\draw[color=white,fill=white] (-1,3) -- (-1, 5) -- (-2, 5) -- (-2, 3);
    	\draw[color=white,fill=white] (-1,4) -- (1.5,4.5) -- (1.5,5) -- (-1, 5);
    	\draw[color=white,fill=white] (1.5,4.5) -- (1,2) -- (2,2) --  (2,4.5);
    	\draw[color=white,fill=white] (-1,3) -- (1,2) -- (1,1) -- (-2,3);
    	\draw[color=white,fill=white] (-2,3) -- (1,2) -- (1,1) -- (-2,1);
    	\draw[color=white,fill=white] (2,2) -- (1,2) -- (1,1) -- (2,1);
    	\draw (1,2) -- (-1,3) -- (-1, 4)  -- (1.5,4.5)  -- (1,2);
    	\draw (1.3,3) -- (2,3) node[right] {$f(\Gamma)$};
    \end{tikzpicture}
\end{figure*}
\begin{example}
	$\Gamma:x=1,f(z)=z^2,f(\Gamma)-?\\z^2=(x+iy)^2=\left.x^2-y^2+2ixy\right|_{x=1}=\underbrace{1-y^2}_{u}+\underbrace{2iy}_{v}\\y\in\Rn\Rightarrow v\in\Rn,y=\dfrac v2\Rightarrow u=1-\dfrac{v^2}{4}$
\end{example}
\begin{figure*}[htp]\centering
	\begin{tikzpicture}[scale=0.8]
		\begin{axis}[color=gray!60,ticks=none,axis lines=middle, xmin=-0.2,xmax=0.8,ymin=-0.5,ymax=0.5]	
		\end{axis}
		\draw[color=gray!60] (3,0.2) -- (3,5.5) node[right,color=gray!60] {$z=1+iy$};
		\draw[color=gray!60] (3,2.5) node[right] {1};
	\end{tikzpicture}
	\begin{tikzpicture}[scale=0.8]
		\begin{axis}[color=gray!60,ticks=none,axis lines=middle, xmin=-0.2,xmax=0.8,ymin=-0.5,ymax=0.5]	
		\end{axis}
		\draw[rotate=180,color=gray!60] (-1.4,-4.8) arc(270:90:0.7 and 2);
		\draw[color=gray!60] (1.2,0.8) node[left,color=gray!60] {-2} -- (1.4,0.8);
		\draw[color=gray!60] (1.2,4.8) node[left,color=gray!60] {-2} -- (1.4,4.8);
		\draw[color=gray!60] (2,2.5) node[right] {1};
	\end{tikzpicture}
\end{figure*}\newpage


























	\section{Основні елементарні функції комплексної змінної.}
\subsection{Показникова функція $\exp z$.}
$z\in\Cn,\tab z=xiy\\\\$
$\deff\exp z=e^x(\cos y+isiny)\\\\$
\textbf{Основні властивості}:
\begin{enumerate}
	\item При $z=x\in\Rn$ співпадає з $e^x$.
		\begin{prooff}
			$\exp z|_{z=x}=\exp x=\langle y=0\rangle=e^x$
		\end{prooff}
	\item При $z=iy\>(x=0):\tab \exp iy=e^0(\cos y+i\sin y)=e^{iy}$ - формула Ойлера.
	\item Зберігається властивість: $$\exp z_1\cdot\exp z_2=exp(z_1+z_2)$$ 
		\begin{prooff}
			$\\\exp z_1\cdot\exp z_2=e^{x_1}(\cos y_1+i\sin y_1)\cdot e^{x_2}(\cos y_2+i\sin y_2)=e^{x_1+x_2}\left(\cos(y_1\right.+\\+\left.y_2)+i\sin(y_1+y_2)\right)\eqdef{\exp}=\exp(x_1+x_2+i(y_1+y_2))=\exp(z_1+z_2)$
		\end{prooff}
		\begin{remark*}
			$\exp z=\exp(x+iy)=\exp z\cdot\exp iy=e^x\cdot e^{iy}=e^x(\cos y+isiny)$
		\end{remark*}
	\item $\forall z\tab\exp z\neq0$.
		\begin{prooff}
			$\\$Відомо, що $W=0\Longleftrightarrow|W|=0.\tab |\exp z|=|e^x(\cos y+isiny)|=|e^x|\cdot\\\cdot\underbrace{|\cos y+isiny|}_{|e^{i\varphi}|}=|e^x|\cdot\sqrt{\cos^ y+\sin^2y}=e^x>0\tab\forall z\in\Cn.$
		\end{prooff}
	\item Періодична $T=2\pi i$.
		\begin{prooff}$ $
			\begin{enumerate}
				\item Нехай $W=\exp z$. \\Тоді для $z+2\pi ik:\tab\exp(z+2\pi ik)=\exp(z+i(y+2\pi k))=e^x(\cos(y+\xcancel{2\pi k})+i\sin(y+\xcancel{2\pi k}))=\exp z=W$
				\item Нехай $W=\exp z_1$ і $W=\exp z_2$.\\ $e^{x_1}(\cos y_1+i\sin y_1)=e^{x_2}(\cos y_2+i\sin y_2)\Longrightarrow\begin{cases}
					e^{x_1}=e^{x_2}\\y_2=y_1+2\pi k,\tab k\in\Zn
				\end{cases}\\\Longrightarrow x_1=x_2$. Тоді $z_2-z_1=x_2+iy_2-(x_1+iy_1)=x_1+i(y_1+2\pi k)-(x_1+iy_1)=2\pi ik$
			\end{enumerate}
		\end{prooff}
\end{enumerate}
\subsection{Логарифмічна функція $\Ln z$.}
$z\in\Cn,\tab(z=x+iy)\\\\$
$\deff W=\Ln z,$ якщо $\exp W=z$
\textbf{Основні властивості}:
\begin{enumerate}
	\item $\Ln z$ - багатозначна, бо $\exp$ - періодична функція.
	\item Визначена на $\forall z\in\Cn\backslash\{0\}$ ($\exp$ не може перетворюватись на 0).
	\item При $z=x\in\Rn$ співпадає з $\Ln x$ (тому що $e^x$ буде обереною функцією).
	\item Зберігається властивість: $$\Ln(z_1\cdot z_2)=\Ln z_1+\Ln z_2$$
		\begin{prooff}
			$\\$Нехай $W_1=\Ln z_1,\>W_2=\Ln z_2\Longleftrightarrow z_1=\exp W_1,\>z_2=\exp W_2$. Знайдемо $z_1\cdot z_2:\tab z_1\cdot z_2=\exp W_1\cdot\exp W_2=\exp(W_1+W_2)$. Тоді $\Ln(z_1\cdot z_2)=W_1+W_2=\Ln z_1+\Ln z_2$.
		\end{prooff}
	\item Усі значення $\Ln z:$ $$\Ln z=\ln|z|+i\Arg z$$
		\begin{prooff}
			$\Ln z=\langle z=|z|(\cos(\arg z+2\pi k)+i\sin(\arg z+2\pi k))\rangle=\Ln(|z|\exp(i\Arg z)=\\=\Ln|z|+\Ln(\exp(\Arg z))=\langle|z|\in\Rn,\>|z|>0,\>z\neq 0\Rightarrow \Ln|z|=\ln|z|\rangle=\ln|z|+i\Arg z$
		\end{prooff}
	\item 
\end{enumerate}
\subsection{Тригонометричні функції.}
$z\in\Cn,\tab(z=x+iy)\\\\$
$\deff \sin z=\dfrac1{2i}(\exp(iz)-\exp(-iz))\\\\\tab\tab\tab\tab\>\cos z=\dfrac12(\exp(iz)-\exp(-iz))\\\\\tab\tab\tab\tab\>\tan z=\dfrac{\sin z}{\cos z},\tab \cot z=\dfrac{\cos z}{\sin z}\\\\$
\textbf{Основні властивості}:
\begin{enumerate}
	\item Визначені на $\forall z\in\Cn:$ $$\sin(-z)=-\sin z,\tab \cos(-z)=\cos z$$
	\item При $z=x\in\Rn$ співпадає з $\sin x,\>\cos x$.
		\begin{prooff}
			$\\\sin z|_{z=x}=\dfrac1{2i}(\exp(ix)-\exp(-ix))=\dfrac1{2i}(\cos x+i\sin x-(cos x-i\sin x))=\sin x$. Аналогічно з $\cos z|_{z=x}=\cos x$. 
		\end{prooff}
	\item ($\exp z$ - період $T=2\pi i\Longrightarrow\exp iz :T=2\pi$) $$T_{\sin z,\cos z}=2\pi,\tab T_{\tan z,\cot z}=\pi$$
	\item Зберігаються усі тригинометричні формули. Зокрема:
		$$\sin(z_1+z_2)=\sin z_1\cos z_2+\cos z_1\sin z_2$$
		$$\cos(z_1+z_2)=\cos z_1\cos z_2+\sin z_1\sin z_2$$
		\begin{prooff}
			\begin{align*}
				\deff \sin z,\>\cos z\tab\Longrightarrow\tab & \textrm{1) }\exp iz=\cos z+i\sin z \tag{$\ast$}\\
				&\textrm{2) }\exp(-iz)=\cos z-i\sin z \tag{$\ast\ast$}
			\end{align*}
			При $z=z_1+z_2:\\$
			\begin{enumerate}[label=\arabic*)]
				\item $\exp i(z_1+z_2)\equset{\textrm{в-сть }exp}\exp iz_1\cdot\exp iz_2\equset{(\ast)}(\cos z_1+i\sin z_1)(\cos z_2+i\sin z_2)=\\=\cos z_1\cdot\cos z_2-\sin z_1\cdot\sin z_2+i(\sin z_1\cdot\cos z_2+\cos z_2\cdot\sin z_2)$
				\item $\exp(-i(z_1+z_2))=\exp(-iz_1)\cdot\exp(-iz_2)=(\cos z_1+i\sin z_1)(\cos z_2+i\sin z_2)=\\=\cos z_1\cdot\cos z_2-\sin z_1\cdot\sin z_2-i(\sin z_1\cdot\cos z_2+\cos z_2\cdot\sin z_2)$
			\end{enumerate}
			$\\\dfrac{(1)+(2)}{2}:\tab \cos z_1\cdot\cos z_2-\sin z_1\cdot \sin z_2=\dfrac12(\exp i(z_1+z_2)+\exp(-i(z_1+z_2))\eqdef{}\\=\cos(z_1+z_2)\\\\\dfrac{(1)-(2)}{2}:\tab\sin z_1\cdot \cos z_2+\cos z_1\cdot\sin z_2=\dfrac1{2i}(\exp i(z_1+z_2)-\exp(-(z_1+z_2))\eqdef{}\\=\sin(z_1+z_2)$
		\end{prooff}
		Надаючи $z_1$ та $z_2$ різні значення, можжна отримати усі інші тригонометричні формули. При $z=z_1=z_2:\tab \sin z\cdot\cos z+\cos z\cdot\sin z=2\sin z\cdot\cos z=\sin 2z$. При $z_1=z,z_2=-z:\cos z\cdot\cos(-z)-\sin z\cdot\sin(-z)=\cos^2z+\sin^2z=1=\cos 0$. При $z_1=\dfrac\pi2,z_2=z:\sin\dfrac\pi2\cdot\cos z+\cos\dfrac\pi2\cdot\sin z=\cos z=\sin\left(\dfrac\pi2+z\right)$
	\item \begin{align*}
		\sin z=0\tab\Longleftrightarrow\tab&z=\pi n,\tab n\in\Zn\\
		\cos z=0\tab\Longleftrightarrow\tab&z=\dfrac\pi2+\pi n,\tab n\in\Zn
	\end{align*}
	\begin{prooff}$\>$
		\begin{enumerate}[label=\arabic*)]
			\item $\sin z=0\tab \Longleftrightarrow\tab\dfrac1{2i}(\exp(iz)-\exp(-iz))=0\tab \Longleftrightarrow\tab\exp i(x+iy)=\\=\exp(-i(x+iy).\tab\exp(-y+ix)=\exp(y-ix).\tab \Longleftrightarrow\tab e^{-y}=\\=(\cos x+i\sin x)=e^y(\cos(-x)+i\sin(-x))\tab \Longleftrightarrow\tab$ "=" в триг формі. $\begin{cases}
				e^{-y}=e^y\\ x=-x+2\pi n,\tab n\in\Zn
			\end{cases},
			\begin{cases}
				y=0\\x\pi n,\tab n\in\Zn 
			\end{cases},$ $\\z=x+iy=\pi n,\tab n\in\Zn$
			\item $\cos z=\sin\left(\dfrac\pi2+z\right)=0$
				$\dfrac\pi2+z=\pi k,\tab k\in\Zn,\tab z=-\dfrac\pi2+\pi k=\\=\pi-\dfrac\pi2+\pi(k-1)=\dfrac\pi2+\pi n,\tab n\in\Zn$
		\end{enumerate}
	\end{prooff}
	\item $\sin z,\>\cos z$ не обмежені.\\
		Наприклад для $z=\dfrac\pi2+iy,\tab y\in\Rn:\tab \sin z=\sin\left(\dfrac\pi2+iy\right)=\dfrac1{2i}\left(\exp\left(i\left(\dfrac\pi2+iy\right)\right)\right.-\\-\left.\exp\left(-i\left(\dfrac\pi2+iy\right)\right)\right)=\dfrac1{2i}\left(\exp\left(i\dfrac\pi2-y\right)-\exp\left(y-i\dfrac\pi2\right)\right)=\dfrac1{2i}\left(e^{-y}\left(\cos\dfrac\pi2+i\sin\dfrac\pi2\right)\right.-\\-\left.e^{-y}\left(\cos\dfrac\pi2-i\sin\dfrac\pi2\right)\right)=\dfrac12\left(e^{-y}+e^{y}\right).$
\end{enumerate}
\subsection{Гіперболічні функції.}
$z=x+iy,\tab x,y\in\Rn\\\\$
$\deff\sh z=\dfrac12(\exp z-\exp(-z))\\\tab\tab\tab\tab\>\ch z=\dfrac12(\exp z+\exp(-z)\\\tab\tab\tab\tab\>\th z=\dfrac{\sh z}{\ch z}\\\tab\tab\tab\tab\>\cth z=\dfrac{\ch z}{\sh z}\\\\$
\textbf{Основні властивості}:
\begin{enumerate}
	\item Визначені на $\forall z\in\Cn:$ $$\sh(-z)=-\sh(z),\tab\ch(-z)=\ch(z)$$
	\item При $z=x\in\Rn$ співпадає з $\sh x,\ch x$
		\begin{prooff}
			В цьому випадку $\exp x= e^x$, і звідци це встановлюється.
		\end{prooff}
	\item Періодичні: $$T_{\sh,\ch}=2\pi i,\tab T_{\th,\cth}=\pi i$$
	\item Зв'язок з тригонометричними функціями: $$\sin iz=i\sh z,\tab \cos iz=i\ch z$$
\end{enumerate}
Обернені тригонометричні і гіперболічні функції:\\\\ $\deff W=\arcsin z,\tab \sin W=z$
\subsection{Степінь з комплексним показником. Загальна степенева $(z^\alpha)$ та показникова $\alpha^z$ функції}
Нехай $\alpha,\beta\in\Cn,\tab\alpha^\beta$ - множина значень. В загальному випадку $\alpha^{\beta_1}\cdot\alpha^{\beta_2}\neq\alpha^{\beta_1+\beta_2}.\\\\$
$\deff\alpha^\beta=\exp(\beta\Ln\alpha)\\\\$
Загальна степенева функція: 
$\\\\\deff f(z)=z^\alpha=\exp(\alpha\Ln z),\tab\alpha\in\Cn\\$
$z^\alpha=\exp(\alpha\Ln z)=\langle\alpha=a+ib,\tab a,b\in\Rn,\tab z=|z|\cdot e^{i\varphi}\rangle=\exp((a+ib)(\ln|z|+\\+i(\varphi+2\pi k)))=\exp(a\ln|z|-b(\varphi+2\pi k)+i(a(\varphi+2\pi k)+b\ln|z|))=e^{a\ln|z|-b(\varphi+2\pi k)}\cdot\\\cdot(\cos(a(\varphi+2\pi k)+b\ln|z|)+i\sin(a(\varphi+2\pi k)+b\ln|z|))\\\\$
При $\alpha\in\Rn:$
\begin{itemize}
	\item[-]  якщо $\alpha=n\in\Zn\>(a=n\in\Zn,b=0):\\z^\alpha|_{\alpha=n}=e^{n\ln|z|}\cdot(\cos(n(\varphi+2\pi k)))+i\sin(n(\varphi+2\pi k))=e^{n\ln|z|}\cdot(\cos n\varphi+i\sin n\varphi)=\\=|z|^n,\tab n\in\Zn$
	\item[-]  якщо $\alpha=\dfrac1n\in\Rn\>\left(a=\dfrac1n\in\Zn,b=0\right):\\z^\alpha|_{\alpha=\frac1n}=e^{\frac1n\ln|z|}\cdot\left(\cos\left(\dfrac1n\varphi+2\pi k)\right)+i\sin\left(\dfrac1n(\varphi+2\pi k)\right)\right)=|z|^\frac1n\left(\cos\dfrac{\varphi+2\pi k}{n}\right.+\\+\left.i\sin\dfrac{\varphi+2\pi k}{n}\right)=\sqrt[n]{z}$
\end{itemize}
Загальна показникова функція: 
$\\\\\deff f(z)=\alpha^z=\exp(z\Ln \alpha),\tab a\in C\\\\$
$\alpha^z=\exp(z\Ln \alpha)=\langle z=x+iy,\tab x,y\in\Rn,\tab\alpha=|z|e^{i\psi}\rangle=\exp((x+iy)(\ln|\alpha|+\\+i(\psi+2\pi k)))=\exp(x\ln|\alpha|-u(\psi+2\pi k)+i(y\ln|\alpha|+x(\psi+2\pi k)))=e^{x\ln|\alpha|-y(\psi+2\pi k)}\cdot\\\cdot(\cos(y\ln|\alpha|+x(\psi+2\pi k))+i\sin(y\ln|\alpha|+x(\psi+2\pi k)))\\\\$
При $\alpha\in\Rn:\tab (|e|=e,\psi=0)$
\begin{itemize}
	\item[-]  $e^z=e^{x\ln e-y\cdot2\pi k}\cdot(\cos(y\ln e+x\cdot2\pi k)+i\sin(y\ln e+x\cdot2\pi k))$
	\item[-] При $k=0:\tab e^z|_{k=0}=e^{x}\cdot(\cos y+i\sin y)=\exp z$
\end{itemize}
\section{Границя функції. Неперервність.}
Нехай $z=x+iy,\tab z_0=x_0+iy_0,\tab A\in\Cn\\\\\deff\lim\limits_{z\to z_0}f(z)=A,$ якщо $\forall\varepsilon>0\>\>\>\>\exists\delta(\varepsilon)>0\>\>\>\>(0<|z-z_0|<\delta(\varepsilon))\Rightarrow|f(z)-A|<\varepsilon)$
\begin{figure*}[htp]\centering
	\begin{tikzpicture}
		\begin{axis}[ticks=none,axis lines=middle, xmin=-0.1, xmax=0.9,ymin=-0.1,ymax=0.9,xshift=-3.4cm]	
		\end{axis}
		\draw (0,0) node {\small$z=x+iy$};
		\draw (-2.5,5.5) node {$y$};
		\draw (3,1) node {$x$};
		\draw (0,3) node[circle,fill,inner sep=1pt] {};
		\draw (0,3) node[below] {\scriptsize$z_0$};
		\draw (0,3) -- (1.06,4.06) node[sloped,midway,above] {\scriptsize$z$};
		\draw (1.06,4.06) node[right] {\scriptsize$\delta(\varepsilon)$};
		\draw[->] (3,3.5) -- (4,3.5) node[midway,above] {$f$};
		\draw (0,3) circle (1.5);
 	    \clip (0,3) circle (1.5);
 		\pgfmathsetseed{24122015}
    	\foreach \p in {1,...,300}
    	{ \pgfmathsetmacro{\x}{2*rand}
        	\pgfmathsetmacro{\y}{rand*sqrt(4-pow(\x,2))+3}
        	\fill[black]    (\x,\y) circle (0.02);
    	}
	\end{tikzpicture}
	\begin{tikzpicture}
		\begin{axis}[ticks=none,axis lines=middle, xmin=-0.1, xmax=0.9,ymin=-0.1,ymax=0.9,xshift=-3.4cm]	
		\end{axis}
		\draw (0,0) node {\small$f(z)=u+iv$};
		\draw (-2.5,5.5) node {$v$};
		\draw (3,1) node {$u$};
		\draw (0,3) node[circle,fill,inner sep=1pt] {};
		\draw (0,3) node[above] {\scriptsize$A$};
		\draw (0,3) -- (1.06,1.95) node[sloped,midway,above] {\scriptsize$\varepsilon$};
		\draw (1.06,1.95) node[right] {\scriptsize$f(z)$};
		\draw (0,3) circle (1.5);
 	    \clip (0,3) circle (1.5);
 		\pgfmathsetseed{24122015}
    	\foreach \p in {1,...,300}
    	{ \pgfmathsetmacro{\x}{2*rand}
        	\pgfmathsetmacro{\y}{rand*sqrt(4-pow(\x,2))+3}
        	\fill[black]    (\x,\y) circle (0.02);
    	}
	\end{tikzpicture}
\end{figure*}
$\\\lim\limits_{z\to z_0}f(z)=\left\langle\begin{array}{c}
	f(z)=u(x,y)+iv(x,y)\\z=x+iy
\end{array} \right\rangle=\lim\limits_{\scriptsize\begin{array}{c}
	x\to x_0\\y\to y_0
\end{array}}(u+iv).\\ f\to A,g\to B$ при $z\to z_0.\tab $ Тоді 
\begin{align*}
	\tab\lim\limits_{z\to z_0}(f+g)=&\lim\limits_{z\to z_0}f+\lim\limits_{z\to z_0}g\\\lim\limits_{z\to z_0}(f\cdot g)=&\lim\limits_{z\to z_0}f\cdot\lim\limits_{z\to z_0}g\\\lim\limits_{z\to z_0}\left(\dfrac fg\right)=&\dfrac{\lim\limits_{z\to z_0}f}{\lim\limits_{z\to z_0}g},\tab \lim\limits_{z\to z_0}g\neq0
\end{align*}
$\\\deff\lim\limits_{z\to z_0} f(z)=\infty,$ якщо $\forall\varepsilon>0\tab\exists\delta(\varepsilon)>0\tab(0<|z-z_0|<\delta(\varepsilon)\Rightarrow|f(z)|>\varepsilon)\\$
\begin{figure*}[htp]\centering
	\begin{tikzpicture}
		\begin{axis}[ticks=none,axis lines=middle, xmin=-0.1, xmax=0.9,ymin=-0.1,ymax=0.9,xshift=-3.4cm]	
		\end{axis}
		\draw (-2.5,5.5) node {$y$};
		\draw (3,1) node {$x$};
		\draw (0,3) node[circle,fill,inner sep=1pt] {};
		\draw (0,3) node[below] {\scriptsize$z_0$};
		\draw (0,3) -- (1.06,4.06) node[sloped,midway,above] {\scriptsize$z$};
		\draw[->] (3,3.5) -- (4,3.5) node[midway,above] {$f$};
		\draw (0,3) circle (1.5);
 	    \clip (0,3) circle (1.5);
 		\pgfmathsetseed{24122015}
    	\foreach \p in {1,...,300}
    	{ \pgfmathsetmacro{\x}{2*rand}
        	\pgfmathsetmacro{\y}{rand*sqrt(4-pow(\x,2))+3}
        	\fill[black]    (\x,\y) circle (0.02);
    	}
	\end{tikzpicture}
	\begin{tikzpicture}
		\pgfmathsetseed{24122015}
    	\foreach \p in {1,...,300}
    	{ \pgfmathsetmacro{\x}{3*rand}
        	\pgfmathsetmacro{\y}{rand*sqrt(9-pow(\x,2))+2.7}
        	\fill[black]    (\x,\y) circle (0.02);
    	}
    	\draw[fill=white] (0.03,2.85) circle (1.5);
		\begin{axis}[ticks=none,axis lines=middle, xmin=-0.5, xmax=0.5,ymin=-0.5,ymax=0.5,xshift=-3.4cm]	
		\end{axis}
		\draw (1,4) node[circle,fill,inner sep=1pt] {};
		\draw (0.03,2.85) -- (1,4) node[right,above] {$\varepsilon$};
		\draw (2.3,3.5) node {$f(z)$};
	\end{tikzpicture}
\end{figure*}


























\end{justify}
\end{document}