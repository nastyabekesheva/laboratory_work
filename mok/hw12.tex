\documentclass[a4paper,12pt]{article}

\usepackage[ukrainian,english]{babel}
\usepackage{ucs}
\usepackage[utf8]{inputenc}
\usepackage[T2A]{fontenc}

\usepackage{amsmath}
\usepackage{amsfonts}
\usepackage[document]{ragged2e}
\usepackage{graphicx}
\usepackage{wrapfig}
\usepackage{enumitem}
\usepackage{changepage}
\usepackage{cases}
\usepackage{multirow}
\usepackage{tikz}
 
\newcommand{\ord}[0]{\textrm{\hspace*{0.15cm}ord\hspace*{0.1cm}}}

\usepackage[left=20mm, top=20mm, right=20mm, bottom=20mm, nohead, nofoot]{geometry}

\newcommand\tab[1][0.5cm]{\hspace*{#1}}
\begin{document}
	\begin{justify}
		\thispagestyle{empty}\setlength{\parindent}{0pt}
  		\topskip0pt
 		\vspace*{\fill}
  		\begin{center}
  			\noindent\makebox[\linewidth]{\rule{\paperwidth}{0.4pt}}
   			\LARGE{\bigbreak ДОМАШНЯ РОБОТА №12\\З ПРЕДМЕТУ\\''МАТЕМАТИЧНІ ОСНОВИ КРИПТОЛОГІЇ''\\\bigbreak} 
   			ФІ-12 Бекешева Анастасія 
   			\noindent\makebox[\linewidth]{\rule{\paperwidth}{0.4pt}}
  		\end{center}
 		\vspace*{\fill}\newpage
 	\begin{enumerate}
 		\item \begin{table}[htp]\centering
\begin{tabular}{|c|c|c|c|c|}
\hline
vec   & in basis    & ind        & ord & min polynomial \\ \hline
(0,0) & 0           &            &     & $x$            \\ \hline
(0,1) & 1           & $\alpha^8$ & 1   & $x+2$          \\ \hline
(1,0) & $\alpha$    & $\alpha^1$ & 8   & $x^2+x+2$      \\ \hline
(1,1) & $\alpha+1$  & $\alpha^7$ & 8   & $x^2+2x+2$     \\ \hline
(0,2) & 2           & $\alpha^4$ & 2   & $x+1$          \\ \hline
(2,0) & $2\alpha$   & $\alpha^5$ & 8   & $x^2+x+1$      \\ \hline
(1,2) & $a+2$       & $\alpha^6$ & 4   & $x^2+1$        \\ \hline
(2,1) & $2\alpha+1$ & $\alpha^2$ & 4   & $x^2+1$        \\ \hline
(2,2) & $2\alpha+2$ & $\alpha^3$ & 8   & $x^2+x+2$      \\ \hline
\end{tabular}
\end{table}
$x^9-x=x^9+2x=x^{3^2}+2x=x(x+2)(x^2+1)(x^2+x+2)(x^2+2x+2)(x^3+x+1)$
 		\item $F_{p^n}$ - розширення полінома $g(x)=x^{p^n}-x$. Будь-який нормований незвідний поліном степеня $n$ ділить $g$. Отже $[F_{p^n}:F_{p}]$ і більше розширень немає. Отже кожен незвідний поліном, що ділить $g$ має бути степеня $n$ або 1. Так як кожен лінійний поліном над $F_p$ ділить $g$ і $g$ має корені, ми маємо $p$ різних поліномів що ділять $g$. Якщо перемножити усі незвідні нормовані поліноми що ділять $g$ ми отримаємо $g$ і сума їх степенів буде дорівнювати $p^n$. Позначимо кількість незвідних нормованих поліномів степеня $n$ як $m$ і отримаємо $mn+p=p^n$. Отже $m=\dfrac{p^n-p}{n}$. Існує $m=\dfrac{3^7-3}{7}=312$ незвідних нормованих поліномів степіня 7. 
 		\item Переберемо можливі значення у $F_4(\{0,1,\beta,\beta+1\}):$
 			\begin{itemize}
 				\item [0:] $0^2+0\cdot\beta+1=1\neq0$
 				\item [1:] $1^2+1\cdot\beta+1=\beta\neq0$
 				\item [$\beta$:] $\beta^2+\beta^2+1=1\neq0$
 				\item [$\beta+1:$] $\beta^2+2\beta+1+\beta^2+\beta+1=\beta\neq0$
 			\end{itemize} Отже поліном $x^2+\beta x+1$ мав би корені, якби $\beta=0$, але воно таким не є, адже це корінь $x^2+x+1$ над $F_2$. Поліном $x^2+\beta x+1$ незвідний над $F_4$.
 	\end{enumerate}
 	\end{justify}
\end{document}