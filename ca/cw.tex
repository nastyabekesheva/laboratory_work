\documentclass[a4paper,12pt]{article}

\usepackage[ukrainian,english]{babel}
\usepackage{ucs}
\usepackage[utf8]{inputenc}
\usepackage[T2A]{fontenc}

\usepackage{amsmath}
\usepackage{amsfonts}
\usepackage{amssymb}
\usepackage[document]{ragged2e}
\usepackage{graphicx}
\usepackage{wrapfig}
\usepackage{enumitem}
\usepackage{subfigure}

\usepackage{color,soul}
\setul{0.5ex}{0.3ex}

\usepackage[left=20mm, top=20mm, right=20mm, bottom=20mm, nohead, nofoot]{geometry}

\usepackage{tikz,pgfplots}
\usetikzlibrary{arrows.meta} 
\usetikzlibrary{shapes}

\makeatletter
\newcommand{\skipitems}[1]{%
  \addtocounter{\@enumctr}{#1}%
}
\makeatother

\newcommand\tab[1][0.5cm]{\hspace*{#1}}
\newcommand\dx[1]{\hspace*{0.2cm}\textrm{d}{#1}\hspace*{0.1cm}}
\newcommand\Res[1]{\hspace*{0.2cm}\underset{#1}{\textrm{Res}}\hspace*{0.1cm}}
\newcommand\Arcsin[0]{\hspace*{0.2cm}\textrm{Arcsin}\hspace*{0.1cm}}
\newcommand\Ln[0]{\hspace*{0.2cm}\textrm{Ln}\hspace*{0.1cm}}
\newcommand\dint[0]{\displaystyle\int}
\newcommand\doint[0]{\displaystyle\oint}
\newcommand{\Int}[0]{\mathfrak{I}}

\newcommand*\circled[1]{\tikz[baseline=(char.base)]{
            \node[shape=circle,draw,inner sep=2pt] (char) {#1};}}
 

\renewcommand{\Im}[0]{\mathfrak{Im}}
\renewcommand{\Re}[0]{\mathfrak{Re}}

\tikzset{
  vlines/.style={
    path picture={
      \draw[xstep=#1, ystep=100cm, shift={(path picture bounding box.south west)} ]
      (path picture bounding box.south west) grid (path picture bounding box.north east);
    }
  }
}

\begin{document}
	\begin{justify}
		\thispagestyle{empty}\setlength{\parindent}{0pt}
  		\topskip0pt
 		\vspace*{\fill}
  		\begin{center}
  			\noindent\makebox[\linewidth]{\rule{\paperwidth}{0.4pt}}
   			\LARGE{\bigbreak РОЗРАХУНКОВА РОБОТА \\З ПРЕДМЕТУ\\''ТЕОРІЯ ФУНКЦІЇ КОМПЛЕКСНОЇ ЗМІННОЇ''\\ВАРІАНТ №2\bigbreak} 
   			ФІ-12 Бекешева Анастасія 
   			\noindent\makebox[\linewidth]{\rule{\paperwidth}{0.4pt}}
  		\end{center}
 		\vspace*{\fill}\newpage
 		\begin{enumerate}
 			\item $$\sqrt[4]{\dfrac{-1+i\sqrt3}{2}}$$ $\sqrt[4]{\dfrac{-1+i\sqrt3}{2}}=\sqrt[4]{\left(-\dfrac12\right)^2+\left(-\dfrac{\sqrt3}2\right)^2}\cdot\left(\cos\left(\dfrac{\arctan(-\sqrt3)+\pi+2\pi k}{4}\right)\right.+\\+\left.i\sin\left(\dfrac{\arctan(-\sqrt3)+\pi+2\pi k}{4}\right)\right)=1\cdot\left(\cos\left(\dfrac\pi6+\dfrac12\pi k\right)+i\sin\left(\dfrac\pi6+\dfrac12\pi k\right)\right)$ $$0\leq k\leq3$$
 				\begin{flalign*}
 					\tab k=0:&\tab\cos\dfrac\pi6+i\sin\dfrac\pi6=\dfrac{\sqrt3}2+\dfrac12i&\\
 					\tab k=1:&\tab\cos\dfrac{2\pi}3+i\sin\dfrac{2\pi}3=-\dfrac12+\dfrac{\sqrt3}2i\\
 					\tab k=2:&\tab\cos\dfrac{7\pi}6+i\sin\dfrac{7\pi}6=-\dfrac{\sqrt3}2-\dfrac12i\\
 					\tab k=3:&\tab\cos\dfrac{5\pi}3+i\sin\dfrac{5\pi}3=\dfrac12-\dfrac{\sqrt3}2i
 				\end{flalign*}
 			\item $$\cos\left(\dfrac\pi6+2i\right)$$ $\cos\left(\dfrac\pi6+2i\right)=\cos\dfrac\pi6\cos2i-\sin\dfrac\pi6\sin2i=\dfrac{\sqrt3}2\cos2i-\dfrac12\sin2i=\dfrac{\sqrt3}2\ch2-\dfrac12i\sh2$
 			\item $$\Arcsin4$$ $\Arcsin4=-i\Ln\left(4i\pm\sqrt{1-16}\right)=-i\Ln\left(i\left(4\pm\sqrt{15}\right)\right)=\left\langle \begin{array}{c}
 				z=4\pm\sqrt{15}\>(x=0)\\ \arg z=\dfrac\pi2
 			\end{array} \right\rangle=\\=-i\left(\ln\left|4\pm\sqrt{15}\right|+i\left(\dfrac\pi2+2\pi k\right)\right)=-i\ln\left(4\pm\sqrt{15}\right)+\dfrac\pi2+2\pi k,\tab k\in\mathbb{Z}$
 			\item $$|z+i|\geq1,\tab |z|<2$$
 				\begin{figure*}[htp]\centering
 					\begin{tikzpicture}
 						%\path[fill=lightgray, opacity =0.2] (6.85,0) rectangle (0,5.7);
 						\draw[vlines=2mm] (3.43,2.85) circle (2);
 						\draw[white,thick] (3.43,2.85) circle (2);
						\draw[thick, fill=white] (3.43,1.85) circle (1);
						\draw[fill=red, opacity =0.2] (3.43,2.85) circle (2);
						\draw[dashed, thick,] (3.43,2.85) circle (2);
						\draw (4,4) -- (5,5.1) node[right] {\scriptsize$|z+i|\geq1,$};
						\draw (5,4.7) node[right] {\scriptsize$|z|<2$};
						\begin{axis}[ticks=none,axis lines=middle, xmin=-0.5, xmax=0.5,ymin=-0.5,ymax=0.5]	
						\end{axis}
						\draw (4.43,3) -- (4.43,2.7) node[below] {\tiny1};
						\draw (5.43,3) -- (5.43,2.7) node[below] {\tiny2};
 					\end{tikzpicture}
 				\end{figure*}\newpage
 			\item $$z=2\sec t-3i\tg t$$ $\left\{\begin{array}{l}
 				\dfrac x2=\dfrac2{\cos t},\\\dfrac y3=-\dfrac{\sin t}{\cos t}
 			\end{array} \right.\Rightarrow\dfrac{x^2}4-\dfrac{y^2}9=-\dfrac{\sin^2 t}{\cos^2 t}+\dfrac1{\cos^2 t}=\dfrac{1-\sin^2t}{\cos^2t}=1$
 				\begin{figure*}[htp]\centering
					\begin{tikzpicture}
						\begin{axis}[ticks=none,axis lines=middle, xmin=-10, xmax=10,ymin=-10,ymax=10]
						\addplot [thick] ({2*cosh(x)}, {3*sinh(x)});
        				\addplot [thick] ({-2*cosh(x)}, {3*sinh(x)});
        				\addplot[dashed] expression {1.5*x};
       					\addplot[dashed] expression {-1.5*x};
						\end{axis}
						\draw (4.1,3) -- (4.1,2.7) node[below] {\tiny2};
						\draw (2.75,3) -- (2.75,2.7) node[below] {\tiny-2};
						\draw (3.3,3.7) -- (3.56,3.7) node[right] {\tiny3};
						\draw (3.3,2) -- (3.56,2) node[right] {\tiny-3};
					\end{tikzpicture}
				\end{figure*}
			\item $$u=x^3-3xy^2+1,\tab f(0)=1$$ $u'_x=3(x^2-y^2),\tab u'_y=-6xy,\tab u'_x=v'_y,v'_x=-u'_y\Longrightarrow v=3\dint(x^2-y^2)\dx{y}=\\=3\left(x^2y-\dfrac13 y^3\right),\tab v'_x=6xy+C'(x),\tab v'_x=-u'_y\Longrightarrow 6xy+C'(x)=6xy,\tab C'(x)=0,\\C(x)=C\in\mathbb{R},\tab v=3x^2y-y^3+C\\f(z)=x^3-3xy^2+1+3x^2yi-y^3i+Ci,\tab f(0)=1,\tab f(0)=0-0+1+0i-0i+Ci=\\=Ci+1=1\Longrightarrow C=0,\tab f(z)x^3-3xy^2+1+3x^2yi-y^3i=(x+iy)^3+1=z^3+1$
			\item $$\dint\limits_\mathcal{L}(z+1)e^z\dx{z},\tab \mathcal{L}:\{|z|=1,\tab\Re z\geq0\}$$ $z=x+iy,\tab (z+1)e^z=(x+iy+1)e^{x+iy}=(z+iy+1)\cdot(\cos(x+iy)+i\sin(x+iy))$
			\item $$f(z)=\dfrac{z-4}{z^4+z^3-2z^2}$$ $\left[\begin{array}{l}
				z_1=0\\z_2=1\\z_3=-2
			\end{array} \right.\left[\begin{array}{l}
				\mathcal{D}_1:\>0<|z|<1\\\mathcal{D}_2:\>1<|z|<2\\\mathcal{D}_3:\> |z|>2
			\end{array} \right. f(z)=\dfrac{z-4}{z^2(z-1)(z+2)}=\dfrac1{z^2}\left(\dfrac A{z-1}+\dfrac B{z+2}\right)=\\=\left\langle A=\dfrac{z-4}{z+2}\bigg|_{z=1}=-\dfrac33=-1,\tab B=\dfrac{z-4}{z-1}\bigg|_{z=-2}=\dfrac{-6}{-3}=2 \right\rangle=\dfrac1{z^2}\left(\dfrac 2{z+2}-\dfrac 1{z-1}\right)\\\dfrac{1}{z-1}=-\dfrac1{1-z}=-\sum\limits_{n=0}^\infty z^n\in \mathcal{D}_1\\\dfrac{1}{z-1}=\dfrac1z\cdot\dfrac{1}{1-\frac1z}=\dfrac1z\sum\limits_{n=0}^\infty \left(\dfrac1z\right)^n=\sum\limits_{n=0}^\infty \dfrac1{z^{n+1}}\in \mathcal{D}_2,\mathcal{D}_3\\\dfrac2{z+2}=\dfrac{1}{1-\left(-\frac z2\right)}=\sum\limits_{n=0}^\infty\left(-\dfrac z2\right)^n=\sum\limits_{n=0}^\infty(-1)^n\dfrac{z^n}{2^n}\in \mathcal{D}_1,\mathcal{D}_2\\\dfrac2{z+2}=\dfrac2z\cdot\dfrac1{1-\left(-\frac 2z\right)}=\dfrac2z\sum\limits_{n=0}^\infty\left(-\dfrac2z\right)^n=\sum\limits_{n=0}^\infty(-1)^n\dfrac{2^{n+1}}{z^{n+1}}\in\mathcal{D}_3\\\\\mathcal{D}_1:\tab f(z)=\dfrac1{z^2}\left(\sum\limits_{n=0}^\infty(-1)^n\dfrac{z^n}{2^n} -\left(-\sum\limits_{n=0}^\infty z^n\right)\right)=\sum\limits_{n=0}^\infty z^{n-2}\left(\dfrac{(-1)^n}{2^n}+1\right)\\\mathcal{D}_2:\tab f(z)=\dfrac1{z^2}\left(\sum\limits_{n=0}^\infty(-1)^n\dfrac{z^n}{2^n}-\sum\limits_{n=0}^\infty \dfrac1{z^{n+1}}\right)=\sum\limits_{n=0}^\infty(-1)^n\dfrac{z^{n-2}}{2^n}-\sum\limits_{n=0}^\infty \dfrac1{z^{n+3}}\\\mathcal{D}_3:\tab f(z)=\dfrac1{z^2}\left(\sum\limits_{n=0}^\infty(-1)^n\dfrac{2^{n+1}}{z^{n+1}}-\sum\limits_{n=0}^\infty \dfrac1{z^{n+1}}\right)=\sum\limits_{n=0}^\infty\dfrac{\left((-1)^n2^{n+1}-1\right)}{z^{n+3}}$
			\begin{figure*}[htp]\centering
 					\begin{tikzpicture}
 						\begin{axis}[ticks=none,axis lines=middle, xmin=-0.5, xmax=0.5,ymin=-0.5,ymax=0.5 ]	
						\end{axis}
						\draw (1.43,2.85) node[circle,fill,inner sep=1pt,color=red] {};
						\draw (1.43,2.85) node[right] {$-2$} ;
						\draw (4.43,2.85) node[circle,fill,inner sep=1pt,color=red] {};
						\draw (4.43,2.85) node[right] {$1$} ;
						\draw (3.43,2.85) circle (2);
						\draw (3.43,2.85) circle (1);
						\draw(4,3.3) node {$\mathcal{D}_1$};
						\draw(4.4,4) node {$\mathcal{D}_2$};
						\draw(4.9,4.8) node {$\mathcal{D}_3$};
 					\end{tikzpicture}
 				\end{figure*}
 			\skipitems{1}
			\item $$f(z)=\sin\left(\dfrac{z}{z-1}\right),\tab z_0=1$$ $f(z)=\sin\left(\dfrac{z+1-1}{z-1}\right)=\sin\left(1+\dfrac1{z-1}\right)=\sin1\cos\left(\dfrac1{z-1}\right)+\cos1\sin\left(\dfrac1{z-1}\right)=\\=\sin(1)\cdot \sum\limits_{n=0}^\infty\dfrac{(-1)^n}{(2n)!}\left(\dfrac1{z-1}\right)^{2n}+\cos(1)\cdot\sum\limits_{n=0}^\infty\dfrac{(-1)^n}{(2n+1)!}\left(\dfrac1{z-1}\right)^{2n+1}=\\=\sin(1)\cdot \sum\limits_{n=0}^\infty\dfrac{(-1)^n}{(2n)!(z-1)^{2n}}+\cos(1)\cdot\sum\limits_{n=0}^\infty\dfrac{(-1)^n}{(2n+1)!(z-1)^{2n+1}}$
			\item $$f(z)=z^3e^{\dfrac7{z^2}}$$ $f(z)=z^3\left(1+\dfrac7{z^2}+\dfrac{7^2}{2!z^4}+\dfrac{7^3}{3!z^6}+\dots\right)=z^3+7z+\underbrace{\dfrac{7^2}{2!z}+\dfrac{7^3}{3!z^3}+\dots}_{\textrm{головна частина}},\tab z_0=0\\$ енскінчена кількість доданків в головній частині $\Longrightarrow$ $z_0$ - істотно особлива.
			\item $$f(z)=\dfrac1{\cos z}$$ $\cos z=0\Longrightarrow z_k=\dfrac\pi2+\pi k,k\in\mathbb{Z},\tab (\cos z)'\big|_{z_k}=-\sin(z_k)=\pm1\neq0\\$ Нулі знаменика не є нулями чисельника. $m_1=0,\tab m_2=1\Longrightarrow z_k$ - полюс I порядку.
			\item $$\mathcal{I}=\doint\limits_\mathcal{L}\dfrac{2\dx{z}}{z^2(z-1)},\tab\mathcal{L}:|z-1-i|=\dfrac54$$ $z_1=0\not\in\mathcal{L},\tab z_2=1:\tab m_1=0,\tab m_2=1$ - полюс I порядку.\\ $\mathcal{I}=2\pi i\Res{z=1}\dfrac{2\dx{z}}{z^2(z-1)}=2\pi i\cdot\dfrac{2}{3z^2-2z}\bigg|_{z=1}=2\pi i\cdot\dfrac2{3-2}=4\pi i$
				\begin{figure*}[htp]\centering
					\begin{tikzpicture}
						\begin{axis}[ticks=none,axis lines=middle, xmin=-0.5, xmax=0.5,ymin=-0.5,ymax=0.5 ]	
						\end{axis}
						\draw (4.43,3.85) circle (1.25);
						\draw (3.43,2.85) node[circle,fill,inner sep=1pt,color=red] {};
						\draw (3.43,2.85) node[left] {$0$} ;
						\draw (4.43,2.85) node[circle,fill,inner sep=1pt,color=red] {};
						\draw (4.43,2.85) node[right] {$1$} ;
					\end{tikzpicture}		
				\end{figure*}
			\item $$\mathcal{I}=\doint\limits_\mathcal{L}\dfrac{2-z^2+3z^3}{4z^3}\dx{z},\tab\mathcal{L}:|z|=\dfrac12$$ $z_0=0:\tab m_1=0,\tab m_2=3$ - полюс III порядку.\\ $\mathcal{I}=2\pi i\Res{z=0}\dfrac{2-z^2+3z^3}{4z^3}=\pi i\lim\limits_{z\to0}\dfrac{\dx{}^2}{\dx{z}^2}\left(\dfrac{2-z^2+3z^3}{4z^3}\cdot(z-0)^3\right)=\\=\pi i\lim\limits_{z\to0}\dfrac{\dx{}}{\dx{z}}\left(\dfrac{\dx{}}{\dx{z}}\left(\dfrac2{4}-\dfrac{z^2}{4}+\dfrac{3z^3}{4}\right)\right)=\pi i\lim\limits_{z\to0}\dfrac{\dx{}}{\dx{z}}\left(-\dfrac12 z+\dfrac94 z^2\right)=\pi i\lim\limits_{z\to0}\left(-\dfrac12+\dfrac92z\right)=\\=\pi i\left(-\dfrac12+0\right)=-\dfrac{\pi}2i$
				\begin{figure*}[htp]\centering
					\begin{tikzpicture}
						\begin{axis}[ticks=none,axis lines=middle, xmin=-0.5, xmax=0.5,ymin=-0.5,ymax=0.5 ]	
						\end{axis}
						\draw (3.43,2.85) circle (1);
						\draw (3.43,2.85) node[circle,fill,inner sep=1pt,color=red] {};
						\draw (3.43,2.85) node[left] {$0$} ;
					\end{tikzpicture}		
				\end{figure*}
			\item $$\mathcal{I}=\doint\limits_{\mathcal{L}}\dfrac{\cos3z-1+\frac92z^2}{z^4\sh\frac94z}\dx{z},\tab \mathcal{L}:|z|=1$$ $z^4\sh\dfrac94z=0,\tab z_1=0,\tab \sin\dfrac94iz=0,\tab z_k=-\dfrac49i\pi k,\tab k\in\mathbb{Z},\tab \cos3z_k-1+\dfrac92z_k^2\neq0\\z_1=0$ - нуль V порядку, $z_k=-\dfrac49i\pi k\setminus\{0\}$ - не нуль. \\$f(z)=\dfrac{\cos3z-1+\frac92z^2}{z^4\sh\frac94z}=\dfrac{\left(1-\dfrac{3^2z^2}{2!}+\dfrac{3^4z^4}{4!}+\dfrac{3^6z^6}{6!}+\dots\right)-1+\dfrac{9z^2}{2}}{z^4\left(\dfrac{9z}{4}+\dfrac{9^3z^3}{4^33!}+\dfrac{9^5z^5}{4^55!}+\dots\right)}=\\=\dfrac{z^4\left(\dfrac{3^4}{4!}+\dfrac{3^6z^2}{6!}+\dots\right)}{z^5\left(\dfrac{9}{4}+\dfrac{9^3z^2}{4^33!}+\dfrac{9^5z^4}{4^55!}+\dots\right)}=\dfrac{z^4g_1(z)}{z^5g_2(z)},\tab g_i(0)\neq0\\z_0=0:\tab m_1=4,\tab m_2=5\Longrightarrow z_0=0$ - полюс I порядку.\\$\mathcal{I}=2\pi i\Res{z=0}f(z)=2\pi i\lim\limits_{z\to0}\dfrac{z^4g_1(z)}{z^5g_2(z)}\cdot z=2\pi i\lim\limits_{z\to0}\dfrac{\dfrac{3^4}{4!}+\dfrac{3^6z^2}{6!}+\dots}{\dfrac{9}{4}+\dfrac{9^3z^2}{4^33!}+\dfrac{9^5z^4}{4^55!}+\dots}=2\pi i\dfrac{\dfrac{3^4}{4!}}{\dfrac{9}{4}}=3\pi i$
				\begin{figure*}[htp]\centering
					\begin{tikzpicture}
						\begin{axis}[ticks=none,axis lines=middle, xmin=-0.5, xmax=0.5,ymin=-0.5,ymax=0.5 ]	
						\end{axis}
						\draw (3.43,2.85) circle (1);
						\draw (3.43,2.85) node[circle,fill,inner sep=1pt,color=red] {};
						\draw (3.43,4.85) node[circle,fill,inner sep=1pt,color=red] {};
						\draw (3.43,.85) node[circle,fill,inner sep=1pt,color=red] {};
						\draw (3.43,2.85) node[left] {$0$} ;
					\end{tikzpicture}		
				\end{figure*}
			\item $$$$	
 		\end{enumerate}
 	\end{justify}
\end{document}