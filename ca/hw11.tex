\documentclass[a4paper,12pt]{article}

\usepackage[ukrainian,english]{babel}
\usepackage{ucs}
\usepackage[utf8]{inputenc}
\usepackage[T2A]{fontenc}

\usepackage{amsmath}
\usepackage{amsfonts}
\usepackage{amssymb}
\usepackage[document]{ragged2e}
\usepackage{graphicx}
\usepackage{wrapfig}
\usepackage{enumitem}
\usepackage{subfigure}

\usepackage{color,soul}
\setul{0.5ex}{0.3ex}

\usepackage[left=20mm, top=20mm, right=20mm, bottom=20mm, nohead, nofoot]{geometry}

\usepackage{tikz,pgfplots}
\usetikzlibrary{arrows.meta} 
\usetikzlibrary{shapes}

\makeatletter
\newcommand{\skipitems}[1]{%
  \addtocounter{\@enumctr}{#1}%
}
\makeatother

\newcommand\tab[1][0.5cm]{\hspace*{#1}}
\newcommand\dx[1]{\hspace*{0.2cm}\textrm{d}{#1}\hspace*{0.1cm}}
\newcommand\Res[0]{\hspace*{0.2cm}\textrm{Res}\hspace*{0.1cm}}
\newcommand\dint[0]{\displaystyle\int}
\newcommand\doint[0]{\displaystyle\oint}
\newcommand{\Int}[0]{\mathfrak{I}}

\newcommand*\circled[1]{\tikz[baseline=(char.base)]{
            \node[shape=circle,draw,inner sep=2pt] (char) {#1};}}
 

\renewcommand{\Im}[0]{\mathfrak{Im}}
\renewcommand{\Re}[0]{\mathfrak{Re}}

\begin{document}
	\begin{justify}
		\thispagestyle{empty}\setlength{\parindent}{0pt}
  		\topskip0pt
 		\vspace*{\fill}
  		\begin{center}
  			\noindent\makebox[\linewidth]{\rule{\paperwidth}{0.4pt}}
   			\LARGE{\bigbreak ДОМАШНЯ РОБОТА №11\\З ПРЕДМЕТУ\\''ТЕОРІЯ ФУНКЦІЇ КОМПЛЕКСНОЇ ЗМІННОЇ''\\\bigbreak} 
   			ФІ-12 Бекешева Анастасія 
   			\noindent\makebox[\linewidth]{\rule{\paperwidth}{0.4pt}}
  		\end{center}
 		\vspace*{\fill}\newpage
 		\begin{enumerate}
 			\item $\doint\limits_{|z-1|=2}\dfrac{z^2+1}{(z-2i)(z+2i)\sin\frac z3}\dx{z}\tab \circled{=}\\z=\pm2i,\tab\sin\dfrac z3=0,\tab z=3\pi n,\tab m_1=0,\tab m_2=1,\tab z=0$ - полюс I порядку.$\\\tab \circled{=}\tab 2\pi i\Res\dfrac{z^2+1}{(z-2i)(z+2i)\sin\frac z3}=2\pi i\lim\limits_{z\to0}\dfrac{z^2+1}{(z-2i)(z+2i)\sin\frac z3}\cdot z=\\=2\pi i\lim_{z\to 0}\dfrac{(z^2+1)z}{(z-2i)(z+2i)\dfrac z3}=2\pi i\dfrac34=\dfrac32\pi i$
 				\begin{figure*}[htp]\centering
 					\begin{tikzpicture}
 						\begin{axis}[ticks=none,axis lines=middle, xmin=-0.5, xmax=0.5,ymin=-0.5,ymax=0.5 ]	
						\end{axis}
						\draw (3.43,4.85) node[left] {$2i$} ;
						\draw (3.43,2.85) node[circle,fill,inner sep=1pt,color=red] {};
						\draw (3.43,4.85) node[circle,fill,inner sep=1pt,color=red] {};
						\draw (3.43,0.85) node[circle,fill,inner sep=1pt,color=red] {};
						\draw (3.43,0.85) node[left] {$-2i$} ;
						\draw (3.43,2.85) node[right] {$0$} ;
						\draw (4.43,2.85) circle (2);
 					\end{tikzpicture}
 				\end{figure*}
 			\item $\doint\limits_{|z|=1}\dfrac{z^2e^{\frac{1}{z^2}}-1}{z}\dx{z}\tab \circled{=}\\z=0\textrm{ (знам.)},\tab f(z)=\dfrac{z^2e^{\frac{1}{z^2}}-1}{z}=\dfrac1z\left(z^2\left(1+\dfrac1{z^2}+\dfrac1{2z^4}+\dots\right)-1\right)=\dfrac1z\left(z^2+\dfrac1{2z^2}+\dots\right)=\\=z+\underbrace{\dfrac1{2z^3}+\dots}_{\textrm{гол. час.}},\tab z=0$ - істотно особлива точка. $\\\tab \circled{=}\tab2\pi i \Res f(z)=2\pi i\cdot 0=0$
 				\begin{figure*}[htp]\centering
 					\begin{tikzpicture}
 						\begin{axis}[ticks=none,axis lines=middle, xmin=-0.5, xmax=0.5,ymin=-0.5,ymax=0.5 ]	
						\end{axis}
						\draw (3.43,2.85) node[circle,fill,inner sep=1pt,color=red] {};
						\draw (3.43,2.85) node[right] {$0$} ;
						\draw (3.43,2.85) circle (2);
 					\end{tikzpicture}
 				\end{figure*}
 			\item $\doint\limits_{|z|=4}\dfrac{\sh iz-\sin iz}{z^3\sh\frac z3}\dx{z}\tab \circled{=}\\z=0,\tab \sh\dfrac z3=0,\tab z=3\pi in,\tab\sh iz-\sin iz=iz-\dfrac{iz^3}{3!}+\dfrac{iz^5}{5!}+\dots -iz-\dfrac{iz^3}{3!}-\dfrac{iz^5}{5!}=-\dfrac{2iz^3}{3!}=z^3\left(-\dfrac{2i}{3!}+\dots\right),\tab z^3\sh\dfrac z3=z^3\left(\dfrac{z}3+\dfrac{z^3}{3^3\cdot 3!}\dots\right)=\dfrac{z^4}3\left(1+\dfrac{z^2}{3^3\cdot 3!}\dots\right),\\f(z)=\dfrac{z^3g_1(z)}{\frac{z^4}3g_2(z)},\tab m_1=3,\tab m_2=4,\tab z=0$ - полюс I порядку.\\ $\tab \circled{=}\tab2\pi i\lim\limits_{z\to0}\dfrac{z^3g_1(z)}{\frac{z^4}3g_2(z)}=2\pi i\dfrac{\frac{-2i}{3!}}{\frac13}=2\pi $
 				\begin{figure*}[htp]\centering
 					\begin{tikzpicture}
 						\begin{axis}[ticks=none,axis lines=middle, xmin=-0.5, xmax=0.5,ymin=-0.5,ymax=0.5 ]	
						\end{axis}
						\draw (3.43,2.85) node[circle,fill,inner sep=1pt,color=red] {};
						\draw (3.43,2.85) node[right] {$0$} ;
						\draw (3.43,2.85) circle (2);
 					\end{tikzpicture}
 				\end{figure*}
 			\item $\doint\limits_{|z+2i|=3}\left(\underbrace{\dfrac{\pi}{e^{\dfrac{\pi z}2}+1}}_{\Int_1}+\underbrace{\dfrac{6\ch\dfrac{\pi i z}{2-2i}}{(z-2+2i)^2(z-4+2i))}}_{\Int_2}\right)\dx{z}=\Int_1+6\Int_2$
 				\begin{itemize}
 					\item [($\Int_1$)] $e^{\dfrac{\pi z}2}+1=0,\tab e^{\dfrac{\pi z}2}=-1,\tab \dfrac{\pi z}2=\ln(-1),\tab z_n=2i(2n+1),\\ z=-2i(n=-1),\tab m_1=0,\tab m_2=1, \tab z=-2i$ - полюс I порядку.$\\f(z)=\dfrac{\pi}{e^{\dfrac{\pi z}2}+1},\tab \left(e^{\dfrac{\pi z}2}+1\right)'=\dfrac\pi2e^{\dfrac{\pi z}2}\neq0\\\Res f(z)=\dfrac{\pi}{\dfrac\pi2e^{\dfrac{\pi z}2}}\Bigg|_{z=-2i}=\dfrac{2}{e^{-\pi i}}=-2,\tab\Int_1=-4\pi i$
 					\item [($\Int_2$)] $(z-2+2i)^2(z-4+2i)=0\\\left[\begin{array}{l}
 						z-2+2i=0\\z-4+2i=0
 					\end{array} \right.\left[\begin{array}{l}
 						z=2-2i\textrm{ (нуль знам., не нуль чис.)}\\z=4-2i
 					\end{array} \right.\\f(z)=\dfrac{\ch\dfrac{\pi i z}{2-2i}}{(z-2+2i)^2(z-4+2i))},\tab z=2-2i,\tab m_1=0,\tab m_2=2,\\z=2-2i$ -  - полюс II порядку. $\Res f(z)=\lim\limits_{z\to2-2i}\dfrac{\partial}{\partial{z}}\left(f(z)(z-2+2i)^2\right)=\\=\lim\limits_{z\to2-2i}\dfrac{\dx{}}{\dx{z}}\left(\dfrac{\ch\dfrac{\pi i z}{2-2i}}{(z-4+2i))}\right)=\lim\limits_{z\to2-2i}\dfrac{\sh\dfrac{\pi i z}{2-2i}\cdot\dfrac{\pi i}{2-2i}(z-4+2i)-\ch\dfrac{\pi i z}{2-2i}}{(z-4+2i)^2}=\\=\langle z=2-2i\rangle=\lim\limits_{z\to2-2i}\dfrac{\sh\pi i\cdot \dfrac{\pi i}{2-2i}-\ch\pi i}{4}=\dfrac14$
 				\end{itemize}
 				$\Int=\Int_1+6\Int_2=-4\pi i+3\pi i=-\pi i$
 				\begin{figure*}[htp]\centering
 					\begin{tikzpicture}
 						\begin{axis}[ticks=none,axis lines=middle, xmin=-0.5, xmax=0.5,ymin=-0.5,ymax=0.5 ]	
						\end{axis}
						\draw (3.43,1.85) node[circle,fill,inner sep=1pt,color=red] {};
						\draw (3.43,1.85) node[right] {$-2i$} ;
						\draw (3.43,-0.85) node[right] {$\Int_1$} ;
						\draw (3.43,1.85) circle (2);
 					\end{tikzpicture}
 					\begin{tikzpicture}
 						\begin{axis}[ticks=none,axis lines=middle, xmin=-0.5, xmax=0.5,ymin=-0.5,ymax=0.5 ]	
						\end{axis}
						\draw (4.43,1.85) node[circle,fill,inner sep=1pt,color=red] {};
						\draw (4.43,1.85) node[above] {$2-2i$} ;
						\draw (3.43,-0.85) node[right] {$\Int_2$} ;
						\draw (3.43,1.85) circle (2);
 					\end{tikzpicture}
 				\end{figure*}
 		\end{enumerate}
 	\end{justify}
\end{document}