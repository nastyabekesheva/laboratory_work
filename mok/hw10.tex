\documentclass[a4paper,12pt]{article}

\usepackage[ukrainian,english]{babel}
\usepackage{ucs}
\usepackage[utf8]{inputenc}
\usepackage[T2A]{fontenc}

\usepackage{amsmath}
\usepackage{amsfonts}
\usepackage[document]{ragged2e}
\usepackage{graphicx}
\usepackage{wrapfig}
\usepackage{enumitem}
\usepackage{changepage}
\usepackage{cases}
\usepackage{multirow}
\usepackage{tikz}
 
\newcommand{\charr}[0]{\textrm{\hspace*{0.15cm}char\hspace*{0.1cm}}}

\usepackage[left=20mm, top=20mm, right=20mm, bottom=20mm, nohead, nofoot]{geometry}

\newcommand\tab[1][0.5cm]{\hspace*{#1}}
\begin{document}
	\begin{justify}
		\thispagestyle{empty}\setlength{\parindent}{0pt}
  		\topskip0pt
 		\vspace*{\fill}
  		\begin{center}
  			\noindent\makebox[\linewidth]{\rule{\paperwidth}{0.4pt}}
   			\LARGE{\bigbreak ДОМАШНЯ РОБОТА №10\\З ПРЕДМЕТУ\\''МАТЕМАТИЧНІ ОСНОВИ КРИПТОЛОГІЇ''\\\bigbreak} 
   			ФІ-12 Бекешева Анастасія 
   			\noindent\makebox[\linewidth]{\rule{\paperwidth}{0.4pt}}
  		\end{center}
 		\vspace*{\fill}\newpage
 	\end{justify}
 	\begin{enumerate}
 		\item $(a+n)^n=\sum\limits_{k=0}^{n}C_k^na^{n-k}b^k\newline\forall\alpha\in\mathcal{F}:\tab p\alpha=0\Longrightarrow C_\alpha^{p^n}=\dfrac{p^n!}{(p^n-\alpha)!\alpha!},\tab0<\alpha<p^n.\newline(a+n)^{p^n}=\sum\limits_{k=0}^{p^n}C_k^{p^n}a^{p^n-k}b^k=a^{p^n}+\sum\limits_{k=1}^{p^n-1}C_k^{p^n-1}a^{p^n-1-k}b^k+b^{p^n}=a^{p^n}+b^{p^n}$
 		\item $\newline$\begin{table}[htp]\centering
\begin{tabular}{|c|c|c|c|c|c|c|c|c|c|}
\hline
$\cdot$ & 0 & 1      & 2      & $x$    & $x+1$  & $x+2$  & $2x$   & $2x+1$ & $2x+2$ \\ \hline
0       & 0 & 0      & 0      & 0      & 0      & 0      & 0      & 0      & 0      \\ \hline
1       & 0 & 1      & 2      & $x$    & $x+1$  & $x+2$  & $2x$   & $2x+1$ & $2x+2$ \\ \hline
2       & 0 & 2      & 1      & $2x$   & $2x+2$ & $2x+1$ & $x$    & $x+2$  & $x+1$  \\ \hline
$x$     & 0 & $x$    & $2x$   & $2x+1$ & 1      & $x+1$  & $x+2$  & $2x+2$ & 2      \\ \hline
$x+1$   & 0 & $x+1$  & $2x+2$ & 1      & $x+2$  & $2x$   & 2      & $x$    & $2x+1$ \\ \hline
$x+2$   & 0 & $x+2$  & $2x+1$ & $x+1$  & $2x$   & 2      & $2x+2$ & 1      & $x$    \\ \hline
$2x$    & 0 & $2x$   & $x$    & $x+2$  & 2      & $2x+2$ & $2x+1$ & $x+1$  & 1      \\ \hline
$2x+1$  & 0 & $2x+1$ & $x+2$  & $2x+2$ & $x$    & 1      & $x+1$  & 2      & $2x$   \\ \hline
$2x+2$  & 0 & $2x+2$ & $x+1$  & 2      & $2x+1$ & $x$    & 1      & $2x$   & $x+2$  \\ \hline
\end{tabular}
\end{table}
		\item $\dfrac1{12}\sum\limits_{m\>|\>12}2^{\dfrac{12}{m}}\mu(m)=\dfrac1{12}2^{12}\mu(1)+\dfrac1{12}2^6\mu(2)+\dfrac1{12}2^4\mu(3)+\dfrac1{12}2^2\mu(6)=\dfrac{2^{12}-2^6-2^4+2^2}{12}=$\\$=335$
		\item $\newline$
\begin{center}
\begin{tikzpicture}[scale=0.2]
\tikzstyle{every node}+=[inner sep=0pt]
\draw [black] (36.3,-3.3) circle (3);
\draw (36.3,-3.3) node {$F_{7^{36}}$};
\draw [black] (19.8,-13.6) circle (3);
\draw (19.8,-13.6) node {$F_{7^{18}}$};
\draw [black] (35.9,-25.7) circle (3);
\draw (35.9,-25.7) node {$F_{7^6}$};
\draw [black] (51.9,-13.7) circle (3);
\draw (51.9,-13.7) node {$F_{7^{12}}$};
\draw [black] (65.1,-26.2) circle (3);
\draw (65.1,-26.2) node {$F_{7^4}$};
\draw [black] (4.6,-27) circle (3);
\draw (4.6,-27) node {$F_{7^9}$};
\draw [black] (19.2,-39.3) circle (3);
\draw (19.2,-39.3) node {$F_{7^3}$};
\draw [black] (51.7,-39.3) circle (3);
\draw (51.7,-39.3) node {$F_{7^2}$};
\draw [black] (35.8,-49.8) circle (3);
\draw (35.8,-49.8) node {$F_7$};
\draw [black] (6.85,-25.02) -- (17.55,-15.58);
\draw [black] (22.34,-12.01) -- (33.76,-4.89);
\draw [black] (6.89,-28.93) -- (16.91,-37.37);
\draw [black] (21.53,-37.41) -- (33.57,-27.59);
\draw [black] (22.2,-15.4) -- (33.5,-23.9);
\draw [black] (38.8,-4.96) -- (49.4,-12.04);
\draw [black] (49.5,-15.5) -- (38.3,-23.9);
\draw [black] (54,-15.85) -- (63,-24.05);
\draw [black] (63.04,-28.38) -- (53.76,-37.12);
\draw [black] (38.17,-27.66) -- (49.43,-37.34);
\draw [black] (49.2,-40.95) -- (38.3,-48.15);
\draw [black] (21.74,-40.9) -- (33.26,-48.2);
\end{tikzpicture}
\end{center}
 	\end{enumerate}
\end{document}