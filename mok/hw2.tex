\documentclass[a4paper,12pt]{article}

\usepackage[ukrainian,english]{babel}
\usepackage{ucs}
\usepackage[utf8]{inputenc}
\usepackage[T2A]{fontenc}

\usepackage{amsmath}
\usepackage[document]{ragged2e}
\usepackage{graphicx}
\usepackage{wrapfig}
\usepackage{enumitem}
\usepackage{cases}

\usepackage[left=20mm, top=20mm, right=20mm, bottom=20mm, nohead, nofoot]{geometry}

\newcommand\tab[1][0.5cm]{\hspace*{#1}}

\begin{document}
	\begin{justify}
		\thispagestyle{empty}\setlength{\parindent}{0pt}
  		\topskip0pt
 		\vspace*{\fill}
  		\begin{center}
  			\noindent\makebox[\linewidth]{\rule{\paperwidth}{0.4pt}}
   			\LARGE{\bigbreak ДОМАШНЯ РОБОТА №2\\З ПРЕДМЕТУ\\''МАТЕМАТИЧНІ ОСНОВИ КРИПТОЛОГІЇ''\\\bigbreak} 
   			ФІ-12 Бекешева Анастасія 
   			\noindent\makebox[\linewidth]{\rule{\paperwidth}{0.4pt}}
  		\end{center}
 		\vspace*{\fill}\newpage
 		\begin{enumerate}
 			\item $$\varphi(14553)$$
 			$\varphi(14553) = \varphi(3\cdot3\cdot3\cdot7\cdot7\cdot11)=14553\times\left(1-\dfrac13\right)\times\left(1-\dfrac17\right)\times\left(1-\dfrac1{11}\right)=7560$
 			\item $$5^82\mod 24$$
 				Перевіримо чи є 5 та 24 взаємнопростими (знайдемо $\gcd(5, 24)$):
 				$\\24=5\cdot4+4,\tab 5=4\cdot1,\tab 4=1\cdot4+0,\tab\gcd(5, 24)=1$
\begin{table}[htp]\centering
\begin{tabular}{|c|c|c|c|c|}
\hline
{$q_i$} & {} & {4} & {1} &    \\ \hline
$u_i$          & 0         & 1          & -4         & 5  \\ \hline
$v_i$          & 1         & 0          & 1          & -1 \\ \hline
\end{tabular}
\end{table}
				\\Тепер порахуємо: $\varphi(24)=\varphi(2\cdot2\cdot2\cdot3)=24\times\left(1-\dfrac12\right)\times\left(1-\dfrac13\right)=8.$
				\\Отже $5^{82}=5^{80}\cdot25\mod 24=1\mod4$
 			\item $$\begin{cases}
 				x=2\mod5\\x=8\mod13\\x=2\mod9\\x=5\mod7
 			\end{cases}$$
 			Перевіримо чи усі $n_i$ попарно взаємнопрості:\\
 			$\gcd(5,13)=1,\gcd(5,9)=1,\gcd(5,7)=1,\gcd(13,9)=1,\gcd(13,7)=1,\gcd(9,7)=1\\$
 			Знайдемо $M$: $M=5\cdot13\cdot9\cdot7=4095$. Отже $M_1=819,M_2=315,M_3=455,M_4=585$
\begin{table}[htp]\centering
\begin{tabular}{ccccc}
\multicolumn{5}{c}{$\gcd(5, 819) = 1$}                                                                                                 \\ \hline
\multicolumn{1}{|c|}{$q_i$} & \multicolumn{1}{c|}{}  & \multicolumn{1}{c|}{163} & \multicolumn{1}{c|}{1}    & \multicolumn{1}{c|}{}    \\ \hline
\multicolumn{1}{|c|}{$u_i$} & \multicolumn{1}{c|}{0} & \multicolumn{1}{c|}{1}   & \multicolumn{1}{c|}{-163} & \multicolumn{1}{c|}{164} \\ \hline
\multicolumn{1}{|c|}{$v_i$} & \multicolumn{1}{c|}{1} & \multicolumn{1}{c|}{0}   & \multicolumn{1}{c|}{1}    & \multicolumn{1}{c|}{-1}  \\ \hline
\end{tabular}\tab\tab
\begin{tabular}{ccccc}
\multicolumn{5}{c}{$\gcd(13, 315) = 1$}                                                                                                                                                                                                  \\ \hline
\multicolumn{1}{|c|}{{$q_i$}} & \multicolumn{1}{c|}{{}} & \multicolumn{1}{c|}{{24}} & \multicolumn{1}{c|}{{4}} & \multicolumn{1}{c|}{}   \\ \hline
\multicolumn{1}{|c|}{$u_i$}                           & \multicolumn{1}{c|}{0}                          & \multicolumn{1}{c|}{1}                            & \multicolumn{1}{c|}{-24}                         & \multicolumn{1}{c|}{97} \\ \hline
\multicolumn{1}{|c|}{$v_i$}                           & \multicolumn{1}{c|}{1}                          & \multicolumn{1}{c|}{0}                            & \multicolumn{1}{c|}{1}                           & \multicolumn{1}{c|}{-4} \\ \hline
\end{tabular}
\end{table}
\begin{table}[htp]\centering
\begin{tabular}{cccccc}
\multicolumn{6}{c}{$\gcd(9, 455) = 1$}                                                                                                                                                                                                                               \\ \hline
\multicolumn{1}{|c|}{{$q_i$}} & \multicolumn{1}{c|}{{}} & \multicolumn{1}{c|}{{50}} & \multicolumn{1}{c|}{{1}} & \multicolumn{1}{c|}{1}  & \multicolumn{1}{c|}{}     \\ \hline
\multicolumn{1}{|c|}{$u_i$}                           & \multicolumn{1}{c|}{0}                          & \multicolumn{1}{c|}{1}                            & \multicolumn{1}{c|}{-50}                         & \multicolumn{1}{c|}{51} & \multicolumn{1}{c|}{-101} \\ \hline
\multicolumn{1}{|c|}{$v_i$}                           & \multicolumn{1}{c|}{1}                          & \multicolumn{1}{c|}{0}                            & \multicolumn{1}{c|}{1}                           & \multicolumn{1}{c|}{-1} & \multicolumn{1}{c|}{2}    \\ \hline
\end{tabular}\tab\tab 
\begin{tabular}{cccccc}
\multicolumn{6}{c}{$\gcd(7, 585)=1$}                                                                                                                                                                                                                                   \\ \hline
\multicolumn{1}{|c|}{{$q_i$}} & \multicolumn{1}{c|}{} & \multicolumn{1}{c|}{{83}} & \multicolumn{1}{c|}{{1}} & \multicolumn{1}{c|}{1}  & \multicolumn{1}{c|}{}     \\ \hline
\multicolumn{1}{|c|}{$u_i$}                           & \multicolumn{1}{c|}{0}                          & \multicolumn{1}{c|}{1}                            & \multicolumn{1}{c|}{-83}                         & \multicolumn{1}{c|}{84} & \multicolumn{1}{c|}{-167} \\ \hline
\multicolumn{1}{|c|}{$v_i$}                           & \multicolumn{1}{c|}{1}                          & \multicolumn{1}{c|}{0}                            & \multicolumn{1}{c|}{1}                           & \multicolumn{1}{c|}{-1} & \multicolumn{1}{c|}{2}    \\ \hline
\end{tabular}
\end{table}\\
			Порахуємо $N_i$: $N_1=819^{-1}\mod 5=-1,N_2=315^{-1}\mod13=-4,\\N_3=455^{-1}\mod9=2,N_4=585^{-1}\mod 7=2$. Знайдемо $x$ за формулою $\\x=a_1N_1m_1+\cdots+a_nN_nM_n:$ $x=-4048 \mod 4095  = 47$\newpage
			\item $$a=x^2\mod13$$
			$\dfrac{p-1}{2}=\dfrac{13-1}{2}=6$. Тобто 13 має 6 квадратичних лишків.
\begin{table}[htp]\centering
\begin{tabular}{|c|c|c|c|c|c|c|c|c|c|c|c|c|}
\hline
$x$          & 1 & 2 & 3 & 4 & 5  & 6  & 7  & 8  & 9 & 10 & 11 & 12 \\ \hline
$x^2\mod 13$ & 1 & 4 & 9 & 3 & 12 & 10 & 10 & 12 & 3 & 9  & 4  & 1  \\ \hline
\end{tabular}
\end{table}\\
			З таблиці легко бачити, що 1,3,4,9,10,12 є квадратичними лишками 13.
			\item $$\left(\dfrac{18}{53}\right)=?$$
				$\left(\dfrac{18}{53}\right)=\left(\dfrac{2}{53}\right)\times\left(\dfrac{3}{53}\right)\times\left(\dfrac{3}{53}\right)=-\left(\dfrac{9}{53}\right)=-1\\\\$
				Отже 18 не є квадратичним лишком 53.
			\item $$\left(\dfrac{2}{8k+5}\right)=-1$$
				$\left(\dfrac{2}{8k+5}\right)=(-1)^{\left(\dfrac{(8k+5)^2-1}{8}\right)}=(-1)^{8k^2+10x+3}$. Розглянемо $8k^2+10x+3$: $8k^2$ та $10k$ завжди будуть парними, адже добуток парного числа з будь-яким іншим числом є парне число. Тобто $(8k^2+10x+3)\mod 2=8k^2\mod 2+10k\mod 2+3\mod 2=1$. 
				Отже $(-1)^{8k^2+10x+3}=-1$. Іншими словами 2 є квадратичним нелишком за модулем $p$, де $p=8k+5$.
 			
 	
 			
 			
 			
 			
 			
 			
 			
 			
 			
 			
 			
 			
 			
 			
 			
 			
 			
 			
 			
 			
 			
 			
 			
 			
 		\end{enumerate}
	\end{justify}
	
	
	
	
	
	
	
	
	
	
	
	
	
	
	
	
	
	
	
	
	
	
	
	
	
	
	
	
	
	
	
	
	
	
	
\end{document}