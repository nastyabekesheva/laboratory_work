\documentclass[a4paper,12pt]{bookest}
\usepackage[russian,english]{babel}
\usepackage{ucs}
\usepackage[utf8]{inputenc}
\usepackage[T2A]{fontenc}
\usepackage{graphicx}
\usepackage{wrapfig}
\usepackage{amsfonts}
\usepackage{amsthm}
\usepackage{amssymb}
\usepackage{amsmath}
\usepackage{mdframed} 
\usepackage{xcolor}
\usepackage{xparse} 
\usepackage{scalerel}
\usepackage{stackengine}
\usepackage{bigints}
\usepackage{enumerate}	
\usepackage{multicol}
\usepackage{longdivision}
\usepackage{polynom}

\polyset{%
  style=C,
  delims={\big(}{\big)},
  div=:
}
\newtheorem{theorem}{Theorem}[section]
\theoremstyle{remark}
\newtheorem*{remark}{Remark}
\newtheorem{cons}{Consequence}
\newtheorem*{cons*}{\textbf{Consequence}}
\newcommand\showdiv[1]{\overline{\smash{\hstretch{.5}{)}\mkern-3.2mu\hstretch{.5}{)}}#1}}
\let\ph\phantom
\newcommand\tab[1][1cm]{\hspace*{#1}}
\newcommand\dx{\textbf{ d}x}
\newcommand\dy{\textbf{ d}}
\newcommand\lcm{\textrm{lcm}}
\newmdenv[   linecolor=black,
  topline=false,
  bottomline=false,
  rightline=false,
  skipabove=\topsep,
  skipbelow=\topsep
]{leftrule}

\NewDocumentEnvironment{example}{O{\textbf{Example:}}} {\begin{leftrule}\noindent\textcolor{black}{#1}\par}
{\end{leftrule}}

\begin{document}
\title{Математический анализ 2}
\maketitle
\date{}
\catcode`@=11
\def\caseswithdelim#1#2{\left#1\,\vcenter{\normalbaselines\m@th
  \ialign{\strut$##\hfil$&\quad##\hfil\crcr#2\crcr}}\right.}\catcode`@=12

\def\bcases#1{\caseswithdelim[{#1}}
\def\vcases#1{\caseswithdelim|{#1}}

\newpage
\tableofcontents
\newpage
\chapter{Неопределенный интеграл}
\section{Понятие первообразной и неопределеного интеграла}
Функция $F(x)$ называется \textbf{первообразной} функции $f(x)$ на множестве $X\subset \mathbb{R}$, или $\forall x\in X\>\>F'(x)=f(x)$.
\begin{theorem}
Если $F(x)$ - некоторая первообразная для $f(x)$ на множестве $X$, то любая другая первообразная имеет вид: $F(x)+c$, где $c=const$ - произвольная.
	
\end{theorem}
\begin{proof}
Пусть $F(x)$ - первообразная функции  $f(x)$, т.е.  $F'(x)=f(x)$; Тогда: $(F(x)+c)'=F'(x)+0=f(x)\Rightarrow F(x)+c$ - первообразная для $f(x)(c=const)$.\\ 
Пусть $F_1(x)$ - тоже первообразная для $f(x)$, т.е. $F_1'(x)=f(x)$.\\
Рассмотрим разность: $F_1(x)-F(x);$\\
$(F_1(x)-F(x))'=F_1'(x)-F'(x)=f(x)-f(x)=0\Rightarrow F_1(x)-F(x)=c=const$, т.е. :$F_1(x)=F(x)+c$
\end{proof}
Таким образом множество всех первообразных функции $f(x)$ имеет вид $F(x)+c$.\\
Множество всех первообразных функции $f(x)$ называется \textbf{неопределенным интегралом} этой функции и обозначается $\int f(x)dx$.\\
$f(x)$ - подинтегральная функция, $f(x)dx$ - подинтегральное выражение, $x$ - переменная инегрирования, $\int$ - неопределенный интеграл.\\ \begin{example}
$${
f(x) = sign(x) = \bcases{1,\>\> x>0\cr0,\>\> x=0\cr -1,\>\> x<0},\>\>x\in(-1;\>1).
}$$
Предположим, что существует такая первообразная $\exists F(x):\> \forall x\in(-1;\>1):\>\>F'(x)=sign(x)$, т.е.
$${
F'(x)=  \bcases{1,\>\> x\in(0;\>1)\cr0,\>\> x=0\cr -1,\>\> x<\in(-1;\>0)}\Rightarrow \begin{cases}
F'(x)=0\cr F_-'(x)=-1\cr F_+'(x)=1
\end{cases}\Rightarrow
}$$ 
$F'(0)$ - не существует $\Rightarrow$ противоречие $\Rightarrow F(x)$ не существует.\end{example}
\begin{remark}
Достаточным условием существования первообразной у функции  на данном множестве является ее непрерывность на этом множестве.
\end{remark}
\newpage
\section{Свойства неопределенного интеграла}
Пусть $\int f(x)\dx=F(x)+c\>(F'(x)=f(x))$.
\begin{enumerate}
	\item Производная от неопределенного интеграла равна подинтегральной функции, дифференциал неопределенного интеграла равен подинтегральному выражению. \\
	$$(\int f(x)\dx)'_x=f(x);\>\> \dy(\int f(x)\dx)=f(x)\dx.$$
	\begin{proof}
		$(\int f(x)\dx)'_x=(F(x)+c)'_x=F'(x)+c'=f(x);$	\\$\dy(\int f(x)\dx)=(\int f(x)\dx)'_x\dx=f(x)\dx$.
	\end{proof}
	\item Неопределенный интеграл от дифференциала некоторой функции равен сумме этой функции и произвольной постоянной.
	$$\int \dy(F(x))=F(x)+c.$$
	\begin{proof}
		$\int \dy(F(x))=\int F'(x)\dx=\int f(x)\dx=F(x)+c$
	\end{proof}
	\item Постоянный множитель можно выносить за знак интеграла.
	$$\int af(x)\dx=a\int f(x)\dx,\>\> a=const.$$
	\begin{proof}
		$\int af(x)\dx=\int a F'(x)\dx=\int(aFx)'\dx=\int \dy(aF(x))=(aF(x)++c_1)=a(F(x)+\frac{c_1}{a}=\Big|c=\frac{c_1}{a}\Big|=a(f(x)+c)=a\int f(x)\dx$
	\end{proof}
	\item Интеграл суммы двух функций равен сумме интегралов этих функций.
	$$\int(f(x)+g(x))dx=\int f(X)dx+\int g(x)dx.$$
	\begin{proof}
		Пусть $\int g(x)\dx=G(x)+c$; тогда $\int(f(x)+g(x))\dx=\int(F'(x)++G'(x))\dx=\int(F(x)+G(x))'\dx=\int \dy(F(x)+G(x))=F(x)+G(x)++c=\Big|c=c_1+c_2\Big|=(F(x)+c_1)+(G(x)c_2)=\int f(x)\dx+\int g(x)\dx$
	\end{proof}
\end{enumerate}
\begin{remark}$ $
	\begin{enumerate}
		\item[-]свойство 4 справедливо для любого конечного числа слогаемых 
		\item[-]свойство 3-4 называются свойством линейности неопределенного интеграла 
		\item[-]свойство 1-2 отражают связь операций дифференцирования и интегрирования  
	\end{enumerate}		
\end{remark}
\section{Таблица основных неопределенных интегралов}
\begin{enumerate}
	\item $$\int x^\alpha \dx=\frac{x^{\alpha+1}}{\alpha+1}+c;\>\>\alpha\in\mathbb{R}\backslash\{-1\}$$
	\item $$\int\frac{\dx}{x}=\ln|x|+c$$
	\item $$\int a^x\dx=\frac{a^x}{\ln a}+c,\>\>a>0$$
	\item $$\int e^x\dx=e^x+c$$
	\item $$\int\sin x\>\dx	=-\cos x+c$$
	\item $$\int\cos x\>\dx=\sin x+c$$
	\item $$\int\frac{\dx}{\cos^2x}=\tan x+c$$
	\item $$\int\frac{\dx}{\sin^2x}=-\cot x+c$$
	\item $$\int\textrm{sh}x\>\dx=\textrm{ch}\>x+c$$
	\item $$\int\textrm{ch}x\>\dx=\textrm{sh}\>x+c$$
	\item $$\int\frac{\dx}{\textrm{ch}^2x}=\textrm{th}\>x+c$$
	\item $$\int\frac{\dx}{\textrm{sh}^2x}=-\textrm{cth}\>x+c$$
	\item $$\int\frac{\dx}{\sqrt{1-x^2}}=\arcsin x+c$$
		$$\int\frac{\dx}{\sqrt{a-x^2}}=\arcsin \frac{x}{a}+c$$
	\item $$\int\frac{\dx}{1+x^2}=\arctan x+c$$
		$$\int\frac{\dx}{a^2+x^2}=\frac{1}{a}\arctan\frac{x}{a}+c$$
		Дополнительные формулы: 
	\item $$\int\frac{\dx}{x^2-a^2}=\frac{1}{2a}\ln|\frac{x-a}{x+a}|+c\textrm{ - высокий логарифм}$$ 
	\item $$\int\frac{\dx}{\sqrt{x^2+A}}=\ln|x+\sqrt{x^2+A}|+c\textrm{ - длинный логарифм }$$
	\item $$\int\sqrt{x^2+A}\>\dx=\frac{x}{2}\sqrt{x^2+A}+\frac{A}{2}\ln|x+\sqrt{x^2+A}|+c$$
	\item $$\int\sqrt{a^2-x^2}\>\dx=\frac{x}{2}\sqrt{a^2-x^2}+\frac{a^2}{2}\arcsin\frac{x}{a}+c$$
\end{enumerate}
В этих формулах вместо $x$ может быть записана произвольная дифференцируемая функция от $x$.
\section{Основные примеры интегрирования}
\subsection{Непосредственное интегрирование}
Непосредственное интегрирование заключается в использовании тождественных преобразований подинтегральнной функции, свойства линейности интеграла и таблицы интегралов.\\
\begin{example}
\begin{enumerate}
	\item $\int(\frac{\sqrt{x}+1}{\sqrt[3]{x}})^2\dx=\int\frac{x+2x^{\frac{1}{2}}+1}{\sqrt[3]{x^2}}\dx=\int(x^{\frac{1}{3}}+2x^{\frac{-1}{6}}+x^{\frac{-2}{3}})\dx=\\=\frac{3}{4}x^{\frac{4}{3}}+2\cdot\frac{x^{\frac{5}{6}}}{5}\cdot 6+x^{\frac{1}{3}}\cdot 3+c$
	\item $\int\frac{\dx}{\sin^2x\cos^2x}=|\cos^2x+\sin^2x=1|=\int\frac{\cos^2x+\sin^2x}{\sin^2x\cos^2x}\dx=\int\frac{1}{\sin^2x}\dx+\int\frac{1}{\cos^2x}\dx=\tan x-\cot x+c$
\end{enumerate}
\end{example}
\subsection{Замена переменной}
\begin{theorem}
Пусть на $\forall x\in(a;\>b)\>\>\int f(x)dx=F(x)+c$, (на всем интервале $(a;\>b)$ известна первообразная функции): $F'(x)=f(x)\>\>x=\varphi(t)$ - функция дифференцируемая; причем $\varphi(t):\>t\in(\alpha;\>\beta)$ и $\varphi:(\alpha;\>\beta)\rightarrow(a;\>b)$.
Тогда справедлива формула:
$$\int f(\varphi(t))\cdot \varphi_t'(t)dt=F(\varphi(t))+c$$	
\end{theorem}
\begin{proof}
	$(f(\varphi(t)))_t'=F_\varphi'(\varphi(t))\cdot\varphi_t'(t)=|\varphi(t)=x|=F_x'(x)\varphi_t'(t)=|F_x'(x)=f(x)|=f(x)\cdot\varphi(t)=|x=\varphi(t)|=f(\varphi(t))\cdot\varphi_t'(t)\Rightarrow F(\varphi(t))\textrm{ первообразная для } \\f(\varphi(t))\cdot\varphi_t'(t)\Rightarrow\int f(\varphi(t))\cdot\varphi_t'(t)dt=F(\varphi(t))+c$
\end{proof}
\begin{remark}
$$\varphi_t'(t)\dy t=\dy(\varphi(t))\Rightarrow\int f(\varphi(t))\cdot \dy\varphi = F(\varphi)+c$$	
\end{remark}
\begin{enumerate}
	\item \textbf{Внесения выражения под знак дифференциала}$$\int f(x)\dx=\int g(\varphi(x))\cdot\varphi_x'(x)\dx=\int g(\varphi)\dy\varphi=|G(x)-\textrm{известно}\>G'(x)=g(x)|=$$$$=G(\varphi)+c=G(\varphi(x))+c.$$
	Часто используются преобразование дифференциала $\dx=\dy(x++a)=\frac{1}{k}\dy(kx)=\frac{1}{k}\dy(kx+b)\\ x^{n-1}\dx=\frac{1}{n}\dy(x^n)$\\
	\textbf{Преобразования дифференциалов}\\
	$\sin x\>\dx=-\dy(\cos x)$\\
	$\cos x\>\dx=\dy(\sin x)$\\
	$\frac{\dx}{\cos^2x}=\dy(\tan x)$\\
	$\frac{\dx}{x}=\dy(\ln x)$\\
	$\frac{\dx}{1+x^2}=\dy(\arctan x)$\\
	$\frac{\dx}{\sqrt{1-x^2}}=\dy(\arcsin x)$\\
	\begin{example}
	$\int \sin^3x\>\dx=\int\sin x\sin^2x\>\dx\int(1-\cos^2x)\cdot(-\dy(\cos x))=\int(\cos^2-1)\dy(\cos x)=\\=\frac{\cos^3x}{3}-\cos x+c.$\end{example}
	\item \textbf{Вынесения выражения из-под знак дифференциала}\\
	$\int f(x)\>\dx = |x=\varphi(t)\Rightarrow \dx=\varphi'(t)\dy t|=\int f(\varphi(t))\cdot\varphi'(t)\dy t=|g(t)=\\=G'(t)|=G(t)+c=|x=\varphi(t)\>\>t=\varphi^{-1}(x)|=G(\varphi^{-1}(x))+c$\\\\
	\begin{example}
	$\int\sqrt{a^2-x^2}\dx=|x=a\sin t\>\>\dx=a\cos t\>\dy t|=\int\sqrt{a^2-a^2\sin^2t}a\cos t\>\dy t=\\=a^2\int\sqrt{1-\sin^2t}\cdot\cos t\>\dy t=a^2\int\cos^2 t\>\dy t=\frac{a^2}{2}\int(1+\cos^2t)\dy t=\frac{a^2}{2}(t+\frac{\sin2t}{2})+c=\frac{a^2}{2}(t+\sin t\cos t)+c=|\cos t=\sqrt{1-\sin t}\>\>\frac{x}{a}\>\>\\t=\arcsin\frac{x}{a}|=\frac{a^2}{2}(\arcsin\frac{x}{a}+\sqrt{1-\frac{x^2}{a^2}}\cdot\frac{x}{a})=\frac{a^2}{2}\arcsin\frac{x}{a}+\frac{x}{2}\sqrt{a^2-x^2}+c$\end{example}
\end{enumerate}
\subsection{Интегрирование по частям}
Пусть $u=u(x),\>v=v(x)$ - две диффренцируемые функции. \\
по свойству дифференциала:\\$\dy(uv)=u\dy v+v\dy u\Rightarrow\int \dy (uv)=\int u\dy v+\int v\dy u$ - формула интегрирования по частям.\\
В исходном интеграле $\int f(x)\dx$ подинтегральное выражение представляется в виде двух сомножителей. Как правило, это можно сделать неоднозначно.\\
После того как $u$ и $\dy v$ выбраны, находим $\dy u ,\>v,\>...$\\
$\int f(x)\dx = |f(x)=u,\>\dx=\dy v|\Rightarrow \dy u=u'\dx=...\Rightarrow v=\int \dy v$ \\
в результате применения формулы полученный интеграл оказывается более простым, чем исходный.\\
При необходимости формула интегрирования по частям применяется несколько раз.
\begin{enumerate}[I.]
	\item $\int P_n(x)$
		$\left\{\begin{array}{lr}
        \sin(kx+b)\\ \cos(kx+b)\\a^{kx}\\e^{kx}\\\textrm{sh}kx,\>\textrm{ch}(kx)
        \end{array}\right\}\dx\tab U=Pn(x);\>\>\dy v=\big\{\dots\big\}$
        \item $\int P_n(x)$
		$\left\{\begin{array}{lr}
        \arcsin x\\ \arccos x\\\arctan x\\\ln x\\
        \end{array}\right\}\dx\tab U=\big\{\dots\big\};\>\>\dy v=Pn(x)\dx$
        \item $\int e^{kx}$
		$\left\{\begin{array}{lr}
        \sin (ax+b)\\ \cos (ax+b)\end{array}\right\}\dx\tab U=e^{kx};\>\>\dy v=\big\{\dots\big\}\dx$
\end{enumerate}

\begin{example}
$\int_{I} e^x\sin2x\dx\underset{III}{=}\bigg|u=e^x\Rightarrow \dy u=e^x\dx;\>\sin2x\dx=\dy v;\>v=\int\sin2x\dx=\\=-\frac{\cos2x}{2}\bigg|=-\frac{e^x\cos2x}{2}+\int\frac{\cos2x}{2}\cdot e^xdx=\bigg|u=e^x;\>\dy u=e^x;\>\dy v=\cos2x\dx;\>v=\\=\frac{\sin2x}{2}\bigg|=-\frac{e^x\cos2x}{2}+\frac{1}{2}(\frac{e^x\sin2x}{2}-\int\frac{\sin2x}{2}\cdot e^x\dx)=-\frac{e^x\cos2x}{2}+\frac{1}{4}e^x\sin2x-\\-\frac{1}{4}\int \underset{I}{e^x\sin2x}\dx\\I=-\frac{e^x\cos2x}{2}+\frac{1}{4}e^x\sin2x-\frac{1}{4}I;\>\>I=-\frac{e^x\cos2x}{2}+\frac{1}{4}e^x\sin2x.\\I=\frac{4}{5}(\frac{1}{4}e^x\sin2x-\frac{1}{2}e^x\cos2x)+c$
\end{example}
\section{Интегрирование рациональных функций}
\subsection{Основные сведения о рациональных функциях}
\begin{enumerate}
	\item \textbf{Многочлен(целая рациональная функция)\\ Многочленом} $P_n(x)$ называется функция вида $P_n(x)=a_nx^n+a_{n-1}x^{n-1}+\dots+a_1x^1+a_0;$ где $n\in\mathbb{N},\>\>a_i\in\mathbb{R},\>\>i=\overline{0,n}$\\
	\textbf{Корнем} многочлена называется значение $x_0$(вообще говоря, комплексное) аргумента $x$, при котором многочлен обращается в ноль.\\$x_0$ - корень $P_n(x)$ или $P_n(x_0)=0$ 
	\begin{theorem}
		$$\textrm{Если }x_0 - \textrm{корень многочлена } P_n(x), \textrm{ то многочлен делится нацело на} (x-x_0), $$$$\textrm{т.е. } P_n(x)\textrm{ представлется в виде: } P_n(x)=(x-x_0)\cdot Q_{n-1}(x),$$$$ \textrm{где } Q - \textrm{многочлен степени} n-1$$
	\end{theorem}
	\begin{theorem}
		$$\textrm{Всякий многочлен степени }n>0\textrm{ имеет по крайней мере один корень,}$$$$ \textrm{действительный или комплексный}$$
	\end{theorem}
	\begin{cons*}$\\$
		\begin{enumerate}[(1)]
			\item Многочлен $n$-ой степени можно представить в виде: $P_n(x)=a_n(x-x_1)(x-x_2)\dots(x-x_n),$ где $x_1,\>\dots,\>x_n$ - корни $P_n(x),\>a_n$ - старший коэффициент 
			\item Если среди корней многочлена имеются одинаковые, то объединим соответствующие или множители. Получим: \\$P_n(x)=a_n(x-x_1)^{k_1}(x-x_2)^{k_2}\dots (x-x_n)^{k_m},$ где $k_1+k_2+\dots+k_m=n.$ для $x_i:\>(x-x_i)^k_i;\>k_i$ - кратность корня $x_i$. \\ Такое представление называется разложением многочлена на линейные множители. 
		\end{enumerate}	
	\end{cons*}
	\begin{theorem}
		$\\$Известно, что если многочлен имеет комплексный корень $\\ x_0=a_i+ib(a,\>b\in\mathbb{R};\>x_0\in\mathbb{C}),$  то комплексное спряженое число  $\\ \bar{x}=a-ib$ - тоже корень $P_n(x).$ Таким образом, в разложении многочлена комплексно спряженные числа входят парами, пeремножим: \\$(x-(a+ib))(x-(a-ib))=x^2-x(a+ib)-x(a-ib)+(a+ib)(a-ib)=x^2-ax-ibx-ax+ibx+a^2+b^2=x^2-2ax+a^2+b^2$.\\Полученый трехчлен имеет действительный коэффициент, причем дискретный $D=B^2-4A\cdot C=4a^2-4(a^2+b^2)=-4b^2<0$\\Получаем, что пару множителей, соответсвующую двум комплексных сопряженным корням можно заменить квадратный трехчлен с действительным коэффициентом и $D<0$.\\Окончательно получим разложение на множители в виде:\\ $P_n(x)=(x-x_1)^{k_1}(x-x_2)^{k_2}(x-x_5)^{k_5}(x^2+p_1x+q_1)^{l_1}\dots(x^2p_mx+q_m)^{l_m},$ где $x_1,\>\dots,\>x_5\in\mathbb{R}$ - корни многочлена $Pn(x); p_i,\>q_i\in\mathbb{R},\>i=\overline{1,m};\\D_i=p_i^2-4q_i<0.\tab k_1+\dots+k_5+2(l_1+\dots+l_m)=n$
	\end{theorem}
	\textbf{Многочлен} называется \textbf{тождественно равным нулю} \\
	$$Pn(x)\equiv0,\textrm{ если }\forall x\in\mathbb{R}\>\>Pn(x)=0$$
	\begin{theorem}
		$$\textrm{Многочлен тожественно равен нулю тогда и только тогда, когда}$$$$ \textrm{все его коэффициенты равны нулю } a_i=0,\>i=\overline{0,n}$$
	\end{theorem}
	\begin{cons*}$\\$
		Два многочлена тождественно равны, если их степени одинаковы и имеют одинаковые коэффициенты при одинаковых степенях $x$
		\begin{proof}
			$P_n(x)\equiv Q_n(X)\\ P_n(x)-Q_n(x)\equiv0\\ \underset{=0}{(a_n+b_n)}x^n+\underset{=0}{(a_{n-1}+b_{n-1})}x^{n-1}+\dots+(a_0+b_0)=0$
		\end{proof} 
	\end{cons*}
	\begin{example}
		$P_3(x)=3x^2-2x+4\\Q_4(x)=a_4x^4+a_3x^3-a_2x^2+a_1x+a_0\\P_3(x)\equiv Q_4(x)\Rightarrow\begin{cases}
			a_4=0\\a_3=3\\a_2=0\\a_1=-2\\a_1=4
		\end{cases}$	
	\end{example}
	\item \textbf{Дробная рациональная функция}\\
	\textbf{Дробной рациональной функцией} называется отношение двух многочленов.
	$\frac{P_n(x)}{Q_m(x)}>$ многочлены $\big\{$ дробная рациональная функция, рациональная дробь. Если $n\geq m$, то рациональная дробь \textbf{неправильная}, если $n<m$ - \textbf{правильная}. 
	\begin{theorem}
		$$\textrm{Неправильная рациональная дробь может быть представлена в виде}$$ $$\textrm{суммы многочлена и правильной рациональной дроби.}$$ $$\frac{P_n(x)}{Q_m(x)}=\underset{\text{многочлен}}{\overset{\text{целая часть}}{L_{n-m}(x)}}+\frac{R_k(x)}{Q_m(x)},\>k<m,\>R_n(x)\textrm{ - многочлен.}$$
	\end{theorem}
	Элементарные(простейшие) рациональные дроби: 
	\begin{enumerate}[I.]
		\item $$\frac{A}{x-a}\tab A,\>a\in\mathbb{R}$$
		\item $$\frac{A}{(x-a)^k}\tab k\in\mathbb{N},\>k>1,\>A,\>a\in\mathbb{R}$$
		\item $$\frac{Mx+N}{x^2+px+q}\tab M,\>n,\>p,\>q\in\mathbb{R},\>D=p^2-4q<0$$
		\item $$\frac{Mx+N}{(x^2+px+q)^k}\tab M,\>n,\>p,\>q\in\mathbb{R},\>D=p^2-4q<0,\>k\in\mathbb{N},\>k>1$$
	\end{enumerate}
	\begin{theorem}
		$\\$Пусть $\frac{P_n(x)}{Q_m(x)}$ - правильная рациональная дробь($n<m$), и знаменатель дроби $Q_m(x)$ разложен на множители: $$Q_m(x)=\underbrace{(x-x_1)^{k_1}\dots(x-x_5)^{k_5}}_{\textrm{действительные корни}}\underbrace{(x^2+p_1x+q_1)^{l_1}\dots(x^2+p_mx+q_m)^{l_m}}_{D<0}$$ Тогда заданная дробь раскладывается в сумму простых дробей следующего вида:$$\frac{P_n(x)}{Q_m(x)}=\frac{A_1}{x-x_1}+\frac{A_2}{(x-x_1)^2}+\dots+\frac{A_{k_1}}{(x-x_1)^{k_1}}+\dots+$$ $$+\frac{F_1}{x-x_5}+\frac{F_2}{(x-x_5)^2}+\dots+\frac{F_{k_5}}{(x-x_5)^{k_5}}+\frac{M_1x+N_1}{(x^2+p_1x+q_1)}+\dots+\frac{M_{l_1}+N_l}{(x^2+p_lx+q_l)^l}+$$При этом: $$(x-x_i)^{k_i}\leftrightarrow\frac{A_1}{x-x_i}+\frac{A_2}{(x-x_i)^2}+\dots+\frac{A_{k_i}}{(x-x_	i)^{K}};$$$$(x^2+p_jx+q_j)^{l_j}\leftrightarrow\frac{M_1x+N_1}{(x^2+p_jx+q_j)}+\frac{M_2x+N_2}{(x^2+p_jx+q_j)}+\dots+\frac{M_{l_j}x+N_{l_j}}{(x^2+p_jx+q_j)}$$В разложении появляются так называемые неопределенные коэффициенты, которые подлежат дальнейшиму определению.
	\end{theorem}
	\begin{example}
		$\frac{3x-2}{(x-1)^3(x+2)(x^2+1)(x^2+2x+3)^2}=\frac{A_1}{x-1}+\frac{A_2}{(x-1)^2}+\frac{A_3}{(x-1)^3}+\frac{B}{x+2}+\frac{Cx+D}{x^2+1}+\frac{Ex+F}{x^2+2x+3}+\\+\frac{Mx+N}{(x^2+2x+3)^2}$
	\end{example}
	Для того, чтобы найти неопределенные коэффициенты в полученном выражении, умножают обе части тождества на знаменатель левой части. Таким образом, получают 2 тождественно равных многочлена. Раскрывая скобки справа, после сего приравнивают коэффициенты при одинаковых степенях. Получают систему линейных уравнений для определения неизвестных коэффициентов.
	\begin{example}
		$\frac{x^4+2x^3+5x^2-1}{x(x^2+1)^2}=\frac{A}{x}+\frac{Bx+C}{x^2+1}+\frac{Dx+E}{(x^2+1)^2}\>\>\bigg|x(x^2+1)^2\\x^4+2x^3+5x^2-1=a(x^2+1)^2+(Bx+C)x(x^2+1)+(Dx+E)x=\\=A(x^4+2x^2+1)+(Bx+C)(x^3+x)+Dx^2+Ex=\\=Ax^4+2Ax^2+A+Bx^4+Cx^3+Bx^2+Cx+Dx^2+Ex=\\=(A+B)x^4+Cx^3)+(2A+B+D)x^2+(E+C)x+A.\\\begin{cases}
			x^4:\tab A+B=1\tab\tab\>\>A=1\\ x^3:\tab C=2\tab\tab\tab  B=2\\ x^2:\tab 2A+B+D=5\tab\>C=2\\ x^1:\tab C+E=0\tab\tab\>D=5\\ x^0:\tab A=-1\tab\tab\>\>\>\>\>\ E=-2\\
		\end{cases}\\\\ \frac{x^4+2x^3+5x^2-1}{x(x^2+1)^2}=- \frac{1}{x}+\frac{2x+2}{x^2+1}+\frac{5x-2}{(x^2+1)^2}$
	\end{example}
	В некоторых случаях для нахождения неопределенных коэффициентов можно воспользоваться так называемым методом частных значений аргумента. Он состоит в том, что аргументу $x$ придаются конкретные числовые значения столько раз, сколько содержится неизвестных коэффициентов в разложении. При этом удобно выбирать $x$ равным значению действительного корня знаменателя. 
	\begin{example}
		$\frac{3x-4}{x(x-2)(x+1)}=\frac{A}{x}+\frac{B}{x-2}+\frac{C}{x+1}\\ A=\frac{3x-4}{(x-2)(x+1)}\Big|_{x=0}=\frac{-4}{-2\cdot 1}=2\\ B=\frac{3x-4}{x(x+1)}\Big|_{x=2}=\frac{6-4}{2\cdot 3}=\frac{1}{3}\\C=\frac{3x-4}{x(x-2)}\Big|_{x=-1}=\frac{-3-4}{-1\cdot (-3)}=-\frac{7}{3}$
	\end{example}
	\begin{example}
		$\frac{x^2+1}{x(x-1)^2}=\frac{A}{x}+\frac{B}{x-1}+\frac{C}{(x-1)^2}\\\\ A=\frac{x^2+1}{(x-1)^2}\Big|_{x=	0}=1\\ C=\frac{x^2+1}{x}\Big|_{x=1}=2\\\\$ при $x=2:\\B=\frac{5}{2}-\frac{1}{2}-2=0$ 
	\end{example} 
\end{enumerate}
\subsection{Интегрирование простейших дробей}
\begin{enumerate}[I.]
	\item $\int \frac{A}{x-a}\dx=A\int\frac{\dy (x-a)}{x-a}=A\ln |x-a|+c$
	\item $\int\frac{A}{(x-a)^k}\dx=A\int(x-a)^{-k}\dy (x-a)=A\frac{(x-a)^{-k+1}}{-k+1}+c$
	\item $\int\frac{Mx+N}{x^2+px+q}\dx=\int\frac{Mx+N}{x^2+2x\cdot{p\over 2}+{p^2\over 4}-{p^2\over4}+q}\dx=M\int \frac{x+\frac p2-\frac p2}{(x+\frac p2)^2+q-{p^2\over 4}}\dy(x+\frac p2)+\\+N\int\frac{\dy (x+\frac p2)}{(x+\frac p2)^2+q-{p^2\over 4}}=M\int \frac{(x+\frac p2-\frac p2)\dy (x+\frac p2)}{(x+\frac p2)^2+q-{p^2\over 4}}+(N-\frac{Mp}2)\int\frac{\dy (x+\frac p2)}{(x+\frac p2)^2+q-{p^2\over 4}}=\vcases{(x+\frac p2)=t \cr q-\frac{p^2}4=a^2}\bigg|=M\int \frac{t\dy t}{t^2+a^2}+(N-\frac{Mp}2)\int\frac{\dy t}{t^2+a^2}=\frac M2(\int\frac{\dy(t^2+a^2)}{(t^2+a^2)^k})+(N-\frac{Mp}2)\cdot I_k=\frac M2\frac{(t^2+a^2)^{-k+1}}{-k+1}+\\+(N-\frac{Mp}2)\cdot I_k\\\\$ Найдем $I_k=\int \frac{\dy t}{(t^2+a^2)^k}=\vcases{U=\frac{1}{(t^2+a^2)^k}\Rightarrow \dy U=-k(t^2+a^2)^{-k-1}\cr \dy V=\dy t;\>V=t;\>2t\dy t=-2k\frac{t\dy t}{(t^2+a^2)^{k+1}}}\Bigg|=\\=\frac{t}{(t^2+a^2)^k}+\int t\cdot 2k\frac{t\dy t}{(t^2+a^2)^k}=\frac{t}{(t^2+a^2)^k}+2k\int\frac{(t^2+a^2-a^2)\dy t}{(t^2+a^2)^{k+1}}=\frac{t}{(t^2+a^2)^k}+\\+2k\int\left( \frac{1}{(t^2+a^2)^k}-\frac{a^2}{(t^2+a^2)^{k+1}} \right)\dy t=\frac{t}{t^2+a^2}+2k\left(I_k-a^2I_{k+1}\right)=\\=\frac{t}{t^2+a^2}+2kI_k-2ka^2I_{k+1}\Rightarrow2ka^2I_{k+1}=\frac{t}{(t^2+a^2)^k}+I_k(2k-1);\\\\$ Пусть $k+1=n\Rightarrow k=n-1\\$ Получим: $I_n=\frac{1}{a^2(2n-2)}\cdot\frac{t}{(t^2+a^2)^{n-1}}+\frac{2n-3}{2n-2}\cdot\frac 1{a^2}\cdot I_{n-1};\>n>2$
\end{enumerate}
\subsection{Общая схема интегрирования рациональных дробей}
\begin{enumerate}
	\item Если дробь неправильная, то разделить числитель на знаменатель и выделить целую часть(т. е. представить дробь в форме многочлена и правильной рациональной дроби).
	\item Знаменатель правильной рациональной дроби раскладываем на множители и записываем разложение правильной дроби в сумму простейших дробей.
	\item Находим неопределенные коэффициенты этого разложения.
	\item Интегрируем полученный многочлен и сумму полученных дробей.	  
\end{enumerate}
\begin{remark}
Интеграл от рациональной функции всегда выражается через элементарные функции.	
\end{remark}
\begin{example}
	$$\int \frac{x^5+2x^3+4x+4}{x^4+2x^3+2x^2}\dx=$$
		\[
    		\polylongdiv{x^5+0x^4+2x^3+0x^2+4x+4}{x^4+2x^3+2x^2}
  		\]
	$$=\int \left((x-2)+\frac{4x^3+4x^2+4x+4}{x^4+2x^3+2x^2}\right)\dx=$$
	\begin{equation}\label{*}
		\frac{4x^3+4x^2+4x+4}{x^4+2x^3+2x^2}=\frac{4x^3+4x^2+4x+4}{x^2(x^2+2x+2)}=\frac Ax+\frac B{x^2}+\frac{Cx+D}{x^2+2x+2}
	\end{equation}
	$
	B=\frac{4x^3+4x^2+4x+4}{x^2+2x+2}\>\bigg|_{x=0}=2
	$\\ при $x=1:(\ref{*})\\ \frac {16}{5}=A+2+\frac{C+D}{2}\big|\cdot 5;\tab  16=5A+10+C+D; \tab 5a+C+D=6\\\\$
	при $x=-1:\\ 0= -A+2+D-C;\tab A+C-D=2;\tab A+C-D=2\\\\$
	при $x=-2:\\ \frac{-32+16-8+4}{16-16+8}=-\frac A2+\frac24+\frac{D-2C}{2}\big|\cdot2;\tab -5=-A+1+D-2C;\Rightarrow\\\\\Rightarrow A=0;\>B=2;\>C=4;\>D=2$
	$$\int \left(x-2 +\frac2{x^2}+\frac{4x+2}{x^2+2x+2} \right)\dx=\frac2{x^2}-2x+2\int \frac{2x+2-1}{(x^2+2x+2)}\dx=$$
	$$=\frac2{x^2}-2x-\frac2x+2\left(\underset{\ln(x^2+2x+2)}{\int \frac{\dy (x^2+2x+2)}{x^2+2x+2}}-\underset{\arctan (x+1)}{\int\frac{\dy (x+1)}{(x+2)^2+1}}\right)=$$
	$$\frac2{x^2}-2x-\frac2x+2\ln(x^2+2x+2)-2\arctan (x+1)$$
\end{example}
\section{Интегрирование тригонометрических функций}
\subsection{Универсальная тригонометрическая замена}
	
Пусть $R(\sin x; \cos x)$ - рациональная функция от $\sin x, \cos x.\\$ \textbf{Замена:} $t=\tan \frac x2\\$ 
\textbf{Тогда:}
$$\sin x=2\sin \frac x2 \cos\frac x2=2\tan \frac x2\cos\frac{x^2}{2}=\frac{2\tan\frac x2}{1+\tan^2\frac x2}=\frac{2t}{1+t^2}$$
$$\cos x=2\cos^2\frac x2 -1=\frac{2}{1+\tan^2\frac x2}-1=\frac{2}{1+t^2}-1=\frac{1-t^2}{1+t^2}$$
$x=2\arctan t;\tab \dx= \frac{2}{1+t^2}\dy t\\$
\textbf{Получаем:}
$$\int R(\sin x, \cos x)\dx=\int R\left(\frac{2}{1+t^2},\frac{1-t^2}{1+t^2}\right)\cdot\frac{2}{1+t^2}\dy t=\int R_1(t)\dy t$$
\begin{remark}
	Этот способ позволяет найти первообразную, но полученная функция $f(t)$ может оказаться слишком громаздкой.
\end{remark}
\subsection{Другие виды подстановок }
\begin{enumerate}
	\item Если подинтегральная функция является нечетной относительно $\sin x$, т. е. $R(-\sin x,\cos x)=-R(\sin x, \cos x)$, то используется замена $t=\cos x$. Фактически это означает внесения $\cos x$ под знак дифференциала. 
	\item Если подинтегральная функция является нечетной относительно $\cos x$, т. е. $R(\sin x,-\cos x)=-R(\sin x, \cos x)$, то используется замена $t=\sin x$. Фактически это означает внесения $\sin x$ под знак дифференциала. 
	\item Если подинтегральная функция является одновременно четной относительно $\sin x,\cos x$ то выполняется замена $t=\tan x$(внесение $\frac1{\cos^2x}$ под знак дифференциала).
\end{enumerate}
\begin{remark}
	$\\$Для $\int R(\tan x)\dx$ замена $\tan x=t \Rightarrow x=\arctan t ;\> \dx=\frac{\dy t}{1+t^2};\\$ и $\int R(\tan x)\dx=\int R(t)\cdot \frac{\dy t}{1+t^2}=\int R_1(t)\dy t$
\end{remark}
\subsection{Использования формул тригонометрии}
\begin{enumerate}
	\item $$\int\cos^2x\dx, \int\sin^2x\dx\Rightarrow \cos^2x=\frac{1+\cos2x}{2}, \sin^2x=\frac{1-\cos2x}{2}$$
	\item $$\int \cos\alpha x\cos\beta x\dx\Rightarrow\cos\alpha x\sin\beta x=\frac12 (\cos(\alpha-\beta)x+\cos(\alpha+\beta)x)$$
	$$\int \sin\alpha x\sin\beta x\dx\Rightarrow\sin\alpha x\sin\beta x=\frac12 (\cos(\alpha-\beta)x-\cos(\alpha+\beta)x)$$
	$$\int \sin\alpha x\cos\beta x\dx\Rightarrow\sin\alpha x\cos\beta x=\frac12 (\sin(\alpha-\beta)x+\sin(\alpha+\beta)x)$$
\end{enumerate}
\begin{example}
	$\int \frac{\dx}{3+\sin x+\cos x}=\left|\begin{array}{ccc}
		t=\tan \frac x2\\\sin x=\frac{2t}{1+t^2}\\\cos x=\frac{1-t^2}{1+t^2} \
	\end{array}\right|=\int \frac{2\dy t}{(1+t^2)(3+\frac{2t}{1+t^2}+\frac{1-t^2}{1+t^2})}=2\int \frac{\dy t}{3+3t^2+2t+1-t^2}=\\=2\int\frac{\dy t}{2t^2+2t+4}=2\int\frac{\dy t}{t^2+t+2}=\int\frac{\dy t}{(t+\frac12)^2+\frac74}=\int\frac{\dy(t+\frac12)}{(t+\frac12)^2+(\frac{\sqrt7}{2})^2}=\frac{2}{\sqrt7}\arctan(2\frac{t+\frac12}{\sqrt7})+c=\\=\frac{2}{\sqrt7}\arctan(2\frac{\tan x+\frac12}{\sqrt7})+c$
\end{example}
\begin{example}
	$\int \frac{\dx}{1+\sin^2x}=\frac{1}{\frac{1}{\cos^2x}+\tan^2x}\cdot\frac{\dx}{\cos^2x}=\left| \cos^2x=\frac{1}{1+\tan^2x}\right|=\int\frac{1}{1+2\tan^2x}\dy(\tan x)=|\tan x=t|=\int\frac{\dy t}{1+2t^2}=\frac1{\sqrt2}\int\frac{\dy (\sqrt2t)}{1+(\sqrt2t)^2}=\frac1{\sqrt2}\arctan(\sqrt2 t)+c=\\=\frac1{\sqrt2}\arctan(\sqrt2 \tan x)+c$
\end{example}
\begin{example}
	$\int\cos^2x\sin^4x\dx= \int\frac{1+\cos2x}{2}\left(\frac{1-\cos2x}{2} \right)^2\dx=\frac18\int(1-\cos ^22x)(1-\cos2x)\dx=\\=\frac18\int(1-\cos2x-\cos^2x+\cos^32x)\dx=\frac18(x-\frac{\sin2x}{2}-\int \cos^22x\dx+\int\cos^32x\dx)=\\=\frac18 x-\frac{\sin2x}{16}-\frac{x}{16}-\frac{\sin4x}{64}-\frac{1}{16}\int(1-\sin^2x)\dy(\sin2x)=\frac{x}{16}-\frac{\sin2x}{16}-\frac{\sin4x}{16}+\frac{\sin2x}{16}-\\-\frac{1}{16}\frac{\sin^32x}{3}+c=\frac{x}{16}-\frac{\sin4x}{16}-\frac{1}{16}\frac{\sin^32x}{3}+c$
\end{example}
\begin{example}
$\int\sin^4x\cos^5x\dx=|\sin x=t|=\int\sin^4x\cos^4x\underbrace{\cos x\dx}_{\dy(\sin x)}=\int\sin^4x(\cos^2x)^2\dy(\sin x)=\\=\int\sin^4x(1-\sin^2x)^2\dy(\sin x)=\int t^4(1-2t^2-t^4)\dy t=\int(t^4-2t^6+t^8)\dy t=\\=\frac{t^5}{5}-\frac{2t^7}{7}+\frac{t^9}{9}+c=\frac{\sin^5x}{5}-\frac{2\sin^7x}{7}+\frac{\sin^9x}{9}+c$	
\end{example}
\section{Интегрирование некоторых иррациональных и транцедентных функций}
\subsection{Дробно-линейная подстановка для интегралов}
$\int R(x;\left(\frac{ax+b}{cx+d}\right)^{\frac{m_1}{n_1}},\>\left(\frac{ax+b}{cx+d}\right)^{\frac{m_2}{n_2}},\dots,\>\left(\frac{ax+b}{cx+d}\right)^{\frac{m_k}{n_k}}\dx,\\ a,\>b,\>c,\>d\in\mathbb{R};\tab m_1,\>n_1,\dots,\>m_k,\>n_k\in\mathbb{N}$
\begin{remark}$ $
	\begin{enumerate}
		\item $\left(\frac{ax+b}{cx+d}\right)^{\frac{m}{n}}=\sqrt[n]{\left(\frac{ax+b}{cx+d}\right)^{{m}}}$
		\item Частичными случаи таких дробей являются
			$\\ax+b(c=0,\>d=1),\tab x=(c=0,\>d=1,\>b=0,\>a=1)\\$
			\textbf{Замена:}
			$\\\frac{ax+b}{cx+d}=t^l,$ где $l=\lcm(n_2,\>n_2,\dots,\>n_k)\Rightarrow\left(\frac{ax+b}{cx+d}\right)^{\frac{m_i}{n_i}}=t^{\frac{m_i}{n_i}l}=t^{p_i},\>p_i\in\mathbb{Z}\\ax+b=t^lcx+dt^l;\>x=(t^lc-a)=b-d\cdot t^l\Rightarrow x=\frac{b-dt^l}{ct^l-a}\\\dx=\left(\frac{b-dt^l}{ct^l-a}\right)_t'\dy t=\frac{-d l\cdot t^{l-1}(ct^l-a)-(b-d t^l)\cdot c\cdot l\cdot t^{l-1}}{(ct^l-a)^2}=\frac{-lt^{l-1}(cd t^l-ad +bc-cdt^l}{(ct^l-a)^2}=\frac{-lt^{l-1}(bc-ad)}{(ct^l-a)^2}$
			Таким образом, подинтегральная функция будет являтся рациональной функцией от $t$.
	\end{enumerate}
\end{remark}
\begin{example}
	$\int\frac{\dx}{\sqrt[3]{(2x+1)^2}-\sqrt{2x+1}}=\left|\begin{array}{lcr}
		n_1=3; & \textrm{Замена:} & 2x+1=t^6\\
		n_2=2 && x=\frac{t^6-1}{2}\\
		\lcm = 6 && \dx=3t^5\dy t
	\end{array}\right|=\int \frac{3t^5\dy t}{(t^6)^{\frac 23}+(t^6)^{\frac 12}}=3\int \frac{t^5\dy t}{t^4 -t^3}=\\=3\int \frac{t^2}{t-1}\dy t\underset{\textrm{выделяем целую часть}}{\Longrightarrow}\polylongdiv{t^2}{t-1}=\\=3\int (t+1+\frac1{t-1})\dy t=3\left(\frac {t^2}{2}+t+\ln|t-1|\right)+c=|t=\sqrt[6]{2x+1}|=\\=\frac {3}{2}\sqrt[3]{2x+1}+3\sqrt[6]{2x+1}+3\ln|\sqrt[6]{2x+1}-1|+c$
\end{example}
\begin{example}
	$\int\frac{1}{(x-1)^2\sqrt[3]{\frac{x+1}{x-1}}}\dx=\left|\begin{array}{lr}
		\frac{x+1}{x-1}=t^3\Rightarrow & \dx =\left(\frac{t^3+1}{t^3-1}\right)\dy t\\
		x+1=xt^3-t^3 \\
		x=\frac{t^3+1}{t^3-1} & =\frac{-6t^2\dy t}{(t^3-1)^2}
	\end{array}\right|=\\=\int\frac{1}{\left(\frac{t^3+1}{t^3-1}-1\right)^2}\cdot t\cdot \frac{-6t^2\dy t}{(t^3-1)^2}=-6\int \frac{1\cdot t^3\dy t}{\frac{2^2}{(t^3-1)}}\cdot(t^3-1)^2=-\frac 64\cdot\frac{t^4}4=-\frac 38\cdot\left(\frac{x+1}{x-1}\right)^{\frac 43}+c$
\end{example}
\subsection{Квадратичные иррациональности}
\begin{enumerate}
	\item Частные случаи \begin{enumerate} 
	\item Интегралы вида $\int R(x,\>\sqrt{x^2\pm a^2})\dx,\>\int R(x,\>\sqrt{a^2-x^2})\dx\\ R()$ - знак рациональной функции. \\
		Для преобразования таких интегралов к интегралам рациональной функции испльзуется замена.\begin{itemize}
			\item [-] для $R(x,\>\sqrt{a^2-x^2})\>\>:\>\>x=a\sin t$ 
			\item [-] для $R(x,\>\sqrt{x^2+ a^2})\>\>:\>\>x=at\tan t;\>(1+\tan^2t=\frac{1}{\cos^2t})$
			\item [-] для $R(x,\>\sqrt{x^2- a^2})\>\>:\>\>x=\frac{a}{\sin t}$

		\end{itemize} 
	\item Интегралы видa $\int\frac{\dx}{\sqrt{ax^2+bx+c}};\>\>\int \sqrt{ax^2+bx+c}\dx;\>\int\frac{(mx+n)\dx}{\sqrt{ax^2+bx+c}}$ можно свести к табличному или к пункту (а) выделением полного квадрата. \\ $ax^2+bx+c=a\left(x^2+\frac ba x+\frac ca\right)=a\left(x^2+2\cdot x\cdot\frac b{2a}+\frac{b^2}{4a^2}- \frac{b^2}{4a^2}+\frac ca \right)=\\=a\left(x^2+\frac ba \right)^2+c-\frac{b^2}{4a};$ замена: $\left(x+\frac b{2a}\right)=t$
	\item Интегралы видa $\int \frac{P_n(x)\dx}{\sqrt{ax^2+bx+c}},\>P_n(x)$ - многочлен $n$-ой степени, можно вычислить поп формуле: $\\\int \frac{P_n(x)\dx}{\sqrt{ax^2+bx+c}}=Q_{n-1}(x)\cdot\sqrt{ax^2+bx+c}+\lambda\int\frac{\dx}{\sqrt{ax^2+bx+c}},$ где $Q_{n-1}(x)$ - многочлен с непоределенными коэффициентами. $\lambda$ - неопределенный коэффициент. Неопределенный коэффициенты находим, дифференцируя обе части этой формулы и умножая полученный результат на знаменатель. $\frac{P_n(x)}{\sqrt{ax^2+bx+c}}=Q_{n-1}'(x)\cdot\sqrt{ax^2+bx+c}+Q_{n-1}(x)frac{2ax+b}{\sqrt{ax^2+bx+c}}+\lambda\cdot\frac1{\sqrt{ax^2+bx+c}}=$ умножаем на $\sqrt{ax^2+bx+c},$ из полученного находим неопределенные коэффициенты.
	\newpage
	\begin{example}
		$\int\frac{\dx}{\sqrt{x^2-a^2}}=\left|\begin{array}{l}
			\>\>x=\frac{a}{\sin t}\\
			\dx = -\frac{a\cos t}{\sin t}\dy t
		\end{array}\right|=\int\frac{-a\cos t\dy t}{\sqrt{\frac{a^2}{\sin^2t}-a^2}\cdot\sin t}=-\int \frac{\cos t\dy t}{\sqrt{\frac{1-\sin^2t}{\sin^2t}}\cdot\sin^2t}=\\=-\int\frac{\dy t}{\sin t}\cdot \frac{\sin t}{\sin t}=\int\frac{\dy(\cos t)}{1-\sin^2t}=\frac12\ln\left|\frac{1-\cos t}{1+\cos t}\right|+c=\\=\left|\left(\begin{array}{l}
			\sin t=\frac{a}{x};\\t=\arcsin \frac ax
		\end{array}\right)\begin{array}{l}
			\cos t=\sqrt{1-\sin^2t}\\\sqrt{1-\frac{a^2}{x^2}}=\sqrt{\frac{x^2-a^2}{x^2}}
		\end{array} \right|=\frac12\ln\left|\frac{1-\frac{\sqrt{x^2-a^2}}{x}}{1+\frac{\sqrt{x^2-a^2}}{x}} \right|+c=\\=\frac12\ln\left|\frac{x-\sqrt{x^2-a^2}}{x+\sqrt{x^2-a^2}} \right|+c$
	\end{example}
	\begin{example}
		$\int\frac{x+4}{\sqrt{6-2x-x^2}}\dx=\left|\begin{array}{l}
			6-2x-x^2=-((x^2+2x+1)-7)=-(x+1)^2+7=\\=7-(x+1)^2=(\sqrt7)^2-(x+1)^2
		\end{array}\right|=\\=|x+1=t|=\int\frac{t+3}{\sqrt{(\sqrt7)^2+(t^2)}}\dy t=\int\frac{t\dy t}{\sqrt{7-t^2}}+3\int\frac{\dy t}{\sqrt{7-t^2}}=-\frac12\int\frac{\dy(7-t^2)}{\sqrt{7-t^2}}+\\+3\arcsin\frac t{\sqrt{7}}=-\sqrt{7-t^2}+3\arcsin\frac t{\sqrt7}+c=-\sqrt{6-2x-x^2}+\\+3\arcsin\frac{x+1}{\sqrt7}+c$
	\end{example}
	\begin{example}
		$\int\frac{x^2\dx}{\sqrt{1-2x-x^2}}=(Ax+B)\sqrt{1-2x-x^2}+\lambda\left.\int\frac{\dx}{\sqrt{1-2x-x^2}}\right|'=\frac{x^2}{\sqrt{1-2x-x^2}}=\\=A\sqrt{1-2x-x^2}+(Ax+B)\cdot\frac{-2-2}{2\sqrt{1-2x-x^2}}+\left.\frac{\lambda}{\sqrt{1-2x-x^2}}\right|\cdot\sqrt{1-2x-x^2}\\x^2=A-2Ax-Ax^2-Ax^2-Ax-Bx-B+\lambda\\\begin{cases}
			-2A=1\\-3A+B=0\\A-B+\lambda=0
		\end{cases}\begin{cases}
			A=-\frac12\\B=\frac32\lambda=2
		\end{cases}\\\underbrace{\left(-\frac12x+\frac32\right)\sqrt{1-2x-x^2}}_{F(x)}+2\int\frac{\dx}{\sqrt{2-(x+1)^2}}=F(x)+2\arcsin\frac{x+1}{\sqrt2}+c$
	\end{example}
	\end{enumerate}
	\item Интегралы общего вида \\ $\int R(x,\>\sqrt{ax^2+bx+c})\dx$
		\begin{itemize}
			\item [Способ 1] Выделим под знаком радикала полный квадрат и выполняем замену: $x+\frac b{2a}=t;$ интеграл сводится к к одному из интегралов \\$\int R(t,\>\sqrt{t^2\pm a^2})\dy t,\int R(t,\>\sqrt{a^2-t^2})\dy t,$ которые находятся при помощи тригонометрической подстановки.\newpage
			\item [Способ 2] Использование подстановки Эйлера 
				\begin{itemize}
					\item [-] eсли $a>0$, то $\sqrt{ax^2+bx+c}=\pm\sqrt a x+t:\>(t\pm\sqrt a x)$ 
					\item [-] eсли $c>0$, то $\sqrt{ax^2+bx+c}=tx\pm \sqrt c$
					\item [-] eсли $D>0$, то $\sqrt{ax^2+bx+c}=(x-\alpha)\cdot t,$\\ где $\alpha$ - корень $ax^2+bx+c|_{x=\alpha}=0.$
				\end{itemize}
			\end{itemize}
			\begin{remark}
			По крайней мере одно из условий будет выполнено всегда; т.к. ситуация $\begin{cases}
				a<0\\c<0\\D<0
			\end{cases}\Rightarrow ax^2+bx+c<0$ запрещена по ОДЗ.	
			\end{remark}
			\begin{example}
				$\int\frac{\dx}{(1+x)\sqrt{1-x-x^2}}=\left|\begin{array}{l}
					a<\\D>0,\\ \textrm{корни иррац.}\\c>o
				\end{array} \right|\textcircled{=}\\$ Замена: $\sqrt{1-x-x^2}=tx+1;\\1-x-x^2=tx^2+2tx+1|:x,\tab -1-x=t^2x+2t\\x(t^2+1)=-2t-1,\tab x=-\frac{2t+1}{t^2+1},\tab t=\frac{\sqrt{1-x-x^2}-1}{x}\\\dx =\left(-\frac{2t+1}{t^2+1} \right)'\dy t=-\frac{2(t^2+1)-(2t+1)\cdot 2t}{(t^2+1)^2}\dy t=\frac{2t^2+2t-2}{(t^2+1)^2}\dy t\\\textcircled{=}\int \frac{2(t^2+t-1)\dy t}{(t^2+1)^2\left(\frac{1-2t+1}{t^2+1}\right)\left(t\cdot \frac{-(2t+1)}{t^2+1}+1\right)}=\int\frac{2(t^2+t-1)\dy t}{t(t-2)(-t^2-t+1)}=-2\int\frac{\dy t}{t(t-2)}=\\=-2\int \left(-\frac 12\cdot\frac1t +\frac12\cdot\frac1{t-2}\right)\dy t=\int \left(\frac1t+\frac1{t-2}\right)\dy t=\ln|t|-\ln|t-2|+c=\\=\ln\left|\frac{\sqrt{1-x-x^2}-1}{x}\right|-\ln\left|\frac{\sqrt{1-x-x^2}-1}{x}-2\right|+c=\ln\left|\frac{\frac{\sqrt{1-x-x^2}-1}{x}}{\frac{\sqrt{1-x-x^2}-1}{x}-2}\right|+c$
			\end{example}
\end{enumerate}
\subsection{Инегрирование дифферециального бинома}
Интеграл $\int x^m(a+bx^n)^p\dx,\>\>m,\>n,\>p\in\mathbb{Q};\>\>a,\>b\in\mathbb{R}\\$ этот интеграл сводится к к интегралу от рациональной функции и первообразная выражается в элементарных функциях только в следующих случаях: \begin{enumerate}
	\item $p\in\mathbb{Z}$
	\item $\frac{m+1}{n}\in\mathbb{Z}$
	\item $\left(\frac{m+1}{n}+p\right)\in\mathbb{Z}$ 
\end{enumerate}   
При этом для сведения заданого интеграла к интегралу от рациональной функции используются подстановки: \begin{enumerate}
	\item $p\in\mathbb{Z}\tab $ Замена: $x=t^k,$ где $k=\lcm(m,\>n)$
	\item $\frac{m+1}{n}\in\mathbb{Z}\tab $ Замена: $a+bx^n=t^s,$ где $s$ - знам. $p$
	\item $\left(\frac{m+1}{n}+p\right)\in\mathbb{Z}\tab $ Замена: $a+bx^n=t^sx^n,$ где $s$ - знам. $p$
\end{enumerate}
В остальных случиях первообразная в элементарных функциях не выражается. Этот результат носит название теоремы Чебышева. 
\begin{example}
	$\int\frac{\sqrt[3]{\sqrt[4]{x}+1}}{\sqrt x}\dx=\int x^{-\frac12}(1+x^{\frac14})^{\frac14 }\dx=\left|\begin{array}{l}
		m=-\frac12\\n=\frac14\\p=\frac13
	\end{array}\right|\textcircled{=}$\begin{enumerate}
		\item $p\not\in\mathbb{Z}$
		\item $\frac{m+1}{n}=\frac{1\cdot 4}{2\cdot 1}=2\in\mathbb{Z}$ - второй случай. \\ Замена: $1+x^{\frac14}=t^3,\>x=(t^3-1)^4;\>\dx=4(t^3-1)^3\cdot3t^2\dy t$
	\end{enumerate}
	$\textcircled{=}\int\left((t^3-1)^4\right)^{-\frac12}(t^3)^{\frac13}\cdot 12t^2(t^3-1)^3\dy t=12\int(t^3-1)t^3\dy t=12\left(\frac{t^7}4-\frac{t^4}4\right)+c=\\=\frac{12}7\left(1+x^{\frac14}\right)^{\frac73}-3\left(1+x^{\frac14}\right)^{\frac43}+c$
\end{example}
\subsection{Интегралы вида $\int R(e^x)\dx,\int R(\sqrt{e^x+e})\dx$}
\begin{enumerate}
	\item Замена: $e^x=t\Rightarrow x=\ln t,\dx=\frac{\dy t}t\\\int R(e^x)\dx=\int R(t)\frac{\dy t}{t}=\int R_1(t)\dy t$
	\item Замена: $\sqrt{e^x+e}=t\Rightarrow e^x=t^2-a,\>x=\ln (t^2-a),\dx=\frac{2t\dy t}{t^2-a}\\\int R(\sqrt{e^x+e})\dx=\int R(t)\cdot \frac{2t\dy t}{t^2-a}=\int R_1(t)\dy t$
\end{enumerate}
\section{Интегралы, не выражающиеся в элементарных функциях}
В интегральном исчислении строго доказывается, что первообразные от некоторых элементарных функций хотя и существуют, но не могут быть выражены элементарной функцией (т.е. как конечное число арифметических операций и композиций над основными элементарными функциями(даже если известно, что первообразная существует)).\\
К таким интегралам относятся:
$$\int e^{-x^2}\dx\textrm{ интеграл Пуассона(теория вероятностей)}$$
$$\int \frac{\dx}{\ln x}\textrm{ интегральный логарифм(теория чисел)}$$
$$\int\frac{\sin x}{x}\dx,\int\frac{\cos x}{x}\dx,\int\frac{e^x}{x}\dx\textrm{ интегральный синус, косинус, показательная функция}$$
$$\int \sqrt{1-k^2\sin^2x}\dx,\int \frac{\dx}{\sqrt{1-k^2\sin^2x}},|k|<1\textrm{ элиптические интегралы}$$
$$\int \cos(x^2)\dx,\int \sin(x^2)\dx\textrm{ интегралы Фринеля(физика, оптика)}$$
$$\int x^\alpha\sin x\dx,\int x^\alpha\cos x\dx,\>\alpha\not=0,1,2\dots\tab\textrm{и другие}$$
Такие функции называются специальными. Для них существуют специальные таблицы для определения значений функции.  

























\end{document}