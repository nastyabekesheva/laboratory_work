\documentclass[a4paper,12pt]{article}
\usepackage[ukrainian,english]{babel}
\usepackage{ucs}
\usepackage[utf8]{inputenc}
\usepackage[T2A]{fontenc}
\usepackage{amsmath}
\usepackage{amsfonts}
\usepackage{graphicx}
\newcommand\tab[1][1cm]{\hspace*{#1}}
\usepackage[left=20mm, top=20mm, right=10mm, bottom=20mm, nohead, nofoot]{geometry}
\begin{document}
\begin{center}
{\LARGE Домашня робота 2}	
\end{center}
1.2(b) Довести, що $(n^2+(n+1)^2)\mod 4 = 1:\\$
\tab $n^2+(n+1)^2 = 2n^2+2n+1=2\cdot n(n+1)+1;\>\>2n(n+1)\>\vdots\>2\Rightarrow n(n+1)\>\vdots\>2?\\\tab n(n+1)\>\vdots\>2$ як два послідовні натуральні числа $\Rightarrow 2n(n+1)\>\vdots\>4+1\\\tab\Rightarrow (n^2+(n+1)^2)\mod 4 = 1\\$
1.3(b) Довести, що $p^2\mod 24 = 1,\>\>p\geq 5:$
\tab 
\\
1.4(b) Довести, що числа виду $2^{4^n}-5,\>n\geq 1$ закінчуються на 1:\\
\begin{enumerate}
	\item $n=1:\>2^{4^1}-5=16-5=11$ - ok
	\item Нехай умова виконуєтся для $n\Rightarrow 2^{4^n}$ - закінчується на 6.$\\(x\cdot 10+1 - 5=(x-1)10+10+1-5=(x-1)10+6)$ 
	\item Доведемо для $n+1:\\ 2^{4^{n+1}}-5=2^{4^n\cdot 4}=\left(\underbrace{2^{4^n}}_{\textrm{зак. на 6}}\right)^4-5;$
\end{enumerate}

\end{document}